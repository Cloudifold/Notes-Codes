\documentclass[a4paper]{article}
\usepackage[dvipsnames, svgnames, x11names]{xcolor}
\usepackage{tikz-cd}
\tikzcdset{row sep/normal=1cm}
\tikzcdset{column sep/normal=1cm}
\usepackage{amsfonts}
\usepackage{hyperref}
\usepackage{amsthm}
\usepackage{enumerate}
\usepackage[left=3cm, right=3cm, top=2cm]{geometry}
\usepackage{sectsty}
\usepackage[perpage]{footmisc}
\usepackage{amsthm}
\usepackage{tikz-cd}
\usepackage{pgfplots}
\usepackage{amsmath}
\usepackage{amssymb}
\usepackage{pdfpages}
\usepackage{multirow}
\usepackage{mathtools}
\usepackage{mathrsfs}
\usepackage{ebproof}
\usepackage{graphicx}
\usepackage{epigraph}
\usepackage{enumitem}
\usepackage{indentfirst}

\usepackage{quiver}


\theoremstyle{plain}
\newtheorem{thm}{Theorem}[section]
\newtheorem{lem}[thm]{Lemma}
\newtheorem{prop}[thm]{Proposition}
\newtheorem*{cor}{Corollary}

\theoremstyle{definition}
\newtheorem{defn}{Definition}[section]
\newtheorem{conj}{Conjecture}[section]
\newtheorem{exmp}{Example}[section]
\newtheorem{intro}[defn]{Introduction}

\newtheorem*{prf}{Proof}

\theoremstyle{remark}
\newtheorem*{rem}{Remark}
\newtheorem*{note}{Note}
\newtheorem{exercise}{Exercise}[section]
\newtheorem{solution}{Solution}[section]

\definecolor{background}{RGB}{255,230,230}
\pagecolor{background}

\tikzcdset{background color = background}

\definecolor{TextColor}{RGB}{255,122,162}

\chapterfont{\color{TextColor}}
\sectionfont{\color{TextColor}\Huge}
\setlist{nolistsep,topsep=-7pt}
\pgfplotsset{width=10cm,compat=newest}
%\usepgfplotslibrary{external}
%\tikzexternalize[prefix=tikz/]
\newcommand{\ssubparagraph}{ \parshape 1 1cm \dimexpr\linewidth-2cm\relax}
%\numberwithin{equation}{section}

% \quad : width of  1 'M'
% \qquad : 4
% \0x20 (\ ) : 1/3
% \; : 2/7
% \, : 1/6
% \! : -1/6

% \addtocounter{<envname>}{nubmer}
% Expl. \addtocounter{thm}{2}
% \setcounter{<envname>}{nubmer}
% Expl. \setcounter{ex}{3}
% -->  Exercise 1.4

\newcommand{\setexcounter}[1]{
    \setcounter{ex}{#1}
    \setcounter{sol}{#1}
}
\newcommand{\setsubsectcounter}[1]{\setcounter{subsection}{#1}}

\newcommand{\bb}[1]{\mathbb{#1}}
\newcommand{\mc}[1]{\mathcal{#1}}
\newcommand{\mf}[1]{\mathfrak{#1}}
\newcommand{\ms}[1]{\mathscr{#1}}
\newcommand{\mbf}[1]{\mathbf{#1}}
\newcommand{\bbA}{\mathbb A}
\newcommand{\bbB}{\mathbb B}
\newcommand{\bbC}{\mathbb C}
\newcommand{\bbD}{\mathbb D}
\newcommand{\bbE}{\mathbb E}
\newcommand{\bbF}{\mathbb F}
\newcommand{\bbG}{\mathbb G}
\newcommand{\bbH}{\mathbb H}
\newcommand{\bbI}{\mathbb I}
\newcommand{\bbJ}{\mathbb J}
\newcommand{\bbK}{\mathbb K}
\newcommand{\bbL}{\mathbb L}
\newcommand{\bbM}{\mathbb M}
\newcommand{\bbN}{\mathbb N}
\newcommand{\bbO}{\mathbb O}
\newcommand{\bbP}{\mathbb P}
\newcommand{\bbQ}{\mathbb Q}
\newcommand{\bbR}{\mathbb R}
\newcommand{\bbS}{\mathbb S}
\newcommand{\bbT}{\mathbb T}
\newcommand{\bbU}{\mathbb U}
\newcommand{\bbV}{\mathbb V}
\newcommand{\bbW}{\mathbb W}
\newcommand{\bbX}{\mathbb X}
\newcommand{\bbY}{\mathbb Y}
\newcommand{\bbZ}{\mathbb Z}
\newcommand{\mcA}{\mc A}
\newcommand{\mcB}{\mc B}
\newcommand{\mcC}{\mc C}
\newcommand{\mcD}{\mc D}
\newcommand{\mcE}{\mc E}
\newcommand{\mcF}{\mc F}
\newcommand{\mcG}{\mc G}
\newcommand{\mcH}{\mc H}
\newcommand{\mcI}{\mc I}
\newcommand{\mcJ}{\mc J}
\newcommand{\mcK}{\mc K}
\newcommand{\mcL}{\mc L}
\newcommand{\mcM}{\mc M}
\newcommand{\mcN}{\mc N}
\newcommand{\mcO}{\mc O}
\newcommand{\mcP}{\mc P}
\newcommand{\mcQ}{\mc Q}
\newcommand{\mcR}{\mc R}
\newcommand{\mcS}{\mc S}
\newcommand{\mcT}{\mc T}
\newcommand{\mcU}{\mc U}
\newcommand{\mcV}{\mc V}
\newcommand{\mcW}{\mc W}
\newcommand{\mcX}{\mc X}
\newcommand{\mcY}{\mc Y}
\newcommand{\mcZ}{\mc Z}
\newcommand{\msA}{\ms A}
\newcommand{\msB}{\ms B}
\newcommand{\msC}{\ms C}
\newcommand{\msD}{\ms D}
\newcommand{\msE}{\ms E}
\newcommand{\msF}{\ms F}
\newcommand{\msG}{\ms G}
\newcommand{\msH}{\ms H}
\newcommand{\msI}{\ms I}
\newcommand{\msJ}{\ms J}
\newcommand{\msK}{\ms K}
\newcommand{\msL}{\ms L}
\newcommand{\msM}{\ms M}
\newcommand{\msN}{\ms N}
\newcommand{\msO}{\ms O}
\newcommand{\msP}{\ms P}
\newcommand{\msQ}{\ms Q}
\newcommand{\msR}{\ms R}
\newcommand{\msS}{\ms S}
\newcommand{\msT}{\ms T}
\newcommand{\msU}{\ms U}
\newcommand{\msV}{\ms V}
\newcommand{\msW}{\ms W}
\newcommand{\msX}{\ms X}
\newcommand{\msY}{\ms Y}
\newcommand{\msZ}{\ms Z}
\newcommand{\mfA}{\mf A}
\newcommand{\mfB}{\mf B}
\newcommand{\mfC}{\mf C}
\newcommand{\mfD}{\mf D}
\newcommand{\mfE}{\mf E}
\newcommand{\mfF}{\mf F}
\newcommand{\mfG}{\mf G}
\newcommand{\mfH}{\mf H}
\newcommand{\mfI}{\mf I}
\newcommand{\mfJ}{\mf J}
\newcommand{\mfK}{\mf K}
\newcommand{\mfL}{\mf L}
\newcommand{\mfM}{\mf M}
\newcommand{\mfN}{\mf N}
\newcommand{\mfO}{\mf O}
\newcommand{\mfP}{\mf P}
\newcommand{\mfQ}{\mf Q}
\newcommand{\mfR}{\mf R}
\newcommand{\mfS}{\mf S}
\newcommand{\mfT}{\mf T}
\newcommand{\mfU}{\mf U}
\newcommand{\mfV}{\mf V}
\newcommand{\mfW}{\mf W}
\newcommand{\mfX}{\mf X}
\newcommand{\mfY}{\mf Y}
\newcommand{\mfZ}{\mf Z}
\newcommand{\mfa}{\mf a}
\newcommand{\mfb}{\mf b}
\newcommand{\mfc}{\mf c}
\newcommand{\mfd}{\mf d}
\newcommand{\mfe}{\mf e}
\newcommand{\mff}{\mf f}
\newcommand{\mfg}{\mf g}
\newcommand{\mfh}{\mf h}
\newcommand{\mfi}{\mf i}
\newcommand{\mfj}{\mf j}
\newcommand{\mfk}{\mf k}
\newcommand{\mfl}{\mf l}
\newcommand{\mfm}{\mf m}
\newcommand{\mfn}{\mf n}
\newcommand{\mfo}{\mf o}
\newcommand{\mfp}{\mf p}
\newcommand{\mfq}{\mf q}
\newcommand{\mfr}{\mf r}
\newcommand{\mfs}{\mf s}
\newcommand{\mft}{\mf t}
\newcommand{\mfu}{\mf u}
\newcommand{\mfv}{\mf v}
\newcommand{\mfw}{\mf w}
\newcommand{\mfx}{\mf x}
\newcommand{\mfy}{\mf y}
\newcommand{\mfz}{\mf z}
\newcommand{\mbfA}{\mathbf A}
\newcommand{\mbfB}{\mathbf B}
\newcommand{\mbfC}{\mathbf C}
\newcommand{\mbfD}{\mathbf D}
\newcommand{\mbfE}{\mathbf E}
\newcommand{\mbfF}{\mathbf F}
\newcommand{\mbfG}{\mathbf G}
\newcommand{\mbfH}{\mathbf H}
\newcommand{\mbfI}{\mathbf I}
\newcommand{\mbfJ}{\mathbf J}
\newcommand{\mbfK}{\mathbf K}
\newcommand{\mbfL}{\mathbf L}
\newcommand{\mbfM}{\mathbf M}
\newcommand{\mbfN}{\mathbf N}
\newcommand{\mbfO}{\mathbf O}
\newcommand{\mbfP}{\mathbf P}
\newcommand{\mbfQ}{\mathbf Q}
\newcommand{\mbfR}{\mathbf R}
\newcommand{\mbfS}{\mathbf S}
\newcommand{\mbfT}{\mathbf T}
\newcommand{\mbfU}{\mathbf U}
\newcommand{\mbfV}{\mathbf V}
\newcommand{\mbfW}{\mathbf W}
\newcommand{\mbfX}{\mathbf X}
\newcommand{\mbfY}{\mathbf Y}
\newcommand{\mbfZ}{\mathbf Z}

% common math symbols

\newcommand{\bs}{\backslash}
\newcommand{\bbRn}{\mathbb R^n}
\newcommand{\bbCn}{\mathbb R^n}
\newcommand{\ecyc}[1]{\langle #1\rangle}
\newcommand{\interior}[1]{\overset{\circ}{#1}}
\newcommand{\ol}[1]{\overline{#1}}
\newcommand{\unitgrp}[1]{#1^{\mspace{-4mu}\times}}
\newcommand{\grad}{\triangledown}
\newcommand{\normal}{\trianglelefteq}
\newcommand{\nnormal}{\ntrianglelefteq}
\newcommand{\norm}[1]{\left\lVert#1\right\rVert}
\newcommand{\gen}[1]{\langle #1 \rangle}
\newcommand{\nilradical}[1]{\sqrt{\gen{0}}_{#1}}
\newcommand{\inv}[1]{#1^{\text{-}1}}
\newcommand{\id}{\mathrm{id}}
\newcommand{\unitcell}[1]{\overset{\!\!\!\! \circ}{\bbD^{#1}}}
\newcommand{\E}{\exists}
\newcommand{\A}{\forall}
\newcommand{\Fct}{\mathrm{Funct}}

% Operators

\DeclareMathOperator{\GL}{\rm{GL}}
\DeclareMathOperator{\SL}{\rm{SL}}
\DeclareMathOperator{\Mod}{\rm{Mod}}
\DeclareMathOperator{\Mor}{\rm{Mor}}
\DeclareMathOperator{\Obj}{\rm{Obj}}
\DeclareMathOperator{\Id}{\rm{Id}}
\DeclareMathOperator{\End}{\rm{End}}
\DeclareMathOperator{\Hom}{\rm{Hom}}
\DeclareMathOperator{\Res}{\rm{Res}}
\DeclareMathOperator{\Spec}{\rm{Spec}}
\DeclareMathOperator{\Proj}{\rm{Proj}}
\DeclareMathOperator{\Supp}{\rm{Supp}}
\DeclareMathOperator{\Ker}{\rm{Ker}}
\DeclareMathOperator{\Nil}{\rm{Nil}}
\DeclareMathOperator{\sh}{\rm{sh}}
\DeclareMathOperator{\Coker}{\rm{Coker}}
\DeclareMathOperator{\Rank}{\rm{Rank}}
\DeclareMathOperator{\Frac}{\rm{Frac}}
\DeclareMathOperator{\Rad}{\rm{Rad}}
\DeclareMathOperator{\Ann}{\rm{Ann}}
\DeclareMathOperator{\Disc}{\rm{Disc}}
\DeclareMathOperator{\Lcm}{\rm{lcm}}
\DeclareMathOperator{\Gcd}{\rm{gcd}}
\DeclareMathOperator{\Conv}{Conv}
\DeclareMathOperator{\Cone}{Cone}
\DeclareMathOperator{\Int}{Int}
\DeclareMathOperator{\Ord}{ord}
\DeclareMathOperator{\Ass}{Ass}
\DeclareMathOperator{\Aut}{Aut}
\DeclareMathOperator{\Sym}{Sym}
\DeclareMathOperator{\Char}{Char}
\DeclareMathOperator{\Span}{Span}
\DeclareMathOperator{\Tor}{Tor}
\DeclareMathOperator{\Ext}{Ext}
\DeclareMathOperator{\Card}{Card}
\DeclareMathOperator{\OPT}{OPT}
\DeclareMathOperator{\Dom}{Dom}
\DeclareMathOperator{\Var}{Var}
\DeclareMathOperator{\Th}{Th}
\DeclareMathOperator{\Frob}{Frob}
\DeclareMathOperator{\Red}{Red}
\DeclareMathOperator{\Aff}{Aff}
\DeclareMathOperator{\Epi}{Epi}
\DeclareMathOperator{\sech}{sech}
\DeclareMathOperator{\csch}{csch}
\DeclareMathOperator*{\Argmin}{Argmin}
\DeclareMathOperator{\Zer}{Zer}
\DeclareMathOperator{\Ri}{Ri}
\DeclareMathOperator{\Prox}{Prox}
\DeclareMathOperator{\sgn}{sgn}
\DeclareMathOperator{\Fix}{Fix}
\DeclareMathOperator{\Gal}{Gal}
\renewcommand{\Re}{\operatorname{Re}}
\renewcommand{\Im}{\operatorname{Im}}
\DeclareMathOperator{\Adj}{Adj}
\renewcommand{\div}{\mathrm{div}}
\DeclareMathOperator{\Div}{Div}
\DeclareMathOperator{\Cl}{Cl}
\DeclareMathOperator{\CDiv}{CDiv}
\DeclareMathOperator{\Bl}{Bl}
\DeclareMathOperator{\codim}{codim}
\DeclareMathOperator{\Sing}{Sing}
\DeclareMathOperator{\Nef}{Nef}
\DeclareMathOperator{\NE}{NE}
\DeclareMathOperator{\Mult}{Mult}
\DeclareMathOperator{\Pic}{Pic}
\DeclareMathOperator{\Tr}{Tr}
\DeclareMathOperator{\Grass}{Grass}
\DeclareMathOperator{\LinSub}{LinSub}
\DeclareMathOperator{\Eq}{Eq}
\DeclareMathOperator{\can}{can}
\DeclareMathOperator{\Rat}{Rat}
\DeclareMathOperator{\trdeg}{trdeg}
\DeclareMathOperator{\QSym}{QSym}
\DeclareMathOperator{\ASym}{ASym}
\DeclareMathOperator{\des}{des}
\DeclareMathOperator{\exc}{exc}
\DeclareMathOperator{\maj}{maj}
\DeclareMathOperator{\wt}{wt}
\DeclareMathOperator{\SST}{SST}
\DeclareMathOperator{\ASST}{\mfA SST}
\DeclareMathOperator{\SSYT}{SSYT}
\DeclareMathOperator{\ST}{ST}
\DeclareMathOperator{\SYT}{SYT}
\DeclareMathOperator{\SV}{SV}
\DeclareMathOperator{\ex}{ex}
\DeclareMathOperator{\sort}{sort}
\DeclareMathOperator{\sq}{sq}
\DeclareMathOperator{\invcode}{invcode}
\DeclareMathOperator{\Ess}{Ess}
\DeclareMathOperator{\Flag}{Flag}
\DeclareMathOperator{\Stab}{Stab}
\DeclareMathOperator{\Orb}{Orb}
\DeclareMathOperator{\pl}{pl}
\DeclareMathOperator{\Mat}{Mat}
\DeclareMathOperator{\pt}{pt}
\DeclareMathOperator{\depth}{depth}
\DeclareMathOperator{\cyc}{cyc}
\DeclareMathOperator{\ev}{ev}
\DeclareMathOperator{\length}{lg}
\DeclareMathOperator{\Quot}{Quot}
\DeclareMathOperator{\Hilb}{Hilb}
\DeclareMathOperator{\PGL}{PGL}
\DeclareMathOperator{\PSL}{PSL}
\newcommand{\Cat}[1]{(\textrm{#1})}

\begin{document}
    \title{CW complexes}
    \author{{\color{pink}{Cloudi}}{\color{Aquamarine}{fold}}}
    \maketitle
    \newpage

    \setcounter{section}{-1}

    \section{Basic Definitions and Lemmas}
    \begin{defn}
        A CW-complex is a space constructed by successively attaching cells:\\
        For $n \in \bbN, n \geq 1$, there are maps $\{ \varphi_i : S^{n-1} \to X^{n-1} \}_{i \in I_n}$
        (called characteristic maps). The way to construct $X^n$ (called $n$-skeleton of $X$) is :\\
        (starting from $X^0 = \prod_{I_0} \ast $)

        \[\begin{tikzcd}
            {\prod_{i \in I_n}S^{n-1}} & {X^{n-1}} \\
            {\prod_{i\in I_n}D^n} & {X^n}
            \arrow[hook, from=1-1, to=2-1]
            \arrow[from=2-1, to=2-2]
            \arrow["{\prod_{i \in I_n}\varphi_i}", from=1-1, to=1-2]
            \arrow[hook, from=1-2, to=2-2]
            \arrow["\lrcorner"{anchor=center, pos=0.125, rotate=180}, draw=none, from=2-2, to=1-1]
        \end{tikzcd}(pushout)\]\\
        and the resulting CW-complex $X$ is $\Colim \{X^0 \to \cdots \to X^n \to X^{n+1} \to \cdots \}$.
        The images of $\interior{D^n_i}$ in $X$ is called open cell $e^n_i$ of $X$.
    \end{defn}

    \begin{defn}
        $A$ is a subcomplex of CW-complex $X$ iff for any open cell $e^n_i$ of $X$, $A$ satisfy:
        $A \cap e^n_i \neq \emptyset \implies \bar{e^n_i} \subseteq A $.\\
        Pair of $X$ and subcomplex $A$ : $(X,A)$ is called a CW-pair.
    \end{defn}

    \begin{defn}
        The Infinite Symmetric Product of a pointed space $(X,x_0)$ is colimit of its $n$-th Symmetric Products
        ( $\SP^n X := (\prod_{\{0,1,\dots , n-1\}} X ) / S_n$ ) :\\
        \begin{align*}
            \Colim \{\dots \hookrightarrow \SP^{n} X & \hookrightarrow \SP^{n+1} X \hookrightarrow \cdots \} \\
            \{x_1, \dots, x_n\} & \mapsto \{x_0, x_1, \dots, x_n\}    
        \end{align*}
    \end{defn}

    \begin{defn}
        For $n \geq 1$, a map between pairs $f : (X,A) \to (Y,B)  $ is an $n$-equivalence
        if:
        \begin{itemize}
            \item $f_*^{-1}(\Im (\pi_0 B \to \pi_0 Y)) = \Im (\pi_0 A \to \pi_0 X)$
            \item For all choices of basepoint $a$ in $A$,
                $$
                f_* : \pi_q(X,A,a) \to \pi_q(Y,B,f(a))
                $$
                is isomorphism for $ 1 \leq q \leq n -1$ and epimorphism for $q = n$.
        \end{itemize}
        
    \end{defn}

    \begin{defn}
        A pair $(X,A)$ of topological spaces is \tbf{$n$-connected}
        if $\pi_0(A) \to \pi_0(X)$ is surjection and $\pi_q(X,A) = 0$ for
        $1 \leq q \leq n$.
    \end{defn}

    \begin{defn}
        For topological spaces $A \hookrightarrow X$, 
        $A$ is a \textbf{strong deformation retract} of a neighborhood $V$ in $X$
        if:\\
        $\exists h : V \times I \to X$ such that\\
        $\forall x \in V,\ h(x,0) = x$\\
        $h(V,1) \subseteq A$\\
        $\forall (a,t) \in A \times I,\ h(a,t) = a$
    \end{defn}

    \begin{defn}
        For topological spaces $i : A \hookrightarrow X$, 
        $A$ is a \textbf{deformation retract} of $X$
        if:\\
        $\exists h : X \times I \to X$ such that\\
        $\forall x \in X,\ h(x,0) = x$\\
        $h(X,1) = A$\\
        $\forall (a,t) \in A \times I,\ h(a,t) = a$\\
        (That is, there are retraction $r : X \to A$ and homotopy $h : \id_X \simeq i \circ r \rel A$)\\
        And $r := h(-,1)$ is called a \tbf{deformation retraction}.
    \end{defn}

    \begin{defn}
        For topological spaces $A \hookrightarrow X$, 
        a neighborhood $V$ of $A$ is \textbf{deformable}\label{hyp:deformable} to $A$ if:\\
        $\exists h : X \times I \to X$ such that\\
        $\forall x \in X,\ h(x,0) = x$\\
        $h(A \times I) \subseteq A$, $h(V \times I) \subseteq V$.\\
        $h(V,1) \subseteq A$
    \end{defn}

    A criterion of weak homotopy equivalence:
    \begin{lem}
        \label{hyp:criterion-HELP-Dn+1,Sn}
        The following on a map $e : Y \to Z$ and any fixed $n \in \bbN$ are equivalent:\\
        \begin{enumerate}
            \item For any $y \in Y$ , $e_* : \pi_q(Y,y) \to \pi_q(Z,e(y))$
                is monomorphism for $q = n$ and is epimorphism for $q = n+1$.
            \item (HELP of $(D^{n+1},S^n)$) Given maps $f : D^{n+1} \to Z$, $g : S^n \to Y$
            and homotopy $h : f \circ i \simeq e \circ g $:
            \[\begin{tikzcd}
                {S^n} & {D^{n+1}} \\
                Y & Z
                \arrow["g"', from=1-1, to=2-1]
                \arrow["e"', from=2-1, to=2-2]
                \arrow["i", hook, from=1-1, to=1-2]
                \arrow["f", from=1-2, to=2-2]
                \arrow["h"', shorten <=8pt, shorten >=8pt, Rightarrow, from=1-2, to=2-1]
            \end{tikzcd}\]
            then we have extension $g^+ : D^{n+1} \to Y$ of $g$
            and $h^+ : f \simeq e \circ g^+$:
            \[\begin{tikzcd}
                S^n & D^{n+1} \\
                Y & Z
                \arrow["g"', from=1-1, to=2-1]
                \arrow[hook, from=1-1, to=1-2]
                \arrow[""{name=0, anchor=center, inner sep=0}, "f", from=1-2, to=2-2]
                \arrow["e"', from=2-1, to=2-2]
                \arrow["{g^+}"', from=1-2, to=2-1]
                \arrow["{h^+}", shorten <=10pt, shorten >=10pt, Rightarrow, from=0, to=2-1]
            \end{tikzcd}\]
            \item Conclusion above holds when the given $h$ is $id_{f \circ i}$.
        \end{enumerate}
    \end{lem}

    \begin{prf}
        Trivially \tit{2.} implies \tit{3.}\\

        Our first goal : \tit{3.} implies \tit{1.}\\

        Fix $n\in \bbN$. $\pi_n(e)$ is monomorphism:\\
        For $n = 0$, \tit{3.} says if we have path $e(y) \simeq e(y') $
        then we have path $y \simeq y'$.That is to say
        $e$ can not map two path-connected component to one.\\
        For $n > 0$, \tit{3.} says if $e \circ g$ is nullhomotopic,
        then $g : S^n \to Y$ could be extend to $g^+ : D^{n+1} \to Y$,
        which can be used to construct nullhomotopy of $g$.\\

        Fix $n \in \bbN$. $\pi_{n+1}(e)$ is epimorphism:\\
        For $[f] \in \pi_{n+1}(Z,e(y)) \cong [D^{n+1},S^n; Z,e(y)]$,
        let $g := s \mapsto y$, the extension $g^+$ satisfy
        $e_* ([g^+]) = [f]$, that proves $e_*$ is epimorphism.\\

        Second goal : \tit{1.} implies \tit{2.}\\

        Fix $g, f, h$ in the condition of \tit{2.} first.
        And observe that
        $\pi_n(Y,y) = [S^n,\ast; Y,y]$, $\pi_{n+1}(Y,y) = [D^{n+1},S^n; Y,y]$.\\
        There is a map $f' : (D^{n+1},S^n) \to Z$ homotopic to $f$
        defined by $f' = f \circ b(-,1)$
        where
        \begin{align*}
            b : CS^n \times I & \to CS^n\\
                (\ol{(x,t)} , s)  & \mapsto
                \begin{cases}
                    \ol{(x,1-2t)} \quad t \leq \frac{s}{2}  \\
                    \ol{(x,\frac{t-s/2}{1-s/2})} \quad t \geq \frac{s}{2}
                \end{cases}
        \end{align*}
        (recall that $D^{n+1} \simeq CS^n $)
        Therefore we can replace $f$ with $f'$.
        Using the epimorphism leads to $h' : e \circ {g^+}' \simeq f'$,
        using the monomorphism leads to $r : {g^+}' \circ i \simeq g $.
        Construct $g^+ := a(-,1) $ using
        \begin{align*}
            a : CS^n \times I & \to Z\\
                (\ol{(x,t)} , s)  & \mapsto
                \begin{cases}
                    r(x, s-2t) \quad t \leq \frac{s}{2}  \\
                    {g^+}'(x,\frac{t-s/2}{1-s/2}) \quad t \geq \frac{s}{2}
                \end{cases}
        \end{align*}

        And that is the end of the proof:

        \[\begin{tikzcd}
            {S^n} && {D^{n+1}} \\
            \\
            Y && Z
            \arrow[from=1-1, to=3-1]
            \arrow[from=1-1, to=1-3]
            \arrow[from=3-1, to=3-3]
            \arrow[""{name=0, anchor=center, inner sep=0}, "f", curve={height=-12pt}, from=1-3, to=3-3]
            \arrow[""{name=1, anchor=center, inner sep=0}, curve={height=-12pt}, from=1-3, to=3-1]
            \arrow[""{name=2, anchor=center, inner sep=0}, curve={height=12pt}, from=1-3, to=3-1]
            \arrow[""{name=3, anchor=center, inner sep=0}, "{f'}"', curve={height=12pt}, from=1-3, to=3-3]
            \arrow["b", shorten <=4pt, shorten >=4pt, Rightarrow, from=1, to=2]
            \arrow[shorten <=5pt, shorten >=5pt, Rightarrow, from=0, to=3]
            \arrow["{h'}", shift left=5, shorten <=4pt, shorten >=0pt, Rightarrow, from=3, to=1]
        \end{tikzcd}\]
        \qed
        

    \end{prf}



    \section{Properties and Examples}

    \begin{thm}
        Homotopy Extension and Lifting property\label{hyp:HELP}:\\
        $A$ : a topological space\\
        $X$ : result of successively attaching cells on $A$ of dimensions $0,1,\dots,k$ ($k \leq n$)\\
        $e : Y \to Z$ : $n$-equivalence\\
        $g : A \to Y$, $f : X \to Z$\\
        $h : f|_A  \simeq e \circ g $\\

        \[\begin{tikzcd}
            A & X \\
            Y & Z
            \arrow["g"', from=1-1, to=2-1]
            \arrow[hook, from=1-1, to=1-2]
            \arrow["f", from=1-2, to=2-2]
            \arrow["e"', from=2-1, to=2-2]
            \arrow["h"', shorten <=7pt, shorten >=7pt, Rightarrow, from=1-2, to=2-1]
        \end{tikzcd}\]\\
        Then there exists
        $g^+ : X \to Y$ extends $g$ ($g^+|_A = g$)\\
        and $h^+ : X \times I \to Z$ extends $h$, $h^+ : f \simeq e \circ g^+$
        \[\begin{tikzcd}
            A & X \\
            Y & Z
            \arrow["g"', from=1-1, to=2-1]
            \arrow[hook, from=1-1, to=1-2]
            \arrow[""{name=0, anchor=center, inner sep=0}, "f", from=1-2, to=2-2]
            \arrow["e"', from=2-1, to=2-2]
            \arrow["{g^+}"', from=1-2, to=2-1]
            \arrow["{h^+}", shorten <=9pt, shorten >=9pt, Rightarrow, from=0, to=2-1]
        \end{tikzcd}\]
    \end{thm}

    \begin{prf}
        It suffices to prove the case $A = S^{k-1}, X = D^k$
        , $e$ is inclusion. (replace $Z$ by $M_e$)
        Apply HEP of $(D^k,S^{k-1})$:\\
        \[\begin{tikzcd}
            {S^{k-1}} & {D^k} \\
            {S^{k-1}\times I} & {D^k \times I} \\
            && Z
            \arrow[from=1-1, to=1-2]
            \arrow[from=1-1, to=2-1]
            \arrow[from=2-1, to=2-2]
            \arrow[from=1-2, to=2-2]
            \arrow["h"', curve={height=12pt}, from=2-1, to=3-3]
            \arrow["f", curve={height=-12pt}, from=1-2, to=3-3]
            \arrow["{\hat{h}}", dashed, from=2-2, to=3-3]
        \end{tikzcd}\]\\
        $f' := \hat{h}(-,1)$, replace $f$ with $f'$ the diagram would be strictly commute.
        Therefore, $f'$ is map of pairs $(D^k,S^{k-1}) \to (Z,Y)$,
        $k\leq n $ implies $f'$ is nullhomotopic, suppose $h^+ : D^k \times I \to Z$ is the nullhomotopy,
        then $g^+ := h^+(-,1)$ satisfy $ g^+(D^k) \subseteq Y $.\\
        \qed
    \end{prf}

    \begin{note}
        In HELP, at condition $Y=Z$ and $e = \id$,
        HELP says $(X,A)$ have HEP
    \end{note}

    \begin{cor}
        If \\
        $A$ : a topological space\\
        $X$ : result of successively attaching cells on $A$ of any dimensions\\
        Then, $(X,A)$ have HEP.
    \end{cor}

    \begin{thm}
        If $X$ is an CW-complex, $e : Y \to Z$ is an $n$-equivalence,
        Then $e_{\ast} : [X,Y] \to [X,Z]$ is a bijection if $\dim{X} < n$,
        and a surjection if $\dim{X} = n$.(Also valid for pointed case)
    \end{thm}

    \begin{prf}
        Surjectivity:\\
        Apply HELP of $(X,\emptyset)$ ($(X,x_0)$ for pointed case) to obtain $e_{\ast} [g^+] \simeq [f] $:
        \[\begin{tikzcd}
            \emptyset & X \\
            Y & Z
            \arrow[from=1-1, to=1-2]
            \arrow["f", from=1-2, to=2-2]
            \arrow[from=1-1, to=2-1]
            \arrow["e"', from=2-1, to=2-2]
            \arrow["{g^+}"', from=1-2, to=2-1]
        \end{tikzcd}\]\\
        Injectivity ($\dim{X} < n$):\\
        Suppose $[g_0],[g_1] \in [X,Y]$, $e_{\ast}[g_0] = e_{\ast}[g_1]$.\\
        Let $f : e \circ g_0 \simeq e \circ g_1$
        Apply HELP to $(X \times I, X  \times \partial I)$:
        \[\begin{tikzcd}
            {X\times \partial I} & {X \times I} \\
            Y & Z
            \arrow[from=1-1, to=1-2]
            \arrow["f", from=1-2, to=2-2]
            \arrow["g"', from=1-1, to=2-1]
            \arrow["e"', from=2-1, to=2-2]
            \arrow["{g^+}"', from=1-2, to=2-1]
        \end{tikzcd}\]
        \qed
    \end{prf}

    \begin{cor}
        If $X$ is a CW-complex, $e : Y \to Z$ is weak homotopy equivalence,
        then $e_{\ast} : [X,Y] \to [X,Z]$ is bijection.
    \end{cor}

    

    \subsection{CW-approximation}

    \begin{defn}
        A \textbf{CW-approximation} of $(X,A) \in \mbf{Top(2)}$ is a CW-pair $(\widetilde{X}, \widetilde{A})$ and a
        weak homotopy equivalence of pairs $\varphi : (\widetilde{X}, \widetilde{A}) \to (X,A)$.
    \end{defn}

    \begin{lem}
        \label{hyp:funCWapprox}
        $\varphi, \psi$ are CW-approximations of $X,Y$,
        $f : X \to Y$, then 
        \[\begin{tikzcd}
            {\widetilde{X}} & X \\
            {\widetilde{Y}} & Y
            \arrow["{\exists \widetilde{f}}"', dashed, from=1-1, to=2-1]
            \arrow["\varphi", from=1-1, to=1-2]
            \arrow["f", from=1-2, to=2-2]
            \arrow["\psi"', from=2-1, to=2-2]
        \end{tikzcd}\] commutes up to homotopy,
        and $\widetilde{f}$ is unique up to homotopy.
    \end{lem}

    \begin{prf}
        Directly from $\psi_{\ast} : [\widetilde{X},\widetilde{Y}] \to [\widetilde{X},Y]$ is bijection.
        \qed
    \end{prf}

    \begin{thm}
        $\varphi, \psi$ are CW-approximations of $(X,A), (Y,B)$,
        $f : (X,A) \to (Y,B)$, then 
        \[\begin{tikzcd}
            {(\widetilde{X},\widetilde{A})} & (X,A) \\
            {(\widetilde{Y},\widetilde{B})} & (Y,B)
            \arrow["{\exists \widetilde{f}}"', dashed, from=1-1, to=2-1]
            \arrow["\varphi", from=1-1, to=1-2]
            \arrow["f", from=1-2, to=2-2]
            \arrow["\psi"', from=2-1, to=2-2]
        \end{tikzcd}\] commutes up to homotopy,
        and $\widetilde{f}$ is unique up to homotopy.
    \end{thm}

    \begin{prf}
        Apply Lemma \ref{hyp:funCWapprox} to obtain map $\widetilde{f}_A : \widetilde{A} \to \widetilde{B}$
        and homotopy $h : \psi |_{\widetilde{B}} \circ \widetilde{f}_A \simeq f \circ \varphi|_{\widetilde{A}}$
        Use HELP of $(\widetilde{X}, \widetilde{A})$ to extend it:
        \[\begin{tikzcd}
            {\widetilde{A}} & {\widetilde{X}} \\
            {\widetilde{B}} & X \\
            {\widetilde{Y}} & Y
            \arrow[hook, from=1-1, to=1-2]
            \arrow["{\widetilde{f}_A}"', from=1-1, to=2-1]
            \arrow[hook, from=2-1, to=3-1]
            \arrow[from=3-1, to=3-2]
            \arrow[from=1-2, to=2-2]
            \arrow[from=2-2, to=3-2]
            \arrow["{\widetilde{f}}"', from=1-2, to=3-1]
        \end{tikzcd}\]
        $\psi_{\ast}$ is bijection implies the uniqueness up to homotopy of $\widetilde{f}$.
        \qed
    \end{prf}

    \begin{thm}
        (Whitehead's Theorem)\\
        Every $n$-equivalence between CW-complexes whose dimension is lower than $n$, is homotopy equivalence.
        Every weak homotopy equivalence between CW-complexes is homotopy equivalence.
    \end{thm}

    \begin{prf}
        $e : Y \to Z$ induce bijections $ [Y,Y] \to [Y,Z]$ and $[Z,Y] \to [Z,Z]$,
        $[f] = e_{\ast}^{-1} [\id_Z]$ implies $[e \circ f] = [\id_Z]$ and $[e \circ f \circ e] = [e]$ ($[f \circ e] = e_{\ast}^{-1}[e] = [\id_Y]$).\\
        \qed
    \end{prf}

    \begin{cor}
        CW-approximation is unique up to homotopy.
    \end{cor}

    \begin{exmp}
        Polish circle (Warsaw circle) : closed topologist's sine curve.
        It is $n$-connected forall $n$ but not contractible.
    \end{exmp}

    \begin{defn}
        A cellular map between CW-pairs is $g:(X,A) \to (Y,B)$ such that
        $g(A\cup X^n) \subseteq B \cup Y^n$.
    \end{defn}

    \begin{thm}
        For any map between CW-pairs $f : (X,A) \to (Y,B)$ there exists
        a cellular map $g$ such that $g \simeq f \rel A$
    \end{thm}

    \begin{prf}
        Construct $g$ inductively:\\
        Start from $A \cup X^0$:\\
        take paths $\gamma_i : f(x_i) \simeq y_i $, where $y_i$ is any point in $Y^0$ and $x_i \in X^0 - A$.\\
        Construct $h_0 : (X^0 \cup A) \times I \to Y $ : $h_0 |_{A} (a,t):= f(a)$,
        $h_0|_{X^0-A} (x_i,t) := \gamma_i(t)$.
        This is a homotopy from $f$ to $g_0 := h_0(-,1) : A\cup X^0 \to B \cup Y^0$\\
        Inductive step:\\
        Assume $g_n : A\cup X^n \to B \cup Y^n$ and homotopy $h_n : f|_{A \cup X^n} \simeq g_n$ is given, try to construct $g_{n+1}$:\\
        For each characteristic map $\varphi_i : S^n \to X^n$, take the resulting cell map $\varphi^+_i : D^{n+1} \to X^{n+1}$ 
        and use HELP of $(D^{n+1}, S^n)$:
        \[\begin{tikzcd}
            {S^n} & {D^{n+1}} \\
            {X^n} & {X^{n+1}} \\
            {B\cup Y^{n+1}} & Y
            \arrow[hook, from=1-1, to=1-2]
            \arrow["{\varphi_i}"', from=1-1, to=2-1]
            \arrow["{g_n}"', from=2-1, to=3-1]
            \arrow[hook, from=3-1, to=3-2]
            \arrow["{\varphi^+_i}", from=1-2, to=2-2]
            \arrow["f", from=2-2, to=3-2]
            \arrow["{g_{n+1,i}}"', from=1-2, to=3-1]
            \arrow["{h_{n+1,i}}", shorten <=4pt, shorten >=4pt, Rightarrow, from=2-2, to=3-1]
        \end{tikzcd}\]
        Glue all $g_{n+1,i}$ and $ h_{n+1,i}$ to produce $g_{n+1}$ and $h_{n+1} : f|_{A \cup X^{n+1}} \simeq g_{n+1}$.\\
        Final stage:\\
        Maps $g_n$ determine a cellular map $g : X \to Y$ since $X$ has the final topology determined by skeletons.
        \qed
    \end{prf}

    \subsection{Operation of CW-complexes}

    Product of CW-complexes:

    \begin{exmp}
        Product topology of two CW-complexes does not coincide with the final topology (union topology):\\
        $X$ (star of countably many edges) :
        $X = X^1 = \bigvee_{n\in \omega} I_n $\\
        $Y$ (star of $\omega^{\omega}$ many edges) :
        $Y = Y^1 = \bigvee_{f \in \omega^{\omega}} I_f $
        ( $(I_n,0) \cong (I_f,0) \cong (I,0) $ )\\
        Consider subset $H$ of $X \times Y$:
        $H := \{ (\frac{1}{f(n)+1} , \frac{1}{f(n)+1}) \in I_n \times I_f \mid n \in \omega, f \in \omega^{\omega} \}$.\\
        $H$ is closed under the final topology
        since every cell of $X \times Y$ contains at most one point of $H$.
        But closure of $H$ contains $(0,0)$ at product topology:\\
        Let $U \times V$ be an open neighborhood (at product topology) of $(0,0)$,
        let $g : \omega \to \omega - 0$ be an increasing function such that
        forall $n \in \omega, [0,\frac{1}{g(n)}) \subseteq U \cap I_n$,
        let $k \in omega$ be sufficiently large that $\frac{1}{g(k)+1} \subseteq V \cap I_g$,
        then $(\frac{1}{g(k)+1} , \frac{1}{g(k)+1}) \in U \times V \cap H$.

    \end{exmp}

    \begin{note}
        Another way to realize $X \times Y$ as CW-complex
        is to change its topology to the compactly generated topology $k(X \times Y)$.
    \end{note}

    \begin{prop}
        $X$ and $Y$ are CW-complexes, $X \times Y$ is CW-complex if\\
        $X$ or $Y$ is locally compact\\
        or\\
        both $X$ and $Y$ have countably many cells.
    \end{prop}

    Quotient of CW-pair:

    \begin{prop}
        For CW-complex $X$ and subcomplex $A$, the Quotient space $X/A$ 
        have a CW-complex structure induced by $X$ and $A$.
    \end{prop}

    \begin{prf}
        Suppose the characteristic maps of $X$ are indexed by $\{I_n\}_{n \in \bbN}$
        and of $A$ are indexed by $\{I'_n\}_{n \in \bbN}$ ($I'_n \subseteq I_n$).
        Then the characteristic maps of $X/A$ are indexed by $\{ K_n\}_{n \in \bbN}$, which defined below:\\
        $K_0 := (I_0 - I'_0) \cup \{ i_0 \} $ where $i_0$ is an arbitrary element
        in $I'_0$\\
        $K_n := I_n - I'_n$ for $n > 0$.\\
        Verify the maps determine the CW-complex structure:\\
        \[\begin{tikzcd}
            {S^{n-1}} & {X^{n-1}} & {X^{n-1}/A^{n-1}} \\
            {D^n} & {X^n} & {X^n/A^{n-1}} & {A^n} \\
            && {X^n/A^n} & \ast \\
            && Z
            \arrow[from=1-1, to=2-1]
            \arrow[from=1-3, to=2-3]
            \arrow[from=2-3, to=3-3]
            \arrow[from=2-4, to=2-3]
            \arrow[from=3-4, to=3-3]
            \arrow["\lrcorner"{anchor=center, pos=0.125, rotate=90}, draw=none, from=3-3, to=2-4]
            \arrow[from=2-4, to=3-4]
            \arrow[curve={height=12pt}, from=2-1, to=4-3]
            \arrow[curve={height=-44pt}, from=1-3, to=4-3]
            \arrow[curve={height=28pt}, dashed, from=2-3, to=4-3]
            \arrow[curve={height=-6pt}, from=3-4, to=4-3]
            \arrow[dashed, from=3-3, to=4-3]
            \arrow[from=2-2, to=2-3]
            \arrow[from=1-2, to=1-3]
            \arrow[from=1-2, to=2-2]
            \arrow[from=1-1, to=1-2]
            \arrow[from=2-1, to=2-2]
            \arrow["\lrcorner"{anchor=center, pos=0.125, rotate=180}, draw=none, from=2-3, to=1-2]
            \arrow["\lrcorner"{anchor=center, pos=0.125, rotate=180}, draw=none, from=2-2, to=1-1]
        \end{tikzcd}\]

        \qed
        
    \end{prf}

    Smash product of CW-complexes:

    \begin{prop}
        If $(X,x_0)$ , $(Y,y_0)$ are pointed CW-complexes with both countably many cell,
        and $X^{r-1} = \{x_0\}$, $Y^{s-1} = \{y_0\}$,
        then $X \wedge Y := X \times Y / X \vee  Y$ is
        an $(r+s-1)$-connected CW-complex.
    \end{prop}

    \begin{prf}
        $X \times Y$ is CW-complex with cells of the form
        $e^n_{i,X} \times \{y_0\}$, $\{x_0\} \times e^m_{j,Y}$
        or $e^n_{i,X} \times e^m_{j,Y}$
        for $n \geq r$, $m \geq s$.
        Cells of the first two forms are contianed in $X \vee Y$,
        therefore $(X \wedge Y)^{r+s-1} = \ast$.
        \qed
    \end{prf}

    \begin{cor}
        If $X$ is a pointed CW-complex,
        then $\Sigma^n X$ is an $(n-1)$-connected CW-complex.
    \end{cor}

    \subsection{Properties of Infinite Symmetric Product}
    Functoriality:\\

    Pointed map $f : X \to Y$
    induces
    \begin{align*}
        f_n : \SP^n X & \to \SP^{n}Y\\
            \{x_1, \dots, x_n\} & \mapsto \{f(x_1), \dots, f(x_n)\}
    \end{align*}
    
    \[\begin{tikzcd}
        {} & {\SP^n X} & {\SP^{n+1}X} & {} \\
        {} & {\SP^nY} & {\SP^{n+1} Y} & {}
        \arrow[from=1-2, to=1-3]
        \arrow[from=1-1, to=1-2]
        \arrow[from=2-1, to=2-2]
        \arrow["{f_n}", from=1-2, to=2-2]
        \arrow[from=2-2, to=2-3]
        \arrow["{f_{n+1}}", from=1-3, to=2-3]
        \arrow[from=2-3, to=2-4]
        \arrow[from=1-3, to=1-4]
    \end{tikzcd}\]

    Which induces map $\SP f : \SP X \to \SP Y$.
    And Functorial properties are directly from the constructions above.\\

    \label{hyp:SP-commute-with-directed-colimit} Commute with directed colimit:\\
    Suppose $P$ is a directed poset (that is $\A x,y \in P, \ \E z \in P, \ x \leq z , y \leq z$)
    and $X_i$ are pointed spaces indexed by $P$ satisfying $i \leq j \implies X_i \subseteq X_j$.\\
    Then $\SP^n (\Colim_{i} X_i) \approx \Colim_{i} (\SP^n X_i)$\\
    (Proof is obtained by showing that $ \SP^n f $ is continuous iff $f$ is,
    which implies final topology on $\Colim_{i} (\SP^n X_i)$ agree on $\SP^n (\Colim_i X_i)$)\\

    Suppose $i : A \hookrightarrow X$ is an pointed inclusion,
    then $\SP i : \SP A \hookrightarrow \SP X$ is also inclusion.
    Furthermore, if $A$ is open (or closed) in $X$, then $\SP A$ is open (or closed) in $\SP X$.\\

    Pointed homotopy $h : X \times I \to Y$ induces
    \begin{align*}
        h_n : \SP^n X \times I & \to \SP^{n}Y\\
            (\{x_1, \dots, x_n\} , t) & \mapsto \{h(x_1,t), \dots, h(x_n,t)\}
    \end{align*}
    which induces $\SP h : \SP X \times I \to \SP Y$.\\


    Then we observe:\\
    $f \simeq g$ implies $\SP f \simeq \SP g$,\\
    $e : X \to Y$ is homotopy equivalence implies $\SP e : \SP X \to \SP Y$ is,\\
    $X$ is contractible implies $\SP^n X$ and $\SP X$ is.


    \begin{thm}
        (Dold-Thom Theorem)\label{hyp:Dold-Thom}\\
        If $X$ is $T_2$ space and $A$ is closed path-connected subspace of $X$,
        and there is neighborhood $V$ \hyperref[hyp:deformable]{deformable} to $A$ in $X$.\\
        Then the quotient map $q : X \to X/A$ induces quasi-fibration
        $\SP q : \SP X \to \SP (X/A)$, which satisfy
        $\forall x \in \SP (X/A), \ \inv{(\SP q)} \{x\} \simeq  \SP A $.

    \end{thm}

    \begin{cor}
        If $X$ , $Y$ are $T_2$ spaces and $Y$ is connected, $f : X \to Y$.
        Then consider $X \to Y \to C_f \to \Sigma X$,
        the map $p : C_f \to \Sigma X$ induces quasi-fibration
        $\SP p : \SP C_f \to \SP (\Sigma X)$ with fiber $\SP Y$.
    \end{cor}

    \begin{cor}
        If $X$ is $T_2$ and path-connected, then for any $q\geq 0$,
        there is $\pi_{q+1}(\SP (\Sigma X)) \cong \pi_q (\SP X)$.
    \end{cor}

    \begin{prf}
        $CX$ is contractible implies $\SP CX$ is contractible,
        use the exat homotopy sequence of quasi-fibration to see:
        \[\begin{tikzcd}
            {} & {\pi_{q+1}(\SP CX)} & {\pi_{q+1}(\SP \Sigma X)} & {\pi_q (\SP X)} & {\pi_q(\SP CX)} & {}
            \arrow[from=1-2, to=1-3]
            \arrow["\partial"', "\cong", from=1-3, to=1-4]
            \arrow[from=1-1, to=1-2]
            \arrow[from=1-4, to=1-5]
            \arrow[from=1-5, to=1-6]
        \end{tikzcd}\]

        \qed
    \end{prf}

    \begin{note}
        The inverse of the isomorphism $\partial$ above is given by
        $$
        [S^q , \SP X] \ni [g] \mapsto [\Sigma g] \in [S^{q+1}, \Sigma \SP X]
        $$
        ($\Sigma \SP X \cong \SP \Sigma X$).
        Because $\partial$ is given by:
        \[\begin{tikzcd}
            {\pi_q(\SP \Sigma X)} & {\pi_q(\SP CX, \SP X)} & {\pi_{q-1}(\SP X)} \\
            {[f]} & {[\hat{f}]} & {[\hat{f}|_{S^{q-1}}]} \\
            {[p \circ Cg] = [\Sigma g]} & {[Cg]} & {[g]}
            \arrow[from=1-1, to=1-2]
            \arrow[from=1-2, to=1-3]
            \arrow[maps to, from=2-1, to=2-2]
            \arrow[maps to, from=2-2, to=2-3]
            \arrow["\ni"{marking}, draw=none, from=1-1, to=2-1]
            \arrow["\ni"{marking}, draw=none, from=1-2, to=2-2]
            \arrow["\ni"{marking}, draw=none, from=1-3, to=2-3]
            \arrow[maps to, from=3-3, to=3-2]
            \arrow[maps to, from=3-2, to=3-1]
        \end{tikzcd}\].
        \qed
    \end{note}

    \begin{cor}
        \label{hyp:quasi-fibration-SP(XUAxI)-to-SP(XUCA)}
        If $X$ is $T_2$ space and $A$ is path-connected subspace of $X$,
        then the canonical map
        $\SP (X \cup (A \times I)) \to \SP (X \cup C A)$ 
        is a quasi-fibration with fiber $\SP A$.
    \end{cor}

    \begin{thm}
        \label{hyp:quasi-fibration-SPX-to-SP(X/A)}
        If $X$ is $T_2$ space and $A$ is path-connected subspace of $X$,
        and $A \hookrightarrow X$ is a cofibration.\\
        Then the quotient map $q : X \to X/A$ induces quasi-fibration
        $\SP q : \SP X \to \SP (X/A)$, which satisfy
        $\forall x \in \SP (X/A), \ \inv{(\SP q)} \{x\} \simeq  \SP A $.
    \end{thm}

    \begin{prf}
        If $A \hookrightarrow X$ is cofibration,
        then $X \cup C A \simeq X/A $ and $X \cup (A \times I) \simeq X$.
        \qed
    \end{prf}

    \begin{prop}
        The inclusion $S^1 \to \SP S^1$ is homotopy equivalence,
        therefore $\pi_q(S^1) \cong \pi_q(\SP S^1)$.
    \end{prop}
    \begin{prf}
        $S^1 \simeq S^2 - \{0,\infty\}$\\
        $\SP^n S^2 = \{\{a_1,\dots, a_n\} \mid a_i \in \bbC \cup \{\infty\}\} = \{\prod_{\{a_1,\dots, a_n\}} (z-a_i) \mid a_i \in \bbC \cup \{\infty\} \}$
        where $(z - \infty) := 1$
        $\SP^n S^2 = \{f \in \bbC [z] - \{ 0 \} \mid \deg(f) \leq n \} = \bbC \bbP ^n$\\
        $$
        \SP^n (S^2 - \{0,\infty\}) = \{f \in \bbC [z] - \{ 0 \} \mid \deg(f) \leq n, f_n \neq 0, f_0 \neq 0 \} = 
        \bbC^n - \bbC^{n-1} \times 0 = \bbC^{n-1} \times (\bbC - 0)
        $$
        it have the same homotopy type of $S^1$
    \end{prf}

    \begin{cor}
        \label{hyp:SP-Sn-is-K(Z,n)}
        $\pi_q(\SP S^n) = \bbZ $ if $q = n$, otherwise $\pi_q(\SP S^n) = 0$.
        (use corollary of \ref{hyp:Dold-Thom} to see $\pi_{q+1}(\SP \Sigma X) \cong \pi_q (\SP X)$)
    \end{cor}


    \section{Homology Groups}

    \subsection{Reduced Homology Groups}

    \begin{defn}
        For a path-connected pointed CW-complex $X$,
        define its \textbf{$n$-th reduced homology group} for $n \geq 0$:\\
        $$
        \waveH_n(X) := \pi_n(\SP X)
        $$
    \end{defn}

    \begin{note}
        All reduced homology groups are abelian since
        $\waveH_n(X) \cong \waveH_{n+1}(\Sigma X)$.
        Thus, we can extend the definition above
        to those $X$ which does not necessarily be path-connected.\\
    \end{note}

    As $\SP$, $\waveH_n$ also satisfy functoriality.
    Furthermore, $\waveH_n$ maps homotopic maps $f \simeq g$ to identical maps $f_* = g_*$.
    ($\SP$ maps homotopic maps to homotopic maps)\\

    Exact Property:
    \begin{prop}
        For any pointed map between CW-complexes $f : X \to Y$,
        we have an exact sequence:
        $$
        \waveH_n(X) \xrightarrow{f_*} \waveH_n(Y) \xrightarrow{i_*} \waveH_n(C_f)
        $$
        where $C_f$ is the mapping cone of $f$, $i : Y \hookrightarrow C_f$.
    \end{prop}

    \begin{prf}
        $Z_f$ := $Y \cup_f (X \times I) / \{ x_0 \} \times I$ is the \textbf{reduced mapping cylinder}
        of $f$.\\
        $q : Z_f \to C_f $ is defined by
        \begin{align*}
            y & \mapsto y\\
            \overline{(x,t)}^{Z_f} & \mapsto \overline{(x,t)}^{C_f}
        \end{align*}
        By \hyperref[hyp:Dold-Thom]{Dold-Thom theorem},
        the induced map $\SP q$ is quasi-fibration $\SP Z_f \to \SP C_f $ with fiber $\SP X$.
        By definition of quasi-fibration, we have
        $$
        \pi_n(\SP X) \cong \waveH_n(X) \xrightarrow{f_*} \pi_n(\SP Z_f) \cong \waveH_n(Y) \xrightarrow{i_*} \pi_n(\SP C_f) = \waveH_n(C_f)
        $$
        \qed
    \end{prf}

    \begin{prop}
        \label{hyp:Sn-is-not-retract-of-Dn+1} There does not exist retraction $r : \bbD^n \to S^{n-1} $.
    \end{prop}

    \begin{prf}
        $id = r \circ i : \bbS^{n-1} \to \bbD^n \to \bbS^{n-1}$
        induces
        $$
        id_* = r_* \circ i_* : \bbZ \cong \waveH_{n-1}\bbS^{n-1} \to \waveH_{n-1}\bbD^n  \cong 0 \to \waveH_{n-1}\bbS^{n-1} \cong \bbZ
        $$
        which lead to contradiction.
        \qed
    \end{prf}

    \begin{thm}
        Fix-point theorem:\\
        If $f : \bbD^n \to \bbD^n$ is continuous,
        then exist $x_0 \in \bbD^n$ such that
        $x_0 = f(x_0)$.
    \end{thm}

    \begin{prf}
        (non-constructive)
        No such $x_0$ implies $\forall x \in \bbD^n , f(x) \neq x$
        therefore, we can construct continuous retraction $r : \bbD^n \to \bbS^{n-1}$
        by\\ $r(x)$:= the intersection of ``ray starting from $f(x)$ to $x$'' and $\bbS^{n-1}$.
        Contradict to \ref{hyp:Sn-is-not-retract-of-Dn+1}.
    \end{prf}

    \begin{defn}
        \label{hyp:homology-group-defined-using-SP}
        Let $(X,A)$ be an CW-pair, define the
        \textbf{$n$-th homology group} for $n \in \bbN$ of $(X,A)$ be:\\
        $$
        H_n(X,A)  :=  \waveH_n(X \cup CA)
        $$
        And for single space:\\
        $$
        H_n(X) := H_n(X, \emptyset) = \waveH(X + 1)
        $$
        where $X + 1 := X \sqcup *$.
    \end{defn}

    \begin{note}
        Map between CW-pair $f : (X,A) \to (Y,B)$,
        induces map $\bar{f} : X \cup CA \to Y \cup CB$
        defined by $(x,t) \mapsto (f(x),t)$,
        which induces $f_* : \waveH_n(X \cup CA) \to \waveH_n(Y \cup CB)$ for any $n \in \bbN$.
    \end{note}

    \subsection{Axioms for Homology}

    \begin{defn}
        \label{hyp:Ordinary-Homology-Theory}
        A (Ordinary) Homology Theory (on $\mbf{TOP}$ with coefficient $G \in \mbf{Ab}$) is
        functors $\{H_n(-,-;G) : \mbf{TOP(2)} \to \mbf{Ab}\}_{n \in \bbN}$,
        with natural transformations ${\partial_{n,(X,A)} : H_n(X,A;G) \to H_n(A,\emptyset;G)}$
        (called connecting homomorphism)\\
        satisfying following axioms:\\
        \begin{itemize}
            \item \textbf{Dimension}:\\
                $H_0 (\ast, \emptyset;G) = G$,
                for any $n > 0$, $H_n (\ast, \emptyset;G) = 0$.\\
            \item \textbf{Weak Equivalence}:\\
                Weak equivalence $f : (X,A) \to (Y,B)$
                induces isomorphism
                $$
                f_* : H_*(X,A;G) \to H_*(Y,B;G)
                $$
            \item \textbf{Long Exact Sequence}:\\
                For any $(X,A) \in \mbf{TOP(2)}$, maps $A \hookrightarrow X$
                and $(X,\emptyset) \to (X,A)$ induce a long exact sequence together
                with $\partial$:
                $$
                \cdots \to H_{q+1}(A;G) \to H_{q+1}(X;G) \to H_{q+1}(X,A;G) \to H_q(A;G) \to \cdots
                $$
                where $H_n(X;G) := H_n(X,\emptyset;G)$.\\
            \item \textbf{Additivity}:\\
                If $(X,A) = \coprod_\lambda (X_\lambda, A_\lambda) $ in $\mbf{TOP(2)}$,
                then inclusions $i_\lambda : (X_\lambda, A_\lambda) \to (X,A)$
                induces isomorphism
                $$
                (\bigoplus i_{\ast ,\lambda})  : \bigoplus_{\lambda} H_*(X_\lambda,A_\lambda;G) \cong H_*(X,A;G)
                $$
            \item \textbf{Excision}:\\
                If $(X;A,B)$ is an \textbf{excisive triad} (that is, $X = \interior{A} \cup \interior{B}$),
                then inclusion $(A,A \cap B) \hookrightarrow (X,B)$
                induces isomorphism
                $$
                H_*(A,A\cap B ; G) \cong H_*(X,B;G)
                $$\\
        \end{itemize}

    \end{defn}

    \begin{note}
        An equivalent form of Excision  Axiom:\\
        If $(X,A) \in \mbf{TOP(2)}$, $U$ is subspace of $A$ and $\ol{U} \subseteq \interior{A}$,
        then inclusion $ i : (X - U, A - U) \hookrightarrow (X,A) $
        induces isomorphism
        $$
        i_* : H_*(X-U, A-U;G) \to H_*(X,A;G)
        $$\\
    \end{note}

    There is a critical criterion about weak homotopy equivalence between excisive triads,
    we prove lemmas first:

    \begin{lem}
        \label{hyp:pushout-preserve-dr}
        For
        \[\begin{tikzcd}
            Z & Y \\
            X & {X \cup_Z Y}
            \arrow["i"', from=1-1, to=2-1]
            \arrow["f", from=1-1, to=1-2]
            \arrow["{i_*}", from=1-2, to=2-2]
            \arrow["{f_*}"', from=2-1, to=2-2]
            \arrow["\lrcorner"{anchor=center, pos=0.125, rotate=180}, draw=none, from=2-2, to=1-1]
        \end{tikzcd}\]
        if $D$ is deformation retract of X and $Z \subseteq D \subseteq X $,
        then $D \cup_Z Y $ is deformation retract of $X \cup_Z Y$.
    \end{lem}

    \begin{prf}
        Let $h : \id_X \simeq r \circ i$ where $r$ is the deformation retraction
        $X \to D$.
        Define $h_* : \id_{X \cup_Z Y} \simeq (i \cup_Z \id_Y) \circ (r \cup_Z \id_Y)$
        \begin{align*}
            h_* : (X \cup_Z Y) \times I & \to X \cup_Z Y\\
            (x,t) & \mapsto f_*(h(x,t))\\
            (y,t) & \mapsto i_*(y)
        \end{align*}

        Observe that $(X \cup_Z Y) \times I = (X \times I) \cup_{Z \times I} (Y \times I)$,
        check that $h^*$ is continuous:

        \[\begin{tikzcd}
            {Z \times I} & {Y \times I} \\
            {X \times I} & {(X\cup_Z Y) \times I} & Y \\
            & X & {X \cup_Z Y}
            \arrow[from=1-1, to=2-1]
            \arrow[from=1-1, to=1-2]
            \arrow[from=2-1, to=2-2]
            \arrow[from=1-2, to=2-2]
            \arrow["\lrcorner"{anchor=center, pos=0.125, rotate=180}, draw=none, from=2-2, to=1-1]
            \arrow["{\id_{\id_Y}}", from=1-2, to=2-3]
            \arrow["h"', from=2-1, to=3-2]
            \arrow["{i_*}"', from=2-3, to=3-3]
            \arrow["{f_*}"', from=3-2, to=3-3]
            \arrow["{h_*}"', from=2-2, to=3-3]
        \end{tikzcd}\]
        \qed
    \end{prf}

    \begin{lem}
        \label{hyp:double-mapping-cylinder-homotopy-pushout}
        For maps $i : C \to A$, $j : C \to B$
        define the double mapping cylinder
        $M(i,j) := A \cup_{C \times \{0\}} C \times I \cup_{C \times \{1\}} B$.
        If $i$ is cofibration,
        then the quotient map
        \begin{align*}
            q : M(i,j) & \to A \cup_C B\\
            a & \mapsto a\\
            b & \mapsto b\\
            (c,t) &  \mapsto c\\
        \end{align*}
        is a homotopy equivalence.
    \end{lem}

    \begin{prf}
        \[\begin{tikzcd}
            C & B \\
            A & {A \cup_C B}
            \arrow["i"', hook, from=1-1, to=2-1]
            \arrow[from=1-1, to=1-2]
            \arrow[from=1-2, to=2-2]
            \arrow["{i_A}"', from=2-1, to=2-2]
        \end{tikzcd}\]
        The canonical quotient $r : M_{i_A} \to A \cup_C B$ is a deformation retraction with homotopy:
        \begin{align*}
            h : (B \cup_{C \times 0} (A \times I)) \times I & \to B \cup_{C \times 0} (A \times I) = M_{i_A}\\
                (a,t,s) & \mapsto (a,(1-s)t)\\
                (b,s) & \mapsto b
        \end{align*}
        Observe that $C \times I \cup_C A \times \{1\}$ is a deformation retract of $A \times I$,
        since $i : C \to A$ is cofibration.\\
        Then we have $ M(i,j) =  B \cup_{C \times \{0\}}  (C \times I \cup_{C \times \{1\}} A \times \{1\})$
        is a deformation retract of $ B \cup_{C \times \{0\}} A\times I = M_{i_A}$.
        (use lemma \ref{hyp:pushout-preserve-dr})\\
        Finally, an easy check shows that $M(i,j) \to M_{i_A} \xrightarrow{r} A \cup_C B $
        is identical to $q$.\\
        \qed
    \end{prf}

    \begin{thm}
        \label{hyp:excisive-triads-weak-equivalence}
        For excisive triads $(X;X_1,X_2)$, $(X';X'_1,X'_2)$
        and map $e : X \to X'$,
        if
        \begin{align*}
            e|_{X_1} : X_1 & \to X'_1\\
            e|_{X_2} : X_2 & \to X'_2\\
            e|_{X_3} : X_3 & \to X'_3
        \end{align*}
        
        are weak equivalences, (where $X_3 := X_1 \cap X_2,\  X'_3 := X'_1 \cap X'_2$)
        then $e$ is.
    \end{thm}

    \begin{prf}
        Use an important \hyperref[hyp:criterion-HELP-Dn+1,Sn]{criterion} of 
        weak homotopy equivalence,
        it suffices to show forall $n \in \bbN$,
        any commutative diagram below:
        \[\begin{tikzcd}
            {S^n} & {D^{n+1}} \\
            X & {X'}
            \arrow["i", hook, from=1-1, to=1-2]
            \arrow["g"', from=1-1, to=2-1]
            \arrow["e"', from=2-1, to=2-2]
            \arrow["f", from=1-2, to=2-2]
        \end{tikzcd}\]
        can be filled like:
        \[\begin{tikzcd}
            {S^n} & {D^{n+1}} \\
            {X} & {X'}
            \arrow["i", hook, from=1-1, to=1-2]
            \arrow["g"', from=1-1, to=2-1]
            \arrow["e"', from=2-1, to=2-2]
            \arrow[""{name=0, anchor=center, inner sep=0}, "f", from=1-2, to=2-2]
            \arrow["{g^+}"', from=1-2, to=2-1]
            \arrow["h", shorten <=8pt, shorten >=14pt, Rightarrow, from=0, to=2-1]
        \end{tikzcd}\]
        whose upper triangle commutes.\\

        Let
        \begin{align*}
            A_1:= g^{-1}(X-\interior{X_1}) & \cup f^{-1}(X' - \interior{X'_1})\\
            A_2:= g^{-1}(X-\interior{X_2}) & \cup f^{-1}(X' - \interior{X'_2})
        \end{align*}
        which are disjoint closed subsets of $D^{n+1}$.
        Choose CW-complex structure on $D^{n+1}$ 
        such that for each $n$-cell $\sigma_i$,
        $\ol{\sigma_i} \cap (A_1 \cup A_2) = \ol{\sigma_i} \cap A_1 \text{ or } \ol{\sigma_i} \cap A_2$.
        Now define
        \begin{align*}
            K_1:= \bigcup \{ \ol{\sigma_i} \mid g(\ol{\sigma_i} \cap S^n) \subseteq \interior{X_1} \text{ and } f(\ol{\sigma_i}) \subseteq \interior{X'_1} \} & = \bigcup \{ \ol{\sigma_i} \mid \ol{\sigma_i} \cap A_1 = \emptyset \}\\
            K_2:= \bigcup \{ \ol{\sigma_i} \mid g(\ol{\sigma_i} \cap S^n) \subseteq \interior{X_2} \text{ and } f(\ol{\sigma_i}) \subseteq \interior{X'_2} \} & = \bigcup \{ \ol{\sigma_i} \mid \ol{\sigma_i} \cap A_2 = \emptyset \}
        \end{align*}
        which are subcomplexes of $D^{n+1}$ and satisfy $K_1 \cup K_2 = D^{n+1}$.
        By \hyperref[hyp:HELP]{HELP}, we have:
        \[\begin{tikzcd}
            {S^n \cap K_1 \cap K_2} & {K_1 \cap K_2} \\
            {X_1 \cap X_2} & {X'_1 \cap X'_2}
            \arrow["i", from=1-1, to=1-2]
            \arrow["{g|_{K_1 \cap K_2}}"', from=1-1, to=2-1]
            \arrow[""{name=0, anchor=center, inner sep=0}, "{f|_{K_1\cap K_2}}", from=1-2, to=2-2]
            \arrow["{e|_{X_1 \cap X_2}}"', from=2-1, to=2-2]
            \arrow["{g_0}"', from=1-2, to=2-1]
            \arrow["{h_0}", shorten <=13pt, shorten >=13pt, Rightarrow, from=0, to=2-1]
        \end{tikzcd}\]
        such that $h_0$ is $f|_{K_1 \cap K_2} \simeq e \circ g_0 \rel (S^n \cap K_1 \cap K_2)$.
        Apply HELP to:
        \[\begin{tikzcd}
            {(S^n \cup K_1) \cap K_2} & {K_2} && {(S^n \cup K_2) \cap K_1} & {K_1} \\
            {X_2} & {X'_2} && {X_1} & {X'_1}
            \arrow["i_2", from=1-1, to=1-2]
            \arrow["{f|_{K_2}}", from=1-2, to=2-2]
            \arrow[from=2-1, to=2-2]
            \arrow["{g_{K_2}}"', from=1-1, to=2-1]
            \arrow["{h_{K_2}}", shorten <=10pt, shorten >=10pt, Rightarrow, from=1-2, to=2-1]
            \arrow["i_1", from=1-4, to=1-5]
            \arrow["{g_{K_1}}"', from=1-4, to=2-4]
            \arrow["{f|_{K_1}}", from=1-5, to=2-5]
            \arrow[from=2-4, to=2-5]
            \arrow["{h_{K_1}}", shorten <=10pt, shorten >=10pt, Rightarrow, from=1-5, to=2-4]
        \end{tikzcd}\]
        where\\
        $g_{K_i}$ are defined by $g_{K_i}|_{S^n \cap K_i} :=  g|_{{S^n \cap K_i}} $
        and $g_{K_i}|_{K_1 \cap K_2} :=  g_0 $,\\
        $h_{K_2}$ are defined by ($h_{K_1}$ is similar):
        \begin{align*}
        h_{K_2} : ((S^n \cup K_1) \cap K_2 ) \times I & \to X'_2\\
            (x,t) & \mapsto
            \begin{cases}
                e(g(x)) \quad x \in S^n \cap K_2 \\
                h_0(x,t) \quad x \in K_1 \cap K_2
            \end{cases}
        \end{align*}
        We get:
        \[\begin{tikzcd}
            {(S^n \cup K_1)\cap K_2} & {K_2} && {(S^n \cup K_2) \cap K_1} & {K_1} \\
            {X_2} & {X'_2} && {X_1} & {X'_1}
            \arrow[from=1-1, to=1-2]
            \arrow["{g_{K_2}}"', from=1-1, to=2-1]
            \arrow[""{name=0, anchor=center, inner sep=0}, "{f|_{K_2}}", from=1-2, to=2-2]
            \arrow["{e|_{X_2}}"', from=2-1, to=2-2]
            \arrow["{g_2}"', from=1-2, to=2-1]
            \arrow[from=1-4, to=1-5]
            \arrow[""{name=1, anchor=center, inner sep=0}, from=1-5, to=2-5]
            \arrow["{g_{K_1}}"', from=1-4, to=2-4]
            \arrow["{e|_{X_1}}"', from=2-4, to=2-5]
            \arrow["{g_1}"', from=1-5, to=2-4]
            \arrow["{h_2}", shorten <=13pt, shorten >=19pt, Rightarrow, from=0, to=2-1]
            \arrow["{h_1}", shorten <=13pt, shorten >=19pt, Rightarrow, from=1, to=2-4]
        \end{tikzcd}\]
        
        Define $g^+$ and $h : f \simeq g \rel {S^n}$ by $g^+|_{K_i} := g_i$
        and $h|_{K_i \times I} := h_i$.\\
        $h|_{S^n \times I} = (e \circ g) \times \id_I$ ($h$ is $\rel S^n$)
        since $h_i(-,t)|_{S^n \cap K_i} = h_{K_i}(-,t)|_{S^n \cap K_i} = e \circ g|_{S^n \cap K_i}$.\\
        \qed
    \end{prf}

    \begin{note}
        The proof above can be easily modified to case each weak equivalence appear in the statement is an $n$-equivalence.\\
    \end{note}

    Following theorem allow us to use CW-triads to approximate excisive triads:

    \begin{thm}
        \label{hyp:CW-triad-approx-excisive-triad}
        For any excisive triad $(X;A,B)$,
        there is a CW-triad $(\widetilde{X}; \widetilde{A}, \widetilde{B})$
        (A CW-triad $(X;A,B)$ is $X$ and its subcomplex $A,B$ such that $A \cup B = X$)
        and a map $r : \widetilde{X} \to X$
        such that
        \begin{align*}
            r|_{\wtl{A}} : \wtl{A} & \to A\\
            r|_{\wtl{B}} :  \wtl{B} & \to B \\
            r|_{\wtl{C}} : \wtl{C} & \to C\\
            r  : \wtl{X} & \to X
        \end{align*}
        are all weak homotopy equivalences (where $\wtl{C} := \wtl{A} \cap \wtl{B}$, $C := A \cap B$).
        Furthermore, such $r$ is natural up to homotopy.
    \end{thm}

    \begin{prf}
        Choose a CW-approximation $r_C : \wtl{C} \to C$
        and extend it to $r_A : \wtl{A} \to A$, $r_B : \wtl{B} \to B$.
        $\wtl{X} := \wtl{A} \cup_{\wtl{C}} \wtl{B}$.
        $i : \wtl{C} \to \wtl{A} $ and $j : \wtl{C} \to \wtl{B}$
        are cofibrations,
        by lemma \ref{hyp:double-mapping-cylinder-homotopy-pushout}
        we have homotopy equivalence $q : M(i,j) \to \wtl{X}$,
        which induces homotopy equivalence of triads:
        \begin{align*}
            q : M(i,j) & \to \wtl{X}\\
            q| : \wtl{A} \cup (\wtl{C} \times [0,\frac{2}{3})) & \to \wtl{A}\\
            q| : \wtl{B} \cup (\wtl{C} \times (\frac{1}{3},1]) & \to \wtl{B}\\
        \end{align*}
        then we can deduce that
        $r \circ q$
        is a weak homotopy equivalence by theorem \ref{hyp:excisive-triads-weak-equivalence}.
        Consquently, $r$ is weak homotopy equivalence.
        $r$ is natural up to homotopy since each CW-approximation $r_C, r_A, r_B$ is.\\
        \qed
    \end{prf}
    
    Then we have:
    \begin{defn}
        A (Ordinary) Homology Theory on  CW-complexes with coefficient $G \in \mbf{Ab}$ is
        functors $\{H_n(-,-;G) : \textbf{CW-pairs} \to \mbf{Ab}\}_{n \in \bbN}$,
        with natural transformations ${\partial_{n,(X,A)} : H_n(X,A;G) \to H_n(A,\emptyset;G)}$
        (called connecting homomorphism)\\
        satisfying \hyperref[hyp:Ordinary-Homology-Theory]{axioms}
        with the excision axiom changed to:
        \begin{itemize}
            \item \tbf{Excision}:\\ 
                If (X;A,B) is an \tbf{CW-triad} (that is $X = A \cup B$ for subcomplexes $A$ and $B$)
                then the inclusion $(A,A \cap B) \hookrightarrow (X,B)$
                induces isomorphism
                $$
                H_*(A,A\cap B ; G) \cong H_*(X,B;G)
                $$\\
        \end{itemize}
    \end{defn}

    \begin{prop}
        The homology groups defined in definition \ref{hyp:homology-group-defined-using-SP}
        is a ordinary homology theory on CW-complexes with coefficient $\bbZ$.
    \end{prop}

    \begin{prf}
        \text{ }\\
        \begin{itemize}
            \item Dimension: by \hyperref[hyp:SP-Sn-is-K(Z,n)]{a corollary}, $H_q(\ast, \emptyset) = \pi_q(\SP S^0) =
            \begin{cases}
                \bbZ \quad q = 0\\
                0  \quad q \geq 1
            \end{cases}$.
            \item Weak Equivalence: $\SP$ preserves weak equivalence.
            \item Long Exact Sequence: use a \hyperref[hyp:quasi-fibration-SP(XUAxI)-to-SP(XUCA)]{corollary}
                of Dold-Thom theorem.
            \item Additivity:
                For index set $\Lambda$, $P := \{S \mid S \subseteq \Lambda \}$.\\
                Then define $Y_S := \bigvee_{\lambda \in S} X_\lambda \cup CA_\lambda = (\coprod_{\lambda \in S} X_\lambda) \cup C (\coprod_{\lambda \in S} A_\lambda)$,\\
                and use fact that \hyperref[hyp:SP-commute-with-directed-colimit]{$\SP$ commutes with directed colimit}, we have\\
                $\bigvee_{\lambda \in \Lambda} \SP (X_\lambda \cup CA_\lambda) = \Colim_{S \in P} \SP Y_S \approx  \SP (\Colim_{S \in P} Y_S) = \SP ((\coprod_{\lambda \in \Lambda} X_\lambda) \ \cup \ C (\coprod_{\lambda \in \Lambda} A_\lambda)) = \SP ( X \cup CA)$.\\
                Which induces $\bigoplus_{\lambda \in \Lambda} \waveH_n(X_\lambda \cup C A_\lambda) \cong \pi_n (\bigvee_{\lambda \in \Lambda} \SP (X_\lambda \cup CA_\lambda)) \cong \pi_n(\SP (X \cup CA)) = \waveH_n(X \cup CA)$.
            \item Excision: For CW-triad $(X;A,B)$,
                $A/(A \cap B) \approx X/B$.
                Apply theorem \ref{hyp:quasi-fibration-SPX-to-SP(X/A)} to $(Y \cup CZ, CZ)$ to show that
                $H_n(Y,Z) \cong \waveH_n(Y/Z)$.\\
        \end{itemize}
        \qed
    \end{prf}

    \section{Homotopy and Eilenberg-Mac Lane Spaces}

    \begin{thm}
        \label{hyp:Homotopy-Excision-Theorem}
        (Blakers–Massey) Homotopy Excision Theorem:\\
        For pointed CW-triad $(X;A,B)$
        such that $C := A \cap B \neq \emptyset$,
        if $(A,C)$ is $(m-1)$-connected and
        $(B,C)$ is $(n-1)$-connected
        where $m \geq 2$, $n \geq 1$.
        Then $i : (A,C) \to (X,B)$ is an $(m+n-2)$-equivalence for pairs.
    \end{thm}

    \begin{note}
        We can replace the "CW-triad" with "excisive triad" in condition
        by theorem \ref{hyp:CW-triad-approx-excisive-triad}.
    \end{note}

    \begin{prf}
        Define (pointed) the triad homotopy group for $q \geq 2$:\\
        $$
        \pi_q(X;A, B) := \pi_{q-1}(P_{i_{B,X}}, P_{i_{C,A}})
        $$
        where $i_{B,X} : B \hookrightarrow X$, $i_{C,A} : C \hookrightarrow A$
        and $P_{f}$ is the homotopy fiber\\
        $$
        \{ (y,\gamma) \in Y \times M(I,Z)_* \mid \gamma(1) = f(y) \}
        $$
        of pointed map $f : Y \to Z$.
        Use long exact sequence of pairs:
        \begin{align*}
            & \cdots \to \pi_{q}(P_{i_{B,X}}, P_{i_{C,A}}) \to \pi_{q-1}(P_{i_{C,A}})
            \to \pi_{q-1} (P_{i_{B,X}}) \to \pi_{q-1}(P_{i_{B,X}}, P_{i_{C,A}})
            \to \pi_{q-2}(P_{i_{C,A}}) \to \cdots \\
            & \cdots \to
            \pi_{1}(P_{i_{B,X}}, P_{i_{C,A}}) \to \pi_{0}(P_{i_{C,A}})
            \to \pi_{0} (P_{i_{B,X}}) 
        \end{align*}
        and observe that $\pi_{q}(P_{i_{X,B}}) \cong \pi_{q+1}(X,B)$
        since for any $f : S^q \to P_{i_{X,B}}$
        we have:
        \[\begin{tikzcd}
            {S^q} \\
            & {P_{i_{B,X}}} & {M(I,X)_*} \\
            & B & X
            \arrow[from=2-2, to=3-2]
            \arrow[from=3-2, to=3-3]
            \arrow[from=2-2, to=2-3]
            \arrow[from=2-3, to=3-3]
            \arrow["f", from=1-1, to=2-2]
            \arrow["{f'}"', curve={height=-12pt}, from=1-1, to=2-3]
            \arrow["g", curve={height=12pt}, from=1-1, to=3-2]
            \arrow["\lrcorner"{anchor=center, pos=0.125}, draw=none, from=2-2, to=3-3]
        \end{tikzcd}\]
        use the fact $f' \in M(S^q,M(I,X)_*)_* \cong M(S^q \wedge I, X)_* \ni f'' $
        and $S^q \wedge I \approx D^{q+1}$ with
        \begin{align*}
            S^q & \hookrightarrow S^q \wedge I \approx D^{q+1}\\
            s  & \mapsto (s,1)
        \end{align*}
        the condition $f'(s)(1) = g(s)$ is equivalent to $f''((s,1)) = g(s)$,
        that is have a map $f$ is equivalent to have a map $f'' : (D^{q+1},S^q) \to (X,B)$.
        With the analogue statement also valid for homotopies $S^q \times I \to P_{i_{X,B}}$,
        we have $\pi_q(P_{i_{B,X}}) = [S^q,\ast; P_{i_{B,X}}, \ast] \cong [D^{q+1},S^q;X,B] = \pi_{q+1}(X,B)$.\\
        Rewrites the long exact sequence of pairs above to:
        \begin{align*}
            & \cdots \to \pi_{q+1}(X;A,B) \to \pi_{q}(A,C)
            \to \pi_{q} (X,B) \to \pi_{q}(X;A,B)
            \to \pi_{q-1}(A,C) \to \cdots \\
            & \cdots \to
            \pi_{2}(X;A,B) \to \pi_{1}(A,C)
            \to \pi_{1}(X,B) 
        \end{align*}
        Conditions $m \geq 1, n \geq 1$ guarantees
        $\pi_0(C) \to \pi_0 (A)$ and $\pi_0(C) \to \pi_0(B)$
        are surjections.\\
        $m \geq 2$ is equivalent to $\pi_1(A,C) = 0$, which implies $\pi_0(C) \to \pi_0 (A)$ is bijection.\\
        For $x \in \pi_0(A \cap_C B)$, we can always find
        $b \in \pi_0(B), i_{B,X\ *} (b) = x$ or $a \in \pi_0(A), i_{A,X\ *} (a) = x$
        which becomes $b \in \pi_0(B), i_{B,X\ *} (b) = x$ or $c \in \pi_0(C), i_{C,X\ *} (c) = x$
        when  $\pi_0(C) \to \pi_0 (A)$ is bijection.
        That is equivalent to $\pi_0(B) \to \pi_0(X) $ is bijection,
        which means $\pi_1(X,B) = 0$.\\

        We only need to show that for $2 \leq q \leq m+n-2$,
        $\pi_q(X;A,B) = 0$.\\

        With $J^{q-1} := (\partial I^{q-1} \times I) \cup (I^{q-1} \times \{ 0 \})$,
        we have:

        \begin{align*}
            \pi_{q}(P_{i_{B,X}}, P_{i_{C,A}})
            & = [I^{q}, \partial I^{q}, J^{q-1}; P_{i_{B,X}}, P_{i_{C,A}},*]\\
            & = [I^{q} \wedge I;\ I^q, \ \partial I^{q} \wedge I,\ J^{q-1} \wedge I \to X; B, A, \ast]\\
            & := \text{relative homotopy classes of pointed maps $f : I^q \wedge I \to X$ who satisfy:}
            \quad \quad \begin{cases}
                f(I^q) & \subseteq B\\
                f(\partial I^{q} \wedge I) & \subseteq A\\
                f(\partial I^{q}) & \subseteq C\\
                f(J^{q-1} \wedge I) & = *
            \end{cases}\\
            & \quad \quad \text{``relative'' means the homotopy $h$ determine the classes satisfy:}
            \quad \quad \begin{cases}
                h(I^q \times I) & \subseteq B\\
                h((\partial I^{q} \wedge I) \times I) & \subseteq A \\
                h(\partial I^{q} \times I) & \subseteq C\\
                h((J^{q-1} \wedge I) \times I) & = *
            \end{cases}\\
            & \quad \quad \text{(notice that $\partial I^{q} \wedge I \cap I^q = \partial I^q$, therefore $f(\partial I^q) \subseteq A \cap B = C$)}\\
            & \quad \quad \quad \text{(this is called (relative) homotopy class of maps of tetrads)}\\
            & \quad \\
            & = [(I^{q} \times I)/ K;\ I^q \times \{1\},\ (\partial I^{q} \times I) / K,\  (J^{q-1} \times I)  / K \to X; B, A, \ast]\\
            & \quad \quad \text{($K := I^q \times \{0\} \cup \{i_0\} \times I$)}\\
            & = [I^{q+1};\ (I^q \times \{1\}) \cup K,\ (\partial I^{q} \times I) \cup K,\  J^{q-1} \times I \cup K \to X; B, A, \ast]\\
            & \quad \\
            & = [I^{q+1};\ I^q \times \{1\},\ I^{q-1} \times \{1\} \times I,\  J^{q-1} \times I \cup I^q \times \{0\} \to X; B, A, \ast]\\
            & \quad \quad \text{(notice that $\partial I^q = \partial I^{q-1} \times I \cup I^{q-1} \times \{ 0 , 1\} $)}
        \end{align*}

        We can assume that $(A,C)$ have no relative $q<m$-cells and
        $(B,C)$ have no relative $q<n$-cells.\\
        And we can assume that $X$ has finite many cells
        since $I^q$ is compact.\\
        For subcomplexes $C \subseteq A' \subseteq A$,
        where $A = e^m \cup A'$ (attaching one cell from $A'$).\\
        Let $X' := A' \cup_C B$,
        if the results hold for $(X';A',B)$ and $(X;A,X')$,
        then it hold for $(X;A,B)$ since we have
        map between exact homotopy sequences of triples
        $(A,A',C)$ and  $(X,X',B)$:
        \[\begin{tikzcd}
            {\pi_{q+1}(A,A')} & {\pi_q(A',C)} & {\pi_q(A,C)} & {\pi_q(A,A')} & {\pi_{q-1}(A',C)} \\
            {\pi_{q+1}(X,X')} & {\pi_q(X',B)} & {\pi_q(X,B)} & {\pi_q(X,X')} & {\pi_{q-1}(X',B)}
            \arrow["{i_{2,q+1}}"', from=1-1, to=2-1]
            \arrow[from=1-1, to=1-2]
            \arrow[from=2-1, to=2-2]
            \arrow[from=2-2, to=2-3]
            \arrow[from=1-2, to=1-3]
            \arrow[from=1-3, to=1-4]
            \arrow[from=2-3, to=2-4]
            \arrow[from=1-4, to=1-5]
            \arrow[from=2-4, to=2-5]
            \arrow["{i_{1,q-1}}"', from=1-5, to=2-5]
            \arrow["{i_{2,q}}"', from=1-4, to=2-4]
            \arrow["{i_{3,q}}"', from=1-3, to=2-3]
            \arrow["{i_{1,q}}"', from=1-2, to=2-2]
        \end{tikzcd}\]
        induced by inclusion $(A,A',C) \hookrightarrow (X,X',B)$.
        If the result hold for $(X';A',B)$ and $(X;A,X')$,
        maps  $i_{1,q},\ i_{2,q}$
        are isomorphisms when $1 \geq q \geq m+n-3$,
        are epimorphisms when $q = m+n-2$.
        Notice the $5$-lemma says that\\
        if $i_{1,q}$ and $i_{2,q}$ are epimorphisms, $i_{1,q-1}$ are monomorphism, then $i_{3,q}$ is epimorphism.\\
        if $i_{1,q}$ and $i_{2,q}$ are monomorphisms, $i_{2,q+1}$ are epimorphism, then $i_{3,q}$ is monomorphism.\\
        We also have if $C \subseteq B' \subseteq B$ with $B = B' \cup e^n$,
        the result hold for CW-triads $(X';A,B')$ and $(X;X',B)$
        where $X' = A \cup_C B'$,
        since $(A,C) \hookrightarrow (X,B)$ factors as
        $(A,C) \hookrightarrow (X',B') \hookrightarrow (X,B)$.\\

        Now we can assume that $A = C \cup D^m$ and $B = C \cup D^n$.\\

        The current goal of proof is to prove any
        $$
        f : (I^{q+1};\ I^q \times \{1\},\ I^{q-1} \times \{1\} \times I,\  J^{q-1} \times I \cup I^q \times \{0\}) \to (X; B, A, \ast)
        $$
        is nullhomotopic for any $q+1$ with $2 \leq q+1 \leq m+n-2$.\\
        For $a \in \interior{D^m}$, $b\in \interior{D^n}$ We have inclusions of based triads:
        $$
            (A;A,A-\{a\}) \hookrightarrow (X - \{b\};X -\{b\}, X - \{a,b\})
            \hookrightarrow (X;X-\{b\},X-\{a\}) \hookleftarrow (X;A,B)
        $$
        The first and the third induces
        isomorphisms on homotopy groups of triads
        since $B$ is a strong deformation retract of $X - \{a\}$ in $X$
        and $A$ is a strong deformation retract of $X - \{b\}$ in $X$.
        $\pi_{*}(A;A,A - \{a\}) = 0$ since
        $\pi_*(A,A - \{a\}) \to \pi_*(A,A \cap (A - \{a\}) )$
        are isomorphisms.\\
        
        Current goal : choose good $a,b$ to show $f$ regarded as a pointed traid map to $(X;X-\{b\},X-\{a\})$
        is homotopic to a map
        $$
        f' : (I^{q+1};\ I^{q-1} \times \{1\} \times I,\ I^q \times \{1\},\  J^{q-1} \times I \cup I^q \times \{0\}) \to (X - \{b\}; X-\{b\}, X-\{a,b\}, \ast)
        $$
        if $2 \leq q+1 \leq m+n-2$.\\

        \begin{note}
            We want to homotopically remove some point $f^{-1}(b)$,
        first we may want to construct some Uryssohn function $u$
        separating $f^{-1}(a) \cup J^{q-1} \times I \cup I^q \times \{0\}$ and $f^{-1}(b)$ 
        and construct homotopy of cube $h^+: (r,s,t) \mapsto (r,(1-u(r,s)t)s)$
        wishing that $f(h^+(r,s,1))$ would miss $b$.
        The problem in this method is that
        points $f^{-1}(b)$ in the cube would be homotopically
        replaced by other points.
        Since our desire homotopy does not change the first $q$
        coordinates of the cube, we want to separate
        $p^{-1}(p(f^{-1}(a))) \cup J^{q-1} \times I$
        and
        $p^{-1}(p(f^{-1}(b)))$
        (where $p : I^q \times I \to I^q$).
        Our problem is to find suitable $a$, $b$ such that
        $p(f^{-1}(a)) \cap p(f^{-1}(b)) = \emptyset$.\\
        \end{note}
        
        We use manifold structure on $D^m$ and $D^n$ to achieve it,
        now we homotopically approximate $f$ by a map $g$ which 
        smooth on $f^{-1}(D^m_{<1/2} \cup D^n_{<1/2})$.\\

        Let $U_{<r} := f^{-1}(D^m_{<r} \cup D^n_{<r})$,
        Use smooth deformation theorem to construct smooth map (for any $0 < \epsilon$)
        $g' : U_{<3/4} \to D^m_{<3/4} \cup D^n_{<3/4}$ with homotopy
        $h_1 : g' \simeq f|_{U_{<3/4}}$ (and bound $|g'(x) - f(x)| < \epsilon$ for any $x \in U_{<1}$)
        and take partition of unity
        $\{\rho, \rho'\}$ with subcoordinates $\{I^{q+1} - \ol{U_{<1/2}},\ U_{<3/4}\}$,
        we have:
        \begin{align*}
            g & := \rho f + \rho' g' \\
            h_2 & : g \simeq f \rel (I^{q+1} - U_{<3/4}) \\
            h_2 & : I^{q+1} \times I \to X\\
            & \quad (x,t) \mapsto \rho(x)f(x) + \rho'(x)h_1(x,t)\\
        \end{align*}
        with scalar multiplication and addition is already defined on smooth structure on $D^m_{<3/4} \cup D^n_{<3/4}$.\\
        We could assmue that $g(I^{q-1} \times \{1\} \times I) \cap D^n_{<1/2} = \emptyset$
        (which implies $g$ is a map of tetrads to $(X;X-\{b\},X-\{a\},\ast)$)
        and $g(I^{q} \times \{1\}) \cap D^m_{<1/2} = \emptyset$
        since $f(I^{q-1} \times \{1\} \times I) \subseteq A$ and $f(I^q\times \{1\}) \subseteq B$
        and we can always tighten the bound $\epsilon$,
        (Similar argument also hold for $h_2$,
        then we have $h_2 : g \simeq f$ as homotopy between maps of tetrads.)\\

        Use the manifold structure to find good $(a,b)$:\\
        $V := g^{-1}(D^m_{<1/2}) \times g^{-1}(D^n_{<1/2})$
        is a sub-manifold of $I^{2(q+1)}$.
        Consider $W := \{ (v,v') \in V \mid p(v)=p(v') \}$,
        which is the zero set of smooth submersion
        $(v,v') \mapsto p(v) - p(v')$.
        $W$ is smooth manifold with codimension $q$.
        Therefore the map $(g,g) : W \to D^m_{<1/2} \times D^n_{<1/2}$
        is smooth map between manifolds of dimension $q+2$ and $m+n$.
        The map is not surjection since $q+2 < m+n$.
        Then we have $(a,b) \notin (g,g)(W)$
        (that is, $p(g^{-1}(a)) \cap p(g^{-1}(b))$).\\

        Since $g(I^{q-1} \times \{1\} \times I) \cap D^n_{<1/2} = \emptyset$
        and $g(J^{q-1} \times I) \cap D^n_{<1/2} = \emptyset$,
        we have $g(\partial I^q \times I) \cap D^n_{<1/2} = \emptyset$.
        By Uryssohn's lemma, we have $u : I^{q} \to I$ separating
        $p(g^{-1}(a)) \cup \partial I^q$ and $p(g^{-1}(b))$.
        Finally we have:
        \begin{align*}
            h' : I^q \times I \times I & \to I^q \times I\\
            (r,s,t) & \mapsto (r,(1-u(r)t)s)
        \end{align*}
        and $h := g \circ h'$, $f' := h(-,1)$.
        $f'(I^{q+1}) \cap \{b\} = \emptyset$ since if
        $\E (r,s) \in I^q \times I, \ f'(r,s)=b$, then
        $b = g(r,(1-u(r))s) = g(r,0) = \ast$ leads to contradiction.\\
        Last step is to check that $h$ is a homotopy between maps 
        $$
        (I^{q+1};\ I^{q-1} \times \{1\} \times I,\ I^q \times \{1\},\  J^{q-1} \times I \cup I^q \times \{0\}) \to (X; X-\{b\}, X-\{a\}, \ast)
        $$
        Since $g$ is, $g \circ h'$ is too.\\
        \qed

    \end{prf}

    \begin{cor}
        Suppose that $Y_0 \hookrightarrow Y$ is cofibration,
        $(Y,Y_0)$ is $(r-1)$-connected and $Y_0$ is $(s-1)$-connected,
        then $(Y,Y_0) \to (Y/Y_0 , \ast)$ is $(r+s-1)$-equivalence.
        ($r \geq 2,\ s \geq 1$)
    \end{cor}

    \begin{prf}
        $Y_0 \hookrightarrow C Y_0$ is cofibration and $(C Y_0 , Y_0)$
        is $s$-connected.
        Use \hyperref[hyp:Homotopy-Excision-Theorem]{homotopy excision theorem} (with
        $X= Y \cup C Y_0,\ A = Y,\ B=C Y_0,\ C = Y_0$)
        to see $(Y,Y_0) \to (Y \cup C Y_0, C Y_0)$
        is $(r+s-1)$-equivalence.
        And $(Y \cup C Y_0, C Y_0) \to (Y/Y_0, \ast)$ is homotopy equivalence
        since $Y_0 \hookrightarrow Y$ is cofibration.\\
        \qed
    \end{prf}

    \begin{cor}
        For $n \geq 2$, $f : X \to Y$ is $(n-1)$-equivalence
        between $(s-1)$-connected spaces,
        then $(M_f,X) \to (C^+_f, \ast)$ is $(n+s-1)$-equivalence.
        Where $C^+_f := Y \cup_f C^+X $, $C^+X := (X \times I) / (X \times \{1\})$.
        is the unreduced mapping cone and the unreduced cone.
    \end{cor}

    \begin{prf}
        $f$ is $(n-1)$-equivalence implies $(M_f,X)$ is $(n-1)$-connected.
        Use corollary above.\\
        \qed
    \end{prf}

    \begin{cor}
        For $n\geq 2$,
        if $f : X \to Y$ is pointed map between
        $(n-1)$-connected well-pointed spaces
        (that is, pointed space whose inclusion of the base point is cofibration).
        Then $C_f$ is $(n-1)$-connected and $\pi_n(M_f,X) \to \pi_n(C_f, \ast)$
        is isomorphism.
    \end{cor}

    \begin{prf}
        Use homotopy extension property to extend to unreduced case.
        $f$ is map between $(n-1)$-connected space
        implies $f$ is at least a $(n-1)$-equivalence.
        Therefore $(M_f,X) \to (C_f,\ast)$ is $(2n-1)$-equivalence,
        Since we have $n < 2n-1$ for any $n \geq 2$,
        $\pi_n(M_f,X) \to \pi_n(C_f, \ast)$ is isomorphism.\\
        \qed
    \end{prf}

    \begin{thm}
        (Freudenthal Suspension Theorem)
        If $X$ is well-pointed and $(n-1)$-connected ($n \geq 1$),
        then the map:
        \begin{align*}
            \sigma : \pi_q(X) & \to \pi_{q+1}(\Sigma X) \\
            f & \mapsto \Sigma f
        \end{align*}
        is isomorphism if $q < 2n-1$
        and epimorphism if $q = 2n-1$.
    \end{thm}

    \begin{prf}
        If we have $f : (I^q, \partial I^q) \to (X, \ast)$
        then $f\times \id_I : I^{q+1} \to X \times I$
        will give a map
        $\ol{f \times \id_I} : (I^{q+1},\ \partial I^{q+1},\ \partial I^q \times I \cup \partial I \times \{1\}) \to (CX,X,\ast)$
        since $J^q = \partial I^q \times I \cup \partial I \times \{0\}$,
        it does not give a map in $\pi_{q+1}(CX,X)$.
        we should change $\ol{f \times \id_I}$ into $\ol{f \times -\id_I}$.
        we have commutative diagram:
        \[\begin{tikzcd}
            {\pi_{q+1}(CX,X)} & {\pi_{q+1}(CX/X,\ast)} && {[\ol{f \times -\id_I}]} & {[p \circ (\ol{f \times -\id_I})]} \\
            {\pi_{q}(X)} & {\pi_{q+1}(\Sigma X)} && {[f]} & {[- \Sigma f]}
            \arrow["{p_*}", from=1-1, to=1-2]
            \arrow[Rightarrow, no head, from=1-2, to=2-2]
            \arrow["\partial"', curve={height=6pt}, from=1-1, to=2-1]
            \arrow["{-\sigma}"', from=2-1, to=2-2]
            \arrow[curve={height=6pt}, maps to, from=1-4, to=2-4]
            \arrow[maps to, from=2-4, to=2-5]
            \arrow[maps to, from=1-4, to=1-5]
            \arrow[Rightarrow, no head, from=1-5, to=2-5]
            \arrow["i"', curve={height=6pt}, from=2-1, to=1-1]
            \arrow[curve={height=6pt}, maps to, from=2-4, to=1-4]
        \end{tikzcd}\]
        Where $p : (CX,X) \to (CX/X, \ast)$
        is the canonical quotient map and 
        $i : [f] \to [\ol{f \times -\id_{I}}]$
        makes $\pi_{q+1}(CX) \to \pi_{q+1}(CX,X) \to \pi_q(X) \to \pi_q(CX)$
        split in middle
        (that is, $i$ is inverse of the connecting homomorphism $\partial$).
        We verify the commutativity:
        \begin{align*}
            -\Sigma f : (I^{q+1}, \partial I^{q+1}) & \to (CX/X, \ast)\\
            (s,t) & \mapsto f(s) \wedge (1-t)\\
            p \circ (\ol{f \times -\id_I}) : (I^{q+1}, \partial I^{q+1}) & \to (CX/X, \ast)\\
            (s,t) & \mapsto f(s) \wedge (1-t)
        \end{align*}
        Since $X \hookrightarrow CX$ is cofibration
        and $n$-equivalence between $(n-1)$-connected spaces,
        $p$ is an $2n$-equivalence.
        Therefore, $q+1 < 2n$ implies $-\sigma$ is isomorphism,
        $q+1 = 2n$ implies $-\sigma$ is epimorphism,
        and we have $-\sigma$ is iff $\sigma$ is.\\
        \qed
        
    \end{prf}

    \begin{defn}
        We now define the $q$-th stable homotopy group:
        $$
        \pi^s_k(X) := \Colim_{r} \pi_{k+r} (\Sigma^r X) \cong \pi_{2k+2} (\Sigma^{k+2} X) \cong \pi_{k+n} (\Sigma^{n} X) \quad \quad (n-1 > k)
        $$
        The relation right side is directly from $\Sigma^n X$ is $(n-1)$-connected.
    \end{defn}

    \begin{note}
        We'll see later that $\{\pi^s_n\}_{n \in \bbN}$ defines a
        generalized homology theory.
    \end{note}

\end{document}