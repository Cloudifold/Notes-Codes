\documentclass[a4paper]{article}
\usepackage[dvipsnames, svgnames, x11names]{xcolor}
\usepackage{tikz-cd}
\tikzcdset{row sep/normal=1cm}
\tikzcdset{column sep/normal=1cm}
\usepackage{amsfonts}
\usepackage{hyperref}
\usepackage{amsthm}
\usepackage{enumerate}
\usepackage[left=3cm, right=3cm, top=2cm]{geometry}
\usepackage{sectsty}
\usepackage[perpage]{footmisc}
\usepackage{amsthm}
\usepackage{tikz-cd}
\usepackage{pgfplots}
\usepackage{amsmath}
\usepackage{amssymb}
\usepackage{pdfpages}
\usepackage{multirow}
\usepackage{mathtools}
\usepackage{mathrsfs}
\usepackage{ebproof}
\usepackage{graphicx}
\usepackage{epigraph}
\usepackage{enumitem}
\usepackage{indentfirst}

\usepackage{quiver}


\theoremstyle{plain}
\newtheorem{thm}{Theorem}[section]
\newtheorem{lem}[thm]{Lemma}
\newtheorem{prop}[thm]{Proposition}
\newtheorem*{cor}{Corollary}

\theoremstyle{definition}
\newtheorem{defn}{Definition}[section]
\newtheorem{conj}{Conjecture}[section]
\newtheorem{exmp}{Example}[section]
\newtheorem{intro}[defn]{Introduction}

\newtheorem*{prf}{Proof}

\theoremstyle{remark}
\newtheorem*{rem}{Remark}
\newtheorem*{note}{Note}
\newtheorem{exercise}{Exercise}[section]
\newtheorem{solution}{Solution}[section]

\definecolor{background}{RGB}{255,230,230}
\pagecolor{background}

\tikzcdset{background color = background}

\definecolor{TextColor}{RGB}{255,122,162}

\chapterfont{\color{TextColor}}
\sectionfont{\color{TextColor}\Huge}
\setlist{nolistsep,topsep=-7pt}
\pgfplotsset{width=10cm,compat=newest}
%\usepgfplotslibrary{external}
%\tikzexternalize[prefix=tikz/]
\newcommand{\ssubparagraph}{ \parshape 1 1cm \dimexpr\linewidth-2cm\relax}
%\numberwithin{equation}{section}

% \quad : width of  1 'M'
% \qquad : 4
% \0x20 (\ ) : 1/3
% \; : 2/7
% \, : 1/6
% \! : -1/6

% \addtocounter{<envname>}{nubmer}
% Expl. \addtocounter{thm}{2}
% \setcounter{<envname>}{nubmer}
% Expl. \setcounter{ex}{3}
% -->  Exercise 1.4

\newcommand{\setexcounter}[1]{
    \setcounter{ex}{#1}
    \setcounter{sol}{#1}
}
\newcommand{\setsubsectcounter}[1]{\setcounter{subsection}{#1}}

\newcommand{\bb}[1]{\mathbb{#1}}
\newcommand{\mc}[1]{\mathcal{#1}}
\newcommand{\mf}[1]{\mathfrak{#1}}
\newcommand{\ms}[1]{\mathscr{#1}}
\newcommand{\mbf}[1]{\mathbf{#1}}
\newcommand{\bbA}{\mathbb A}
\newcommand{\bbB}{\mathbb B}
\newcommand{\bbC}{\mathbb C}
\newcommand{\bbD}{\mathbb D}
\newcommand{\bbE}{\mathbb E}
\newcommand{\bbF}{\mathbb F}
\newcommand{\bbG}{\mathbb G}
\newcommand{\bbH}{\mathbb H}
\newcommand{\bbI}{\mathbb I}
\newcommand{\bbJ}{\mathbb J}
\newcommand{\bbK}{\mathbb K}
\newcommand{\bbL}{\mathbb L}
\newcommand{\bbM}{\mathbb M}
\newcommand{\bbN}{\mathbb N}
\newcommand{\bbO}{\mathbb O}
\newcommand{\bbP}{\mathbb P}
\newcommand{\bbQ}{\mathbb Q}
\newcommand{\bbR}{\mathbb R}
\newcommand{\bbS}{\mathbb S}
\newcommand{\bbT}{\mathbb T}
\newcommand{\bbU}{\mathbb U}
\newcommand{\bbV}{\mathbb V}
\newcommand{\bbW}{\mathbb W}
\newcommand{\bbX}{\mathbb X}
\newcommand{\bbY}{\mathbb Y}
\newcommand{\bbZ}{\mathbb Z}
\newcommand{\mcA}{\mc A}
\newcommand{\mcB}{\mc B}
\newcommand{\mcC}{\mc C}
\newcommand{\mcD}{\mc D}
\newcommand{\mcE}{\mc E}
\newcommand{\mcF}{\mc F}
\newcommand{\mcG}{\mc G}
\newcommand{\mcH}{\mc H}
\newcommand{\mcI}{\mc I}
\newcommand{\mcJ}{\mc J}
\newcommand{\mcK}{\mc K}
\newcommand{\mcL}{\mc L}
\newcommand{\mcM}{\mc M}
\newcommand{\mcN}{\mc N}
\newcommand{\mcO}{\mc O}
\newcommand{\mcP}{\mc P}
\newcommand{\mcQ}{\mc Q}
\newcommand{\mcR}{\mc R}
\newcommand{\mcS}{\mc S}
\newcommand{\mcT}{\mc T}
\newcommand{\mcU}{\mc U}
\newcommand{\mcV}{\mc V}
\newcommand{\mcW}{\mc W}
\newcommand{\mcX}{\mc X}
\newcommand{\mcY}{\mc Y}
\newcommand{\mcZ}{\mc Z}
\newcommand{\msA}{\ms A}
\newcommand{\msB}{\ms B}
\newcommand{\msC}{\ms C}
\newcommand{\msD}{\ms D}
\newcommand{\msE}{\ms E}
\newcommand{\msF}{\ms F}
\newcommand{\msG}{\ms G}
\newcommand{\msH}{\ms H}
\newcommand{\msI}{\ms I}
\newcommand{\msJ}{\ms J}
\newcommand{\msK}{\ms K}
\newcommand{\msL}{\ms L}
\newcommand{\msM}{\ms M}
\newcommand{\msN}{\ms N}
\newcommand{\msO}{\ms O}
\newcommand{\msP}{\ms P}
\newcommand{\msQ}{\ms Q}
\newcommand{\msR}{\ms R}
\newcommand{\msS}{\ms S}
\newcommand{\msT}{\ms T}
\newcommand{\msU}{\ms U}
\newcommand{\msV}{\ms V}
\newcommand{\msW}{\ms W}
\newcommand{\msX}{\ms X}
\newcommand{\msY}{\ms Y}
\newcommand{\msZ}{\ms Z}
\newcommand{\mfA}{\mf A}
\newcommand{\mfB}{\mf B}
\newcommand{\mfC}{\mf C}
\newcommand{\mfD}{\mf D}
\newcommand{\mfE}{\mf E}
\newcommand{\mfF}{\mf F}
\newcommand{\mfG}{\mf G}
\newcommand{\mfH}{\mf H}
\newcommand{\mfI}{\mf I}
\newcommand{\mfJ}{\mf J}
\newcommand{\mfK}{\mf K}
\newcommand{\mfL}{\mf L}
\newcommand{\mfM}{\mf M}
\newcommand{\mfN}{\mf N}
\newcommand{\mfO}{\mf O}
\newcommand{\mfP}{\mf P}
\newcommand{\mfQ}{\mf Q}
\newcommand{\mfR}{\mf R}
\newcommand{\mfS}{\mf S}
\newcommand{\mfT}{\mf T}
\newcommand{\mfU}{\mf U}
\newcommand{\mfV}{\mf V}
\newcommand{\mfW}{\mf W}
\newcommand{\mfX}{\mf X}
\newcommand{\mfY}{\mf Y}
\newcommand{\mfZ}{\mf Z}
\newcommand{\mfa}{\mf a}
\newcommand{\mfb}{\mf b}
\newcommand{\mfc}{\mf c}
\newcommand{\mfd}{\mf d}
\newcommand{\mfe}{\mf e}
\newcommand{\mff}{\mf f}
\newcommand{\mfg}{\mf g}
\newcommand{\mfh}{\mf h}
\newcommand{\mfi}{\mf i}
\newcommand{\mfj}{\mf j}
\newcommand{\mfk}{\mf k}
\newcommand{\mfl}{\mf l}
\newcommand{\mfm}{\mf m}
\newcommand{\mfn}{\mf n}
\newcommand{\mfo}{\mf o}
\newcommand{\mfp}{\mf p}
\newcommand{\mfq}{\mf q}
\newcommand{\mfr}{\mf r}
\newcommand{\mfs}{\mf s}
\newcommand{\mft}{\mf t}
\newcommand{\mfu}{\mf u}
\newcommand{\mfv}{\mf v}
\newcommand{\mfw}{\mf w}
\newcommand{\mfx}{\mf x}
\newcommand{\mfy}{\mf y}
\newcommand{\mfz}{\mf z}
\newcommand{\mbfA}{\mathbf A}
\newcommand{\mbfB}{\mathbf B}
\newcommand{\mbfC}{\mathbf C}
\newcommand{\mbfD}{\mathbf D}
\newcommand{\mbfE}{\mathbf E}
\newcommand{\mbfF}{\mathbf F}
\newcommand{\mbfG}{\mathbf G}
\newcommand{\mbfH}{\mathbf H}
\newcommand{\mbfI}{\mathbf I}
\newcommand{\mbfJ}{\mathbf J}
\newcommand{\mbfK}{\mathbf K}
\newcommand{\mbfL}{\mathbf L}
\newcommand{\mbfM}{\mathbf M}
\newcommand{\mbfN}{\mathbf N}
\newcommand{\mbfO}{\mathbf O}
\newcommand{\mbfP}{\mathbf P}
\newcommand{\mbfQ}{\mathbf Q}
\newcommand{\mbfR}{\mathbf R}
\newcommand{\mbfS}{\mathbf S}
\newcommand{\mbfT}{\mathbf T}
\newcommand{\mbfU}{\mathbf U}
\newcommand{\mbfV}{\mathbf V}
\newcommand{\mbfW}{\mathbf W}
\newcommand{\mbfX}{\mathbf X}
\newcommand{\mbfY}{\mathbf Y}
\newcommand{\mbfZ}{\mathbf Z}

% common math symbols

\newcommand{\bs}{\backslash}
\newcommand{\bbRn}{\mathbb R^n}
\newcommand{\bbCn}{\mathbb R^n}
\newcommand{\ecyc}[1]{\langle #1\rangle}
\newcommand{\interior}[1]{\overset{\circ}{#1}}
\newcommand{\ol}[1]{\overline{#1}}
\newcommand{\unitgrp}[1]{#1^{\mspace{-4mu}\times}}
\newcommand{\grad}{\triangledown}
\newcommand{\normal}{\trianglelefteq}
\newcommand{\nnormal}{\ntrianglelefteq}
\newcommand{\norm}[1]{\left\lVert#1\right\rVert}
\newcommand{\gen}[1]{\langle #1 \rangle}
\newcommand{\nilradical}[1]{\sqrt{\gen{0}}_{#1}}
\newcommand{\inv}[1]{#1^{\text{-}1}}
\newcommand{\id}{\mathrm{id}}
\newcommand{\unitcell}[1]{\overset{\!\!\!\! \circ}{\bbD^{#1}}}
\newcommand{\E}{\exists}
\newcommand{\A}{\forall}
\newcommand{\Fct}{\mathrm{Funct}}

% Operators

\DeclareMathOperator{\GL}{\rm{GL}}
\DeclareMathOperator{\SL}{\rm{SL}}
\DeclareMathOperator{\Mod}{\rm{Mod}}
\DeclareMathOperator{\Mor}{\rm{Mor}}
\DeclareMathOperator{\Obj}{\rm{Obj}}
\DeclareMathOperator{\Id}{\rm{Id}}
\DeclareMathOperator{\End}{\rm{End}}
\DeclareMathOperator{\Hom}{\rm{Hom}}
\DeclareMathOperator{\Res}{\rm{Res}}
\DeclareMathOperator{\Spec}{\rm{Spec}}
\DeclareMathOperator{\Proj}{\rm{Proj}}
\DeclareMathOperator{\Supp}{\rm{Supp}}
\DeclareMathOperator{\Ker}{\rm{Ker}}
\DeclareMathOperator{\Nil}{\rm{Nil}}
\DeclareMathOperator{\sh}{\rm{sh}}
\DeclareMathOperator{\Coker}{\rm{Coker}}
\DeclareMathOperator{\Rank}{\rm{Rank}}
\DeclareMathOperator{\Frac}{\rm{Frac}}
\DeclareMathOperator{\Rad}{\rm{Rad}}
\DeclareMathOperator{\Ann}{\rm{Ann}}
\DeclareMathOperator{\Disc}{\rm{Disc}}
\DeclareMathOperator{\Lcm}{\rm{lcm}}
\DeclareMathOperator{\Gcd}{\rm{gcd}}
\DeclareMathOperator{\Conv}{Conv}
\DeclareMathOperator{\Cone}{Cone}
\DeclareMathOperator{\Int}{Int}
\DeclareMathOperator{\Ord}{ord}
\DeclareMathOperator{\Ass}{Ass}
\DeclareMathOperator{\Aut}{Aut}
\DeclareMathOperator{\Sym}{Sym}
\DeclareMathOperator{\Char}{Char}
\DeclareMathOperator{\Span}{Span}
\DeclareMathOperator{\Tor}{Tor}
\DeclareMathOperator{\Ext}{Ext}
\DeclareMathOperator{\Card}{Card}
\DeclareMathOperator{\OPT}{OPT}
\DeclareMathOperator{\Dom}{Dom}
\DeclareMathOperator{\Var}{Var}
\DeclareMathOperator{\Th}{Th}
\DeclareMathOperator{\Frob}{Frob}
\DeclareMathOperator{\Red}{Red}
\DeclareMathOperator{\Aff}{Aff}
\DeclareMathOperator{\Epi}{Epi}
\DeclareMathOperator{\sech}{sech}
\DeclareMathOperator{\csch}{csch}
\DeclareMathOperator*{\Argmin}{Argmin}
\DeclareMathOperator{\Zer}{Zer}
\DeclareMathOperator{\Ri}{Ri}
\DeclareMathOperator{\Prox}{Prox}
\DeclareMathOperator{\sgn}{sgn}
\DeclareMathOperator{\Fix}{Fix}
\DeclareMathOperator{\Gal}{Gal}
\renewcommand{\Re}{\operatorname{Re}}
\renewcommand{\Im}{\operatorname{Im}}
\DeclareMathOperator{\Adj}{Adj}
\renewcommand{\div}{\mathrm{div}}
\DeclareMathOperator{\Div}{Div}
\DeclareMathOperator{\Cl}{Cl}
\DeclareMathOperator{\CDiv}{CDiv}
\DeclareMathOperator{\Bl}{Bl}
\DeclareMathOperator{\codim}{codim}
\DeclareMathOperator{\Sing}{Sing}
\DeclareMathOperator{\Nef}{Nef}
\DeclareMathOperator{\NE}{NE}
\DeclareMathOperator{\Mult}{Mult}
\DeclareMathOperator{\Pic}{Pic}
\DeclareMathOperator{\Tr}{Tr}
\DeclareMathOperator{\Grass}{Grass}
\DeclareMathOperator{\LinSub}{LinSub}
\DeclareMathOperator{\Eq}{Eq}
\DeclareMathOperator{\can}{can}
\DeclareMathOperator{\Rat}{Rat}
\DeclareMathOperator{\trdeg}{trdeg}
\DeclareMathOperator{\QSym}{QSym}
\DeclareMathOperator{\ASym}{ASym}
\DeclareMathOperator{\des}{des}
\DeclareMathOperator{\exc}{exc}
\DeclareMathOperator{\maj}{maj}
\DeclareMathOperator{\wt}{wt}
\DeclareMathOperator{\SST}{SST}
\DeclareMathOperator{\ASST}{\mfA SST}
\DeclareMathOperator{\SSYT}{SSYT}
\DeclareMathOperator{\ST}{ST}
\DeclareMathOperator{\SYT}{SYT}
\DeclareMathOperator{\SV}{SV}
\DeclareMathOperator{\ex}{ex}
\DeclareMathOperator{\sort}{sort}
\DeclareMathOperator{\sq}{sq}
\DeclareMathOperator{\invcode}{invcode}
\DeclareMathOperator{\Ess}{Ess}
\DeclareMathOperator{\Flag}{Flag}
\DeclareMathOperator{\Stab}{Stab}
\DeclareMathOperator{\Orb}{Orb}
\DeclareMathOperator{\pl}{pl}
\DeclareMathOperator{\Mat}{Mat}
\DeclareMathOperator{\pt}{pt}
\DeclareMathOperator{\depth}{depth}
\DeclareMathOperator{\cyc}{cyc}
\DeclareMathOperator{\ev}{ev}
\DeclareMathOperator{\length}{lg}
\DeclareMathOperator{\Quot}{Quot}
\DeclareMathOperator{\Hilb}{Hilb}
\DeclareMathOperator{\PGL}{PGL}
\DeclareMathOperator{\PSL}{PSL}
\newcommand{\Cat}[1]{(\textrm{#1})}

\begin{document}
    \title{CW complexes}
    \author{{\color{pink}{Cloudi}}{\color{Aquamarine}{fold}}}
    \maketitle
    \newpage

    \setcounter{section}{-1}

    \section{Constructions and Definitions}
    \begin{defn}
        A CW-complex is A space Constructed by Successively attaching cells:\\
        For $n \in \bbN, n \geq 1$, there are maps $\{ \varphi_i : S^{n-1} \to X^{n-1} \}_{i \in I_n}$
        (called characteristic maps). The way to construct $X^n$ (called $n$-skeleton of $X$) is :\\
        (starting from $X^0 = \prod_{I_0} \ast $)

        \[\begin{tikzcd}
            {\prod_{i \in I_n}S^{n-1}} & {X^{n-1}} \\
            {\prod_{i\in I_n}D^n} & {X^n}
            \arrow[hook, from=1-1, to=2-1]
            \arrow[from=2-1, to=2-2]
            \arrow["{\prod_{i \in I_n}\varphi_i}", from=1-1, to=1-2]
            \arrow[hook, from=1-2, to=2-2]
            \arrow["\lrcorner"{anchor=center, pos=0.125, rotate=180}, draw=none, from=2-2, to=1-1]
        \end{tikzcd}(pushout)\]\\
        and the resulting CW-complex $X$ is $\Colim \{X^0 \to \cdots \to X^n \to X^{n+1} \to \cdots \}$.
        The images of $\interior{D^n_i}$ in $X$ is called open cell $e^n_i$ of $X$.
    \end{defn}

    \begin{defn}
        $A$ is a subcomplex of CW-complex $X$ iff for any open cell $e^n_i$ of $X$ satisfy:
        $A \cap e^n_i \neq \emptyset \implies \bar{e^n_i} \subseteq A $.\\
        Pair of $X$ and subcomplex $A$ : $(X,A)$ is called a CW-pair.
    \end{defn}

    \begin{defn}
        The Infinite Symmetric Product of a pointed space $(X,x_0)$ is colimit of its $n$-th Symmetric Products
        ( $\SP^n X := (\prod_{\{0,1,\dots , n-1\}} X ) / S_n$ ) :\\
        \begin{align*}
            \Colim \{\dots \hookrightarrow \SP^{n} X & \hookrightarrow \SP^{n+1} X \hookrightarrow \cdots \} \\
            \{x_1, \dots, x_n\} & \mapsto \{x_0, x_1, \dots, x_n\}    
        \end{align*}
    \end{defn}



    \section{Properties and Examples}

    \begin{thm}
        Homotopy Extension and Lifting property\label{hyp:HELP}:\\
        $A$ : a topological space\\
        $X$ : result of successively attaching cells on $A$ of dimensions $0,1,\dots,k$ ($k \leq n$)\\
        $e : Y \to Z$ : $n$-equivalence\\
        $g : A \to Y$, $f : X \to Z$\\
        $h : f|_A  \simeq e \circ g $\\

        \[\begin{tikzcd}
            A & X \\
            Y & Z
            \arrow["g"', from=1-1, to=2-1]
            \arrow[hook, from=1-1, to=1-2]
            \arrow["f", from=1-2, to=2-2]
            \arrow["e"', from=2-1, to=2-2]
            \arrow["h"', shorten <=7pt, shorten >=7pt, Rightarrow, from=1-2, to=2-1]
        \end{tikzcd}\]\\
        Then there exists
        $g^+ : X \to Y$ extends $g$ ($g^+|_A = g$)\\
        and $h^+ : X \times I \to Z$ extends $h$, $h^+ : f \simeq e \circ g^+$
        \[\begin{tikzcd}
            A & X \\
            Y & Z
            \arrow["g"', from=1-1, to=2-1]
            \arrow[hook, from=1-1, to=1-2]
            \arrow[""{name=0, anchor=center, inner sep=0}, "f", from=1-2, to=2-2]
            \arrow["e"', from=2-1, to=2-2]
            \arrow["{g^+}"', from=1-2, to=2-1]
            \arrow["{h^+}", shorten <=9pt, shorten >=9pt, Rightarrow, from=0, to=2-1]
        \end{tikzcd}\]
    \end{thm}

    \begin{prf}
        It suffices to prove the case $A = S^{k-1}, X = D^k$
        , $e$ is inclusion. (replace $Z$ by $M_e$)
        Apply HEP of $(D^k,S^{k-1})$:\\
        \[\begin{tikzcd}
            {S^{k-1}} & {D^k} \\
            {S^{k-1}\times I} & {D^k \times I} \\
            && Z
            \arrow[from=1-1, to=1-2]
            \arrow[from=1-1, to=2-1]
            \arrow[from=2-1, to=2-2]
            \arrow[from=1-2, to=2-2]
            \arrow["h"', curve={height=12pt}, from=2-1, to=3-3]
            \arrow["f", curve={height=-12pt}, from=1-2, to=3-3]
            \arrow["{\hat{h}}", dashed, from=2-2, to=3-3]
        \end{tikzcd}\]\\
        $f' := \hat{h}(-,1)$, replace $f$ with $f'$ the diagram would be strictly commute.
        Therefore, $f'$ is map of pairs $(D^k,S^{k-1}) \to (Z,Y)$,
        $k\leq n $ implies $f'$ is nullhomotopic, suppose $h^+ : D^k \times I \to Z$ is the nullhomotopy,
        then $g^+ := h^+(-,1)$ satisfy $ g^+(D^k) \subseteq Y $.\\
        \qed
    \end{prf}

    \begin{note}
        In HELP\ref{hyp:HELP}, at condition $Y=Z$ and $e = \id$,
        HELP says $(X,A)$ have HEP
    \end{note}

    \begin{cor}
        If \\
        $A$ : a topological space\\
        $X$ : result of successively attaching cells on $A$ of any dimensions\\
        Then, $(X,A)$ have HEP.
    \end{cor}

    \begin{thm}
        If $X$ is an CW-complex, $e : Y \to Z$ is an $n$-equivalence,
        Then $e_{\ast} : [X,Y] \to [X,Z]$ is a bijection if $\dim{X} < n$,
        and a surjection if $\dim{X} = n$.(Also valid for pointed case)
    \end{thm}

    \begin{prf}
        Surjectivity:\\
        Apply HELP of $(X,\emptyset)$ ($(X,x_0)$ for pointed case) to obtain $e_{\ast} [g^+] \simeq [f] $:
        \[\begin{tikzcd}
            \emptyset & X \\
            Y & Z
            \arrow[from=1-1, to=1-2]
            \arrow["f", from=1-2, to=2-2]
            \arrow[from=1-1, to=2-1]
            \arrow["e"', from=2-1, to=2-2]
            \arrow["{g^+}"', from=1-2, to=2-1]
        \end{tikzcd}\]\\
        Injectivity ($\dim{X} < n$):\\
        Suppose $[g_0],[g_1] \in [X,Y]$, $e_{\ast}[g_0] = e_{\ast}[g_1]$.\\
        Let $f : e \circ g_0 \simeq e \circ g_1$
        Apply HELP to $(X \times I, X  \times \partial I)$:
        \[\begin{tikzcd}
            {X\times \partial I} & {X \times I} \\
            Y & Z
            \arrow[from=1-1, to=1-2]
            \arrow["f", from=1-2, to=2-2]
            \arrow["g"', from=1-1, to=2-1]
            \arrow["e"', from=2-1, to=2-2]
            \arrow["{g^+}"', from=1-2, to=2-1]
        \end{tikzcd}\]
        \qed
    \end{prf}

    \begin{cor}
        If $X$ is a CW-complex, $e : Y \to Z$ is weak homotopy equivalence,
        then $e_{\ast} : [X,Y] \to [X,Z]$ is bijection.
    \end{cor}

    \subsection{CW-approximation}

    \begin{defn}
        A \textbf{CW-approximation} of $(X,A) \in Top(2)$ is a CW-pair $(\widetilde{X}, \widetilde{A})$ and a
        weak homotopy equivalence of pairs $\varphi : (\widetilde{X}, \widetilde{A}) \to (X,A)$.
    \end{defn}

    \begin{lem}
        $\varphi, \psi$ are CW-approximations of $X,Y$,
        $f : X \to Y$, then 
        \[\begin{tikzcd}
            {\widetilde{X}} & X \\
            {\widetilde{Y}} & Y
            \arrow["{\exists \widetilde{f}}"', dashed, from=1-1, to=2-1]
            \arrow["\varphi", from=1-1, to=1-2]
            \arrow["f", from=1-2, to=2-2]
            \arrow["\psi"', from=2-1, to=2-2]
        \end{tikzcd}\] commutes up to homotopy,
        and $\widetilde{f}$ is unique up to homotopy.
    \end{lem}

    \begin{proof}
        Directly from $\psi_{\ast} : [\widetilde{X},\widetilde{Y}] \to [\widetilde{X},Y]$ is bijection.
        \qed
    \end{proof}

    \begin{thm}
        $\varphi, \psi$ are CW-approximations of $(X,A), (Y,B)$,
        $f : (X,A) \to (Y,B)$, then 
        \[\begin{tikzcd}
            {(\widetilde{X},\widetilde{A})} & (X,A) \\
            {(\widetilde{Y},\widetilde{B})} & (Y,B)
            \arrow["{\exists \widetilde{f}}"', dashed, from=1-1, to=2-1]
            \arrow["\varphi", from=1-1, to=1-2]
            \arrow["f", from=1-2, to=2-2]
            \arrow["\psi"', from=2-1, to=2-2]
        \end{tikzcd}\] commutes up to homotopy,
        and $\widetilde{f}$ is unique up to homotopy.
    \end{thm}

    \begin{prf}
        Apply \textbf{Lemma 1.3} to obtain map $\widetilde{f}_A : \widetilde{A} \to \widetilde{B}$
        and homotopy $h : \psi |_{\widetilde{B}} \circ \widetilde{f}_A \simeq f \circ \varphi|_{\widetilde{A}}$
        Use HELP of $(\widetilde{X}, \widetilde{A})$ to extend it:
        \[\begin{tikzcd}
            {\widetilde{A}} & {\widetilde{X}} \\
            {\widetilde{B}} & X \\
            {\widetilde{Y}} & Y
            \arrow[hook, from=1-1, to=1-2]
            \arrow["{\widetilde{f}_A}"', from=1-1, to=2-1]
            \arrow[hook, from=2-1, to=3-1]
            \arrow[from=3-1, to=3-2]
            \arrow[from=1-2, to=2-2]
            \arrow[from=2-2, to=3-2]
            \arrow["{\widetilde{f}}"', from=1-2, to=3-1]
        \end{tikzcd}\]
        $\psi_{\ast}$ is bijection implies the uniqueness up to homotopy of $\widetilde{f}$.
        \qed
    \end{prf}

    \begin{thm}
        (Whitehead's Theorem)\\
        Every $n$-equivalence between CW-complexes whose dimension is lower than $n$, is homotopy equivalence.
        Every weak homotopy equivalence between CW-complexes is homotopy equivalence.
    \end{thm}

    \begin{prf}
        $e : Y \to Z$ induce bijections $ [Y,Y] \to [Y,Z]$ and $[Z,Y] \to [Z,Z]$,
        $[f] = e_{\ast}^{-1} [\id_Z]$ implies $[e \circ f] = [\id_Z]$ and $[e \circ f \circ e] = [e]$ ($[f \circ e] = e_{\ast}^{-1}[e] = [\id_Y]$).\\
        \qed
    \end{prf}

    \begin{thm}
        Every weak homotopy equivalences between CW-pairs is a homotopy equivalence.
    \end{thm}

    \begin{cor}
        CW-approximation is unique up to homotopy.
    \end{cor}

    \begin{exmp}
        Polish circle (Warsaw circle) : closed topologist's sine curve.
        It is $n$-connected forall $n$ but not contractible.
    \end{exmp}

    \begin{defn}
        A cellular map between CW-pairs is $g:(X,A) \to (Y,B)$ such that
        $g(A\cup X^n) \subseteq B \cup Y^n$.
    \end{defn}

    \begin{thm}
        For any map between CW-pairs $f : (X,A) \to (Y,B)$ there exists
        a cellular map $g$ such that $g \simeq f \rel A$
    \end{thm}

    \begin{prf}
        Construct $g$ inductively:\\
        Start from $A \cup X^0$:\\
        take paths $\gamma_i : f(x_i) \simeq y_i $, where $y_i$ is any point in $Y^0$ and $x_i \in X^0 - A$.\\
        Construct $h_0 : (X^0 \cup A) \times I \to Y $ : $h_0 |_{A} (a,t):= f(a)$,
        $h_0|_{X^0-A} (x_i,t) := \gamma_i(t)$.
        This is a homotopy from $f$ to $g_0 := h_0(-,1) : A\cup X^0 \to B \cup Y^0$\\
        Inductive step:\\
        Assume $g_n : A\cup X^n \to B \cup Y^n$ and homotopy $h_n : f|_{A \cup X^n} \simeq g_n$ is given, try to construct $g_{n+1}$:\\
        For each characteristic map $\varphi_i : S^n \to X^n$, take the resulting cell map $\varphi^+_i : D^{n+1} \to X^{n+1}$ 
        and use HELP of $(D^{n+1}, S^n)$:
        \[\begin{tikzcd}
            {S^n} & {D^{n+1}} \\
            {X^n} & {X^{n+1}} \\
            {B\cup Y^{n+1}} & Y
            \arrow[hook, from=1-1, to=1-2]
            \arrow["{\varphi_i}"', from=1-1, to=2-1]
            \arrow["{g_n}"', from=2-1, to=3-1]
            \arrow[hook, from=3-1, to=3-2]
            \arrow["{\varphi^+_i}", from=1-2, to=2-2]
            \arrow["f", from=2-2, to=3-2]
            \arrow["{g_{n+1,i}}"', from=1-2, to=3-1]
            \arrow["{h_{n+1,i}}", shorten <=4pt, shorten >=4pt, Rightarrow, from=2-2, to=3-1]
        \end{tikzcd}\]
        Glue all $g_{n+1,i}$ and $ h_{n+1,i}$ to produce $g_{n+1}$ and $h_{n+1} : f|_{A \cup X^{n+1}} \simeq g_{n+1}$.\\
        Final stage:\\
        Maps $g_n$ determine a cellular map $g : X \to Y$ since $X$ has the weak topology determined by skeletons.
        \qed
    \end{prf}

    \subsection{Operation of CW-complexes}

    Product of CW-complexes:

    \begin{exmp}
        Product topology of two CW-complexes does not coincide with weak topology:\\
        $X$ (star of countably many edges) :
        $X = X^1 = \bigvee_{n\in \omega} I_n $\\
        $Y$ (star of $\omega^{\omega}$ many edges) :
        $Y = Y^1 = \bigvee_{f \in \omega^{\omega}} I_f $
        ( $(I_n,0) \cong (I_f,0) \cong (I,0) $ )\\
        Consider subset $H$ of $X \times Y$:
        $H := \{ (\frac{1}{f(n)+1} , \frac{1}{f(n)+1}) \in I_n \times I_f \mid n \in \omega, f \in \omega^{\omega} \}$.\\
        $H$ is closed under the weak topology
        since every cell of $X \times Y$ contains at most one point of $H$.
        But closure of $H$ contains $(0,0)$ at product topology:\\
        Let $U \times V$ be an open neighborhood (at product topology) of $(0,0)$,
        let $g : \omega \to \omega - 0$ be an increasing function such that
        forall $n \in \omega, [0,\frac{1}{g(n)}) \subseteq U \cap I_n$,
        let $k \in omega$ be sufficiently large that $\frac{1}{g(k)+1} \subseteq V \cap I_g$,
        then $(\frac{1}{g(k)+1} , \frac{1}{g(k)+1}) \in U \times V \cap H$.

    \end{exmp}

    \begin{prop}
        $X$ and $Y$ are CW-complexes, $X \times Y$ is CW-complex if\\
        $X$ or $Y$ is locally compact\\
        or\\
        both $X$ and $Y$ have countably many cells.
    \end{prop}

    Quotient of CW-pair:

    \begin{prop}
        For CW-complex $X$ and subcomplex $A$, the Quotient space $X/A$ 
        have a CW-complex structure induced by $X$ and $A$.
    \end{prop}

    \begin{prf}
        Suppose the characteristic maps of $X$ are indexed by $\{I_n\}_{n \in \bbN}$
        and of $A$ are indexed by $\{I'_n\}_{n \in \bbN}$ ($I'_n \subseteq I_n$).
        Then the characteristic maps of $X/A$ are indexed by $\{ K_n\}_{n \in \bbN}$, which defined below:\\
        $K_0 := (I_0 - I'_0) \cup \{ i_0 \} $ where $i_0$ is an arbitrary element
        in $I'_0$\\
        $K_n := I_n - I'_n$ for $n > 0$.\\
        Verify the maps determine the CW-complex structure:\\
        \[\begin{tikzcd}
            {S^{n-1}} & {X^{n-1}} & {X^{n-1}/A^{n-1}} \\
            {D^n} & {X^n} & {X^n/A^{n-1}} & {A^n} \\
            && {X^n/A^n} & \ast \\
            && Z
            \arrow[from=1-1, to=2-1]
            \arrow[from=1-3, to=2-3]
            \arrow[from=2-3, to=3-3]
            \arrow[from=2-4, to=2-3]
            \arrow[from=3-4, to=3-3]
            \arrow["\lrcorner"{anchor=center, pos=0.125, rotate=90}, draw=none, from=3-3, to=2-4]
            \arrow[from=2-4, to=3-4]
            \arrow[curve={height=12pt}, from=2-1, to=4-3]
            \arrow[curve={height=-44pt}, from=1-3, to=4-3]
            \arrow[curve={height=28pt}, dashed, from=2-3, to=4-3]
            \arrow[curve={height=-6pt}, from=3-4, to=4-3]
            \arrow[dashed, from=3-3, to=4-3]
            \arrow[from=2-2, to=2-3]
            \arrow[from=1-2, to=1-3]
            \arrow[from=1-2, to=2-2]
            \arrow[from=1-1, to=1-2]
            \arrow[from=2-1, to=2-2]
            \arrow["\lrcorner"{anchor=center, pos=0.125, rotate=180}, draw=none, from=2-3, to=1-2]
            \arrow["\lrcorner"{anchor=center, pos=0.125, rotate=180}, draw=none, from=2-2, to=1-1]
        \end{tikzcd}\]

        \qed
        
    \end{prf}

    Smash product of CW-complexes:

    \begin{prop}
        If $(X,x_0)$ , $(Y,y_0)$ are pointed CW-complexes with both countably many cell,
        and $X^{r-1} = \{x_0\}$, $Y^{s-1} = \{y_0\}$,
        then $X \wedge Y := X \times Y / X \vee  Y$ is
        an $(r+s-1)$-connected CW-complex.
    \end{prop}

    \begin{prf}
        $X \times Y$ is CW-complex with cells of the form
        $e^n_{i,X} \times \{y_0\}$, $\{x_0\} \times e^m_{j,Y}$
        or $e^n_{i,X} \times e^m_{j,Y}$
        for $n \geq r$, $m \geq s$.
        Cells of the first two forms are contianed in $X \vee Y$,
        therefore $(X \wedge Y)^{r+s-1} = \ast$.
        \qed
    \end{prf}

    \begin{cor}
        If $X$ is a pointed CW-complex,
        then $\Sigma^n X$ is an $(n-1)$-connected CW-complex.
    \end{cor}

    \subsection{Properties of Infinite Symmetric Product}
    Functoriality:\\

    Pointed map $f : X \to Y$
    induces
    \begin{align*}
        f_n : \SP^n X & \to \SP^{n}Y\\
            \{x_1, \dots, x_n\} & \mapsto \{f(x_1), \dots, f(x_n)\}
    \end{align*}
    
    \[\begin{tikzcd}
        {} & {\SP^n X} & {\SP^{n+1}X} & {} \\
        {} & {\SP^nY} & {\SP^{n+1} Y} & {}
        \arrow[from=1-2, to=1-3]
        \arrow[from=1-1, to=1-2]
        \arrow[from=2-1, to=2-2]
        \arrow["{f_n}", from=1-2, to=2-2]
        \arrow[from=2-2, to=2-3]
        \arrow["{f_{n+1}}", from=1-3, to=2-3]
        \arrow[from=2-3, to=2-4]
        \arrow[from=1-3, to=1-4]
    \end{tikzcd}\]

    Which induces map $\SP f : \SP X \to \SP Y$.
    And Functorial properties are directly from the constructions above.\\

    Suppose $i : A \hookrightarrow X$ is an pointed inclusion,
    then $\SP i : \SP A \hookrightarrow \SP X$ is also inclusion.
    Furthermore, if $A$ is open (or closed) in $X$, then $\SP A$ is open (or closed) in $\SP X$.\\

    Pointed homotopy $h : X \times I \to Y$ induces
    \begin{align*}
        h_n : \SP^n X \times I & \to \SP^{n}Y\\
            (\{x_1, \dots, x_n\} , t) & \mapsto \{h(x_1,t), \dots, h(x_n,t)\}
    \end{align*}
    
    which induces $\SP h : \SP X \times I \to \SP Y$.\\


    Then we observe:\\
    $f \simeq g$ implies $\SP f \simeq \SP g$,\\
    $e : X \to Y$ is homotopy equivalence implies $\SP e : \SP X \to \SP Y$ is,\\
    $X$ is contractible implies $\SP^n X$ and $\SP X$ is.

\end{document}