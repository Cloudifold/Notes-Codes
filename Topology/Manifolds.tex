\documentclass[a4paper]{article}
\usepackage[dvipsnames, svgnames, x11names]{xcolor}
\usepackage{tikz-cd}
\tikzcdset{row sep/normal=1cm}
\tikzcdset{column sep/normal=1cm}
\usepackage{amsfonts}
\usepackage{hyperref}
\usepackage{amsthm}
\usepackage{enumerate}
\usepackage[left=3cm, right=3cm, top=2cm]{geometry}
\usepackage{sectsty}
\usepackage[perpage]{footmisc}
\usepackage{amsthm}
\usepackage{tikz-cd}
\usepackage{pgfplots}
\usepackage{amsmath}
\usepackage{amssymb}
\usepackage{pdfpages}
\usepackage{multirow}
\usepackage{mathtools}
\usepackage{mathrsfs}
\usepackage{ebproof}
\usepackage{graphicx}
\usepackage{epigraph}
\usepackage{enumitem}
\usepackage{indentfirst}

\usepackage{quiver}


\theoremstyle{plain}
\newtheorem{thm}{Theorem}[section]
\newtheorem{lem}[thm]{Lemma}
\newtheorem{prop}[thm]{Proposition}
\newtheorem*{cor}{Corollary}

\theoremstyle{definition}
\newtheorem{defn}{Definition}[section]
\newtheorem{conj}{Conjecture}[section]
\newtheorem{exmp}{Example}[section]
\newtheorem{intro}[defn]{Introduction}

\newtheorem*{prf}{Proof}

\theoremstyle{remark}
\newtheorem*{rem}{Remark}
\newtheorem*{note}{Note}
\newtheorem{exercise}{Exercise}[section]
\newtheorem{solution}{Solution}[section]

\definecolor{background}{RGB}{255,230,230}
\pagecolor{background}

\tikzcdset{background color = background}

\definecolor{TextColor}{RGB}{255,122,162}

\chapterfont{\color{TextColor}}
\sectionfont{\color{TextColor}\Huge}
\setlist{nolistsep,topsep=-7pt}
\pgfplotsset{width=10cm,compat=newest}
%\usepgfplotslibrary{external}
%\tikzexternalize[prefix=tikz/]
\newcommand{\ssubparagraph}{ \parshape 1 1cm \dimexpr\linewidth-2cm\relax}
%\numberwithin{equation}{section}

% \quad : width of  1 'M'
% \qquad : 4
% \0x20 (\ ) : 1/3
% \; : 2/7
% \, : 1/6
% \! : -1/6

% \addtocounter{<envname>}{nubmer}
% Expl. \addtocounter{thm}{2}
% \setcounter{<envname>}{nubmer}
% Expl. \setcounter{ex}{3}
% -->  Exercise 1.4

\newcommand{\setexcounter}[1]{
    \setcounter{ex}{#1}
    \setcounter{sol}{#1}
}
\newcommand{\setsubsectcounter}[1]{\setcounter{subsection}{#1}}

\newcommand{\bb}[1]{\mathbb{#1}}
\newcommand{\mc}[1]{\mathcal{#1}}
\newcommand{\mf}[1]{\mathfrak{#1}}
\newcommand{\ms}[1]{\mathscr{#1}}
\newcommand{\mbf}[1]{\mathbf{#1}}
\newcommand{\bbA}{\mathbb A}
\newcommand{\bbB}{\mathbb B}
\newcommand{\bbC}{\mathbb C}
\newcommand{\bbD}{\mathbb D}
\newcommand{\bbE}{\mathbb E}
\newcommand{\bbF}{\mathbb F}
\newcommand{\bbG}{\mathbb G}
\newcommand{\bbH}{\mathbb H}
\newcommand{\bbI}{\mathbb I}
\newcommand{\bbJ}{\mathbb J}
\newcommand{\bbK}{\mathbb K}
\newcommand{\bbL}{\mathbb L}
\newcommand{\bbM}{\mathbb M}
\newcommand{\bbN}{\mathbb N}
\newcommand{\bbO}{\mathbb O}
\newcommand{\bbP}{\mathbb P}
\newcommand{\bbQ}{\mathbb Q}
\newcommand{\bbR}{\mathbb R}
\newcommand{\bbS}{\mathbb S}
\newcommand{\bbT}{\mathbb T}
\newcommand{\bbU}{\mathbb U}
\newcommand{\bbV}{\mathbb V}
\newcommand{\bbW}{\mathbb W}
\newcommand{\bbX}{\mathbb X}
\newcommand{\bbY}{\mathbb Y}
\newcommand{\bbZ}{\mathbb Z}
\newcommand{\mcA}{\mc A}
\newcommand{\mcB}{\mc B}
\newcommand{\mcC}{\mc C}
\newcommand{\mcD}{\mc D}
\newcommand{\mcE}{\mc E}
\newcommand{\mcF}{\mc F}
\newcommand{\mcG}{\mc G}
\newcommand{\mcH}{\mc H}
\newcommand{\mcI}{\mc I}
\newcommand{\mcJ}{\mc J}
\newcommand{\mcK}{\mc K}
\newcommand{\mcL}{\mc L}
\newcommand{\mcM}{\mc M}
\newcommand{\mcN}{\mc N}
\newcommand{\mcO}{\mc O}
\newcommand{\mcP}{\mc P}
\newcommand{\mcQ}{\mc Q}
\newcommand{\mcR}{\mc R}
\newcommand{\mcS}{\mc S}
\newcommand{\mcT}{\mc T}
\newcommand{\mcU}{\mc U}
\newcommand{\mcV}{\mc V}
\newcommand{\mcW}{\mc W}
\newcommand{\mcX}{\mc X}
\newcommand{\mcY}{\mc Y}
\newcommand{\mcZ}{\mc Z}
\newcommand{\msA}{\ms A}
\newcommand{\msB}{\ms B}
\newcommand{\msC}{\ms C}
\newcommand{\msD}{\ms D}
\newcommand{\msE}{\ms E}
\newcommand{\msF}{\ms F}
\newcommand{\msG}{\ms G}
\newcommand{\msH}{\ms H}
\newcommand{\msI}{\ms I}
\newcommand{\msJ}{\ms J}
\newcommand{\msK}{\ms K}
\newcommand{\msL}{\ms L}
\newcommand{\msM}{\ms M}
\newcommand{\msN}{\ms N}
\newcommand{\msO}{\ms O}
\newcommand{\msP}{\ms P}
\newcommand{\msQ}{\ms Q}
\newcommand{\msR}{\ms R}
\newcommand{\msS}{\ms S}
\newcommand{\msT}{\ms T}
\newcommand{\msU}{\ms U}
\newcommand{\msV}{\ms V}
\newcommand{\msW}{\ms W}
\newcommand{\msX}{\ms X}
\newcommand{\msY}{\ms Y}
\newcommand{\msZ}{\ms Z}
\newcommand{\mfA}{\mf A}
\newcommand{\mfB}{\mf B}
\newcommand{\mfC}{\mf C}
\newcommand{\mfD}{\mf D}
\newcommand{\mfE}{\mf E}
\newcommand{\mfF}{\mf F}
\newcommand{\mfG}{\mf G}
\newcommand{\mfH}{\mf H}
\newcommand{\mfI}{\mf I}
\newcommand{\mfJ}{\mf J}
\newcommand{\mfK}{\mf K}
\newcommand{\mfL}{\mf L}
\newcommand{\mfM}{\mf M}
\newcommand{\mfN}{\mf N}
\newcommand{\mfO}{\mf O}
\newcommand{\mfP}{\mf P}
\newcommand{\mfQ}{\mf Q}
\newcommand{\mfR}{\mf R}
\newcommand{\mfS}{\mf S}
\newcommand{\mfT}{\mf T}
\newcommand{\mfU}{\mf U}
\newcommand{\mfV}{\mf V}
\newcommand{\mfW}{\mf W}
\newcommand{\mfX}{\mf X}
\newcommand{\mfY}{\mf Y}
\newcommand{\mfZ}{\mf Z}
\newcommand{\mfa}{\mf a}
\newcommand{\mfb}{\mf b}
\newcommand{\mfc}{\mf c}
\newcommand{\mfd}{\mf d}
\newcommand{\mfe}{\mf e}
\newcommand{\mff}{\mf f}
\newcommand{\mfg}{\mf g}
\newcommand{\mfh}{\mf h}
\newcommand{\mfi}{\mf i}
\newcommand{\mfj}{\mf j}
\newcommand{\mfk}{\mf k}
\newcommand{\mfl}{\mf l}
\newcommand{\mfm}{\mf m}
\newcommand{\mfn}{\mf n}
\newcommand{\mfo}{\mf o}
\newcommand{\mfp}{\mf p}
\newcommand{\mfq}{\mf q}
\newcommand{\mfr}{\mf r}
\newcommand{\mfs}{\mf s}
\newcommand{\mft}{\mf t}
\newcommand{\mfu}{\mf u}
\newcommand{\mfv}{\mf v}
\newcommand{\mfw}{\mf w}
\newcommand{\mfx}{\mf x}
\newcommand{\mfy}{\mf y}
\newcommand{\mfz}{\mf z}
\newcommand{\mbfA}{\mathbf A}
\newcommand{\mbfB}{\mathbf B}
\newcommand{\mbfC}{\mathbf C}
\newcommand{\mbfD}{\mathbf D}
\newcommand{\mbfE}{\mathbf E}
\newcommand{\mbfF}{\mathbf F}
\newcommand{\mbfG}{\mathbf G}
\newcommand{\mbfH}{\mathbf H}
\newcommand{\mbfI}{\mathbf I}
\newcommand{\mbfJ}{\mathbf J}
\newcommand{\mbfK}{\mathbf K}
\newcommand{\mbfL}{\mathbf L}
\newcommand{\mbfM}{\mathbf M}
\newcommand{\mbfN}{\mathbf N}
\newcommand{\mbfO}{\mathbf O}
\newcommand{\mbfP}{\mathbf P}
\newcommand{\mbfQ}{\mathbf Q}
\newcommand{\mbfR}{\mathbf R}
\newcommand{\mbfS}{\mathbf S}
\newcommand{\mbfT}{\mathbf T}
\newcommand{\mbfU}{\mathbf U}
\newcommand{\mbfV}{\mathbf V}
\newcommand{\mbfW}{\mathbf W}
\newcommand{\mbfX}{\mathbf X}
\newcommand{\mbfY}{\mathbf Y}
\newcommand{\mbfZ}{\mathbf Z}

% common math symbols

\newcommand{\bs}{\backslash}
\newcommand{\bbRn}{\mathbb R^n}
\newcommand{\bbCn}{\mathbb R^n}
\newcommand{\ecyc}[1]{\langle #1\rangle}
\newcommand{\interior}[1]{\overset{\circ}{#1}}
\newcommand{\ol}[1]{\overline{#1}}
\newcommand{\unitgrp}[1]{#1^{\mspace{-4mu}\times}}
\newcommand{\grad}{\triangledown}
\newcommand{\normal}{\trianglelefteq}
\newcommand{\nnormal}{\ntrianglelefteq}
\newcommand{\norm}[1]{\left\lVert#1\right\rVert}
\newcommand{\gen}[1]{\langle #1 \rangle}
\newcommand{\nilradical}[1]{\sqrt{\gen{0}}_{#1}}
\newcommand{\inv}[1]{#1^{\text{-}1}}
\newcommand{\id}{\mathrm{id}}
\newcommand{\unitcell}[1]{\overset{\!\!\!\! \circ}{\bbD^{#1}}}
\newcommand{\E}{\exists}
\newcommand{\A}{\forall}
\newcommand{\Fct}{\mathrm{Funct}}

% Operators

\DeclareMathOperator{\GL}{\rm{GL}}
\DeclareMathOperator{\SL}{\rm{SL}}
\DeclareMathOperator{\Mod}{\rm{Mod}}
\DeclareMathOperator{\Mor}{\rm{Mor}}
\DeclareMathOperator{\Obj}{\rm{Obj}}
\DeclareMathOperator{\Id}{\rm{Id}}
\DeclareMathOperator{\End}{\rm{End}}
\DeclareMathOperator{\Hom}{\rm{Hom}}
\DeclareMathOperator{\Res}{\rm{Res}}
\DeclareMathOperator{\Spec}{\rm{Spec}}
\DeclareMathOperator{\Proj}{\rm{Proj}}
\DeclareMathOperator{\Supp}{\rm{Supp}}
\DeclareMathOperator{\Ker}{\rm{Ker}}
\DeclareMathOperator{\Nil}{\rm{Nil}}
\DeclareMathOperator{\sh}{\rm{sh}}
\DeclareMathOperator{\Coker}{\rm{Coker}}
\DeclareMathOperator{\Rank}{\rm{Rank}}
\DeclareMathOperator{\Frac}{\rm{Frac}}
\DeclareMathOperator{\Rad}{\rm{Rad}}
\DeclareMathOperator{\Ann}{\rm{Ann}}
\DeclareMathOperator{\Disc}{\rm{Disc}}
\DeclareMathOperator{\Lcm}{\rm{lcm}}
\DeclareMathOperator{\Gcd}{\rm{gcd}}
\DeclareMathOperator{\Conv}{Conv}
\DeclareMathOperator{\Cone}{Cone}
\DeclareMathOperator{\Int}{Int}
\DeclareMathOperator{\Ord}{ord}
\DeclareMathOperator{\Ass}{Ass}
\DeclareMathOperator{\Aut}{Aut}
\DeclareMathOperator{\Sym}{Sym}
\DeclareMathOperator{\Char}{Char}
\DeclareMathOperator{\Span}{Span}
\DeclareMathOperator{\Tor}{Tor}
\DeclareMathOperator{\Ext}{Ext}
\DeclareMathOperator{\Card}{Card}
\DeclareMathOperator{\OPT}{OPT}
\DeclareMathOperator{\Dom}{Dom}
\DeclareMathOperator{\Var}{Var}
\DeclareMathOperator{\Th}{Th}
\DeclareMathOperator{\Frob}{Frob}
\DeclareMathOperator{\Red}{Red}
\DeclareMathOperator{\Aff}{Aff}
\DeclareMathOperator{\Epi}{Epi}
\DeclareMathOperator{\sech}{sech}
\DeclareMathOperator{\csch}{csch}
\DeclareMathOperator*{\Argmin}{Argmin}
\DeclareMathOperator{\Zer}{Zer}
\DeclareMathOperator{\Ri}{Ri}
\DeclareMathOperator{\Prox}{Prox}
\DeclareMathOperator{\sgn}{sgn}
\DeclareMathOperator{\Fix}{Fix}
\DeclareMathOperator{\Gal}{Gal}
\renewcommand{\Re}{\operatorname{Re}}
\renewcommand{\Im}{\operatorname{Im}}
\DeclareMathOperator{\Adj}{Adj}
\renewcommand{\div}{\mathrm{div}}
\DeclareMathOperator{\Div}{Div}
\DeclareMathOperator{\Cl}{Cl}
\DeclareMathOperator{\CDiv}{CDiv}
\DeclareMathOperator{\Bl}{Bl}
\DeclareMathOperator{\codim}{codim}
\DeclareMathOperator{\Sing}{Sing}
\DeclareMathOperator{\Nef}{Nef}
\DeclareMathOperator{\NE}{NE}
\DeclareMathOperator{\Mult}{Mult}
\DeclareMathOperator{\Pic}{Pic}
\DeclareMathOperator{\Tr}{Tr}
\DeclareMathOperator{\Grass}{Grass}
\DeclareMathOperator{\LinSub}{LinSub}
\DeclareMathOperator{\Eq}{Eq}
\DeclareMathOperator{\can}{can}
\DeclareMathOperator{\Rat}{Rat}
\DeclareMathOperator{\trdeg}{trdeg}
\DeclareMathOperator{\QSym}{QSym}
\DeclareMathOperator{\ASym}{ASym}
\DeclareMathOperator{\des}{des}
\DeclareMathOperator{\exc}{exc}
\DeclareMathOperator{\maj}{maj}
\DeclareMathOperator{\wt}{wt}
\DeclareMathOperator{\SST}{SST}
\DeclareMathOperator{\ASST}{\mfA SST}
\DeclareMathOperator{\SSYT}{SSYT}
\DeclareMathOperator{\ST}{ST}
\DeclareMathOperator{\SYT}{SYT}
\DeclareMathOperator{\SV}{SV}
\DeclareMathOperator{\ex}{ex}
\DeclareMathOperator{\sort}{sort}
\DeclareMathOperator{\sq}{sq}
\DeclareMathOperator{\invcode}{invcode}
\DeclareMathOperator{\Ess}{Ess}
\DeclareMathOperator{\Flag}{Flag}
\DeclareMathOperator{\Stab}{Stab}
\DeclareMathOperator{\Orb}{Orb}
\DeclareMathOperator{\pl}{pl}
\DeclareMathOperator{\Mat}{Mat}
\DeclareMathOperator{\pt}{pt}
\DeclareMathOperator{\depth}{depth}
\DeclareMathOperator{\cyc}{cyc}
\DeclareMathOperator{\ev}{ev}
\DeclareMathOperator{\length}{lg}
\DeclareMathOperator{\Quot}{Quot}
\DeclareMathOperator{\Hilb}{Hilb}
\DeclareMathOperator{\PGL}{PGL}
\DeclareMathOperator{\PSL}{PSL}
\newcommand{\Cat}[1]{(\textrm{#1})}

\begin{document}
    \title{Manifolds}
    \author{{\color{pink}{Cloudi}}{\color{Aquamarine}{fold}}}
    \maketitle
    \newpage

    \setcounter{section}{-1}

    \section{Manifolds and Maps}

    \begin{defn}
        A map $f : U \to \bbR^m$ (where $U$ is an open subset of $\bbR^n$)
        is called $C^k$ if
        the $k$-th derivative of $f$ exist and continuous.\\
        $f$ is called $C^{\infty}$ or \tbf{smooth} if $f$ is $C^k$ for any $k \in \bbN$.\\
        $f : U \to V$ is called $C^{\omega}$ or \tbf{analytic} if
        $f$ is $C^\infty$ and
        forall $ x \in U$, there exists open neighborhood $U_x$
        such that the Taylor series expansion of $f$ at $x$
        pointwise converges to $f$ on $U_x$
    \end{defn}

    \begin{defn}
        For a $C^r$ ($r \geq 1$) map $f : U \to R^m$ (where $U$ is an open subset of $\bbR^n$),
        $y \in f(U)$ is called a \tbf{regular value} of $f$
        if for any $x \in f^{-1}(y)$,
        the rank of the Jacobian matrix of $f$ at $x$
        is $m$.
    \end{defn}

    \begin{note}
        By \tbf{Weierstrass M-test},
        the ``pointwise converges'' condition in the definition of $C^\omega$ function
        can be replaced equivalently by ``uniform converges''.
    \end{note}

    \begin{defn}
        An \tbf{$n$-dimensional $C^r$ manifold} (we assume $r \geq 1$)
        $(M,A)$ is datas:
        \begin{itemize}
            \item A $T_2$ space $M$ whose topology have a countable basis
            \item $A = \{(U_j,\varphi_j)\}_{j \in J}$,
            which consists of open cover $\{U_j\}_{j \in J}$ of $M$
            and homeomorphisms $\{\varphi_j : U_j \to O_j \}_{j \in J}$.
            (where $O_j$ is an open subset of $\bbR^n$)\\
        \end{itemize}
        with condition:
        \begin{itemize}
            \item Forall $i,j \in J$,
            the homeomorphism
            $\varphi_j \circ \varphi_i^{-1} : \varphi_i(U_i \cap U_j) \to \varphi_j(U_i \cap U_j)$
            is $C^r$. (these maps are called \tbf{transition maps})\\
        \end{itemize}
        A single $(U_j,\varphi_j)$ is called a \tbf{chart},
        $A$ is called ($C^r$) \tbf{atlas} or \tbf{differential structure}.
    \end{defn}

    \begin{note}
        Manifolds defined above are locally compact,
        hence paracompact and  compactly generated Hausdorff.
    \end{note}

    \begin{defn}
        Suppose $M,N$ are $C^r$ manifolds with dimensions $n,k$
        and atlas $A = \{(U_j,\varphi_j)\}_{j \in J}, \ B = \{(V_i,\phi_i)\}_{i \in I}$.
        The \tbf{product of two manifolds}
        is a $(n+k)$-dimensional manifold
        with underlying space $M \times N$
        and atlas $A \times B := \{(U_j \times V_i,\varphi_j \times \phi_i)\}_{(j,i) \in J \times I}$.
        
    \end{defn}

    \begin{defn}
        Suppose $N$ is $C^r$ manifolds with dimension $n$
        and atlas $A = \{(U_j,\varphi_j)\}_{j \in J}$,
        $M$ is another manifold.
        $h : M \hookrightarrow N$ is topological embedding.
        The \tbf{induced atlas} on $M$
        is atlas $h^* A := \{(h^{-1}(U_j),\varphi_j \circ h)\}_{j \in J}$.
        
    \end{defn}

    \begin{defn}
        Suppose $N$ is $C^r$ manifolds with dimension $n$
        and atlas $A = \{(U_j,\varphi_j)\}_{j \in J}$,
        $M$ is subspace of $N$.
        $M$ is a $C^r$ \tbf{submanifold of dimension $k$} of $(N,A)$
        if
        for any $x \in M$
        there exists a chart $(U_j,\varphi_j)$
        such that $x \in U_j$ and 
        \[
            \varphi_j (M \cap U_j) = \phi_j(U_j) \cap \bbR^k
        \]
        Such charts combine together give a atlas of $M$.\\
        We also say that $M$ is a submanifold of \tbf{codimension} $n-k$.

    \end{defn}

    \begin{note}
        Open subset $U$ of a $n$-dimensional manifold $M$
        is obviously a submanifold of dimension $n$.
    \end{note}

    \begin{exmp}
        The \tbf{real general linear group}
        $\GL_n(\bbR)$
        is an open analytic submanifold
        of $(n \times n)$-dimensional manifold $M_n(\bbR)$
        since $ \GL_n(\bbR) = \det^{-1}(\bbR - \{0\}) $
        and
        the manifold structure (and topology) of $\Mat_{n \times n}(\bbR)$
        is given by bijection in ($\mbf{Set}$)
        $ \Mat_{n\times n}(\bbR) \leftrightarrows \bbR^{n \times n} $.
    \end{exmp}

    \begin{exmp}
        The (non-compact) \tbf{real Stiefel manifold}
        $\St_k(\bbR^n)$
        is the set of $\bbR$-linear monomorphisms $\bbR^k \to \bbR^n$,
        which is equivalent (in $\mbf{Set}$) to
        the set of linearly independent elements
        \[
            (a_1, \dots , a_k) \in \bbR^n \times \cdots \times \bbR^n = \bbR^{n \times k}
        \]
        (also called set of $k$-frames in $\bbR^n$)\\
        $\St_k(\bbR^n)$ is an open analytic submanifold of
        $\bbR^{n \times k}$,
        since $\St_k(\bbR^n) = \bigcap\limits_{1 \leq r \leq n-k+1} \det_{r}^{-1}(\bbR - \{0\})$.
        Where $\det_r$ is defined by:
        \[
        \textstyle{\det_r}
        \left( {\begin{bmatrix}
            a_{1,1} & \cdots & a_{1,k} \\
            a_{2,1} & \cdots & a_{2,k} \\
            \vdots &   \ & \vdots \\
            a_{n-1,1} & \cdots & a_{n-1,k} \\
            a_{n,1} & \cdots & a_{n,k} \\
        \end{bmatrix} } \right)
        := \det \left({\begin{bmatrix}
            a_{r,1} & \cdots & a_{r,k} \\
            a_{r+1,1} & \cdots & a_{r+1,k} \\
            \vdots &   \ & \vdots \\
            a_{r+k-1,1} & \cdots & a_{r+k-1,k} \\
        \end{bmatrix} } \right)
        \]
        
    \end{exmp}

    \begin{exmp}
        The \tbf{Grassmann manifold}
        $\Gr_k(\bbR^n) = \St_k (\bbR^n)/\GL_k(\bbR)$
        is the set $k$-dimensional subspaces of $\bbR^n$.
        Define $U_V := \{W \in \Gr_k(\bbR^n) \mid \rank(p_V|_{W} : W \to V) = k  \}$,
        where
        $p_V : \bbR^n \to V$
        is the orthogonal projection.
        Let $i_W : V \to W$ be inverse of $p_V|_W$,
        define bijection:
        \begin{align*}
            U_V & \xrightarrow{\varphi_V} \Hom_{\bbR}(V,V^\bot) \approx \bbR^{k \times (n-k)}\\
            W & \longmapsto p_{V^\bot} \circ i_{W}\\
            \Im(\id_V \times M) & \longmapsfrom M
        \end{align*}
        Where $p_{V^\bot} : \bbR^n = V \oplus V^{\bot } \to V^{\bot}$.\\
        Atlas  on $\Gr_k(\bbR^n)$ is given by $\{(U_V, \varphi_V)\}_{V \in \Gr_k(\bbR^n)}$.\\
        We verify that the transition maps are analytic.
        Suppose $E$ is spaned by $\mbfe_1, \cdots ,\mbfe_k$,
        $V$ have orthonormal basis $[\mbfv_1, \cdots , \mbfv_k]$
        and $V^\bot$ have orthonormal basis $[\mbfv_{k+1}, \cdots , \mbfv_{n}]$.
        Let $\GL_n(\bbR) \ni T : E \oplus E^\bot \to V \oplus V^\bot$
        be the transformation sends each $\mbfe_i$ to $\mbfv_i$.
        Let $[T]$ be its matrix respect to $[\mbfe_1, \mbfe_2, \cdots, \mbfe_n]$,
        and $[T^{-1}]$ is matrix of its inverse.
        then $p_{V^\bot}$
        is given by
        \[
        a = 
        [\mbfe_1, \mbfe_2, \cdots, \mbfe_n]
        \begin{bmatrix}
            a_1 \\
            a_2 \\
            \vdots \\
            a_n
        \end{bmatrix}
        \mapsto
        [\mbfv_{k+1}, \mbfv_{k+2}, \cdots, \mbfv_n]
        \begin{bmatrix}
            T^{-1}
        \end{bmatrix}_{i > k}
        \begin{bmatrix}
            a_1 \\
            a_2 \\
            \vdots \\
            a_n
        \end{bmatrix}
        \]
        Suppose $M \in \Hom_{\bbR}(E,E^\bot)$
        satisfy $W := \Im(\id_E \times M) \in U_V$,
        (that is $M \in \varphi_E^{-1} (U_E \cap U_V)$)
        let $[m_{i,j}]_{k\leq i \leq n, 1 \leq j \leq k}$ be its matrix respect to $[\mbfe_1, \mbfe_2, \cdots, \mbfe_n]$.
        Then $p_{V}|_{W}$ is given by:
        \[
        a = 
        [\mbfe_1, \mbfe_2, \cdots, \mbfe_n]
        \begin{bmatrix}
            I_k \\
            M
        \end{bmatrix}
        \begin{bmatrix}
            a_1 \\
            a_2 \\
            \vdots \\
            a_k
        \end{bmatrix}
        \longmapsto
        [\mbfv_1, \mbfv_2, \cdots, \mbfv_k]
        \begin{bmatrix}
            T^{-1}
        \end{bmatrix}_{i \leq k}
        \begin{bmatrix}
            I_k \\
            M
        \end{bmatrix}
        \begin{bmatrix}
            a_1 \\
            a_2 \\
            \vdots \\
            a_k
        \end{bmatrix}
        \]
        And $i_W$ is given by:
        \[
        b = 
        [\mbfv_1, \mbfv_2, \cdots, \mbfv_k]
        \begin{bmatrix}
            b_1 \\
            b_2 \\
            \vdots \\
            b_k
        \end{bmatrix}
        \longmapsto
        [\mbfe_1, \mbfe_2, \cdots, \mbfe_n]
        \begin{bmatrix}
            I_k \\
            M
        \end{bmatrix}
        \left(\begin{bmatrix}
            T^{-1}
        \end{bmatrix}_{i \leq k}
        \begin{bmatrix}
            I_k \\
            M
        \end{bmatrix}\right)^{-1}
        \begin{bmatrix}
            b_1 \\
            b_2 \\
            \vdots \\
            b_k
        \end{bmatrix}
        \]
        Finally, we see that $p_{V^\bot} \circ i_W$ is:
        \[
        b = 
        [\mbfv_1, \mbfv_2, \cdots, \mbfv_k]
        \begin{bmatrix}
            b_1 \\
            b_2 \\
            \vdots \\
            b_k
        \end{bmatrix}
        \longmapsto
        [\mbfv_{k+1}, \mbfv_{k+2}, \cdots, \mbfv_n]
        \begin{bmatrix}
            T^{-1}
        \end{bmatrix}_{i > k}
        \begin{bmatrix}
            I_k \\
            M
        \end{bmatrix}
        \left(\begin{bmatrix}
            T^{-1}
        \end{bmatrix}_{i \leq k}
        \begin{bmatrix}
            I_k \\
            M
        \end{bmatrix}\right)^{-1}
        \begin{bmatrix}
            b_1 \\
            b_2 \\
            \vdots \\
            b_k
        \end{bmatrix}
        \]
        and it is an analytic function on $M$.

        
    \end{exmp}

    \begin{defn}
        Let $(M,A)$, $(N,B)$ be $C^r$-manifolds,
        a continuous map $f : M \to N$
        is said to be $C^k$ ($k \leq r$)
        if for any charts $(U,\varphi_U) \in A,\ (V, \phi_V) \in B$,
        map
        \[
            \phi_V \circ f \circ \varphi_U^{-1} : \varphi_U (U \cap f^{-1}(V)) \to \phi_V(V)
        \]
        is $C^k$.\\
        A \tbf{$C^r$-diffeomorphism}
        between two $C^r$-manifolds $M,N$
        is a $C^r$-map $f : M \to N$
        with its $C^r$-inverse $f^{-1} : N \to M$.\\
        If $C^r$-diffeomorphism between $C^r$-manifolds $M,N$
        exists, then $M,N$
        are said to be \tbf{$C^r$-diffeomorphic}.\\
        Set of $C^k$ maps from $M$ to $N$ is noted $\Hom_{C^r}(M,N)$.
    \end{defn}

    \begin{note}
        A $C^r$ map which is a $C^1$-diffeomorphism is a $C^r$-diffeomorphism.
    \end{note}

    \begin{exmp}
        $(-)^{\bot} : \Gr_{k}(\bbR^n) \to \Gr_{n-k}(\bbR^n)$
        is a $C^{\omega}$-diffeomorphism.
        \[
        \varphi_{V^\bot} \circ (-)^\bot \circ \varphi_V^{-1} :  \Mat_{k\times (n-k)}(\bbR) \ni M \mapsto M^T \in \Mat_{(n-k) \times k}(\bbR)
        \]
    \end{exmp}

    \begin{defn}
        Let $(M,A)$ be a $C^{r}$-manifold, $x \in M$
        the \tbf{tangency relation}
        defined on $\Hom_{C^1}(\bbR|_{\sim 0},M)_x := \{ \gamma \in \Hom_{C^1} (O, M) \mid 0 \in O \in \tau_\bbR ,\ \gamma(0) = x\}$
        ($\tau_X$ is set of all open sets of $X$)
        is equivalence relation which have two equivalent definitions:
        \begin{enumerate}
            \item $\gamma_1(t) \sim_T \gamma_2(t) $ if there exists chart $(U,\varphi_U) \in A$
            such that $x \in U$ and
            \[
            \dfrac{\d (\varphi_U \circ \gamma_1)}{\d t}|_{t = 0} = \dfrac{\d (\varphi_U \circ \gamma_2)}{\d t}|_{t = 0}
            \]
            \item $\gamma_1(t) \sim_T \gamma_2(t) $ if for any chart $(U,\varphi_U) \in A$
            such that $x \in U$, we have
            \[
            \dfrac{\d (\varphi_U \circ \gamma_1)}{\d t}|_{t = 0} = \dfrac{\d (\varphi_U \circ \gamma_2)}{\d t}|_{t = 0}
            \]
        \end{enumerate}
    \end{defn}

    \begin{note}
        $2. \implies 1.$ is obvious.\\
        $1. \implies 2.$ because transition maps are $C^r$-diffeomorphisms,
        whose derivatives are linear isomorphisms.
    \end{note}

    \begin{defn}
        Let $(M,A)$ be a $n$-dimensional $C^r$-manifold, $x \in M$
        the \tbf{space of tangent vectors} on $M$ at $x$
        is defined by:
        $T_x (M) := \Hom_{C^1}(\bbR|_{\sim 0}, M)_x / \sim_T$.
    \end{defn}

    \begin{note}
        If $(U,\varphi_U)$ is a chart with $x \in U$,
        then $\varphi_U$ defines a bijection
        $\d( \varphi_U \circ - ) : T_x(M) \to \bbR^n$
        by:
        % https://q.uiver.app/?q=WzAsMTMsWzEsMSwiW1xcZ2FtbWFfe1xcbWJmdn1dX3tcXHNpbV9UfSJdLFswLDAsIlxcYmJSXm4iXSxbMSwwLCJUX3ggKE0pIl0sWzAsMSwiXFxtYmZ2Il0sWzMsMiwiXFx2YXJwaGlfVShVKSJdLFs0LDIsIlUiXSxbMiwyLCJcXGdhbW1hX3tcXG1iZnZ9IDpmX3tcXG1iZnZ9XnstMX0gKFxcdmFycGhpX1UoVSkpICJdLFsyLDEsIiB0Il0sWzMsMSwiXFx2YXJwaGlfVSAoeCkgKyB0XFxtYmZ2Il0sWzIsMCwiZl97XFxtYmZ2fSA6IFxcYmJSIl0sWzMsMCwiXFxiYlJebiJdLFsxLDIsIltcXGdhbW1hXV97XFxzaW1fVH0iXSxbMCwyLCJcXGRmcmFje1xcZCAoXFx2YXJwaGlfVSBcXGNpcmNcXGdhbW1hKX17XFxkIHR9Il0sWzEsMl0sWzEsMywiXFxuaSIsMyx7InN0eWxlIjp7ImJvZHkiOnsibmFtZSI6Im5vbmUifSwiaGVhZCI6eyJuYW1lIjoibm9uZSJ9fX1dLFszLDAsIiIsMyx7InN0eWxlIjp7InRhaWwiOnsibmFtZSI6Im1hcHMgdG8ifX19XSxbNCw1LCJcXHZhcnBoaV9VXnstMX0iXSxbMiwwLCJcXG5pIiwzLHsic3R5bGUiOnsiYm9keSI6eyJuYW1lIjoibm9uZSJ9LCJoZWFkIjp7Im5hbWUiOiJub25lIn19fV0sWzYsNCwiZl97XFxtYmZ2fSJdLFs3LDgsIiIsMyx7InN0eWxlIjp7InRhaWwiOnsibmFtZSI6Im1hcHMgdG8ifX19XSxbOSwxMF0sWzExLDEyLCIiLDAseyJzdHlsZSI6eyJ0YWlsIjp7Im5hbWUiOiJtYXBzIHRvIn19fV0sWzksNywiXFxuaSIsMyx7InN0eWxlIjp7ImJvZHkiOnsibmFtZSI6Im5vbmUifSwiaGVhZCI6eyJuYW1lIjoibm9uZSJ9fX1dLFsxMCw4LCJcXG5pIiwzLHsic3R5bGUiOnsiYm9keSI6eyJuYW1lIjoibm9uZSJ9LCJoZWFkIjp7Im5hbWUiOiJub25lIn19fV1d
        \[\begin{tikzcd}[row sep = 0.5cm, column sep = 0.5cm]
            {\bbR^n} & {T_x (M)} & {f_{\mbfv} : \bbR} & {\bbR^n} \\
            \mbfv & {[\gamma_{\mbfv}]_{\sim_T}} & { t} & {\varphi_U (x) + t\mbfv} \\
            {\dfrac{\d (\varphi_U \circ\gamma)}{\d t}} & {[\gamma]_{\sim_T}} & {\gamma_{\mbfv} :f_{\mbfv}^{-1} (\varphi_U(U)) } & {\varphi_U(U)} & U
            \arrow[from=1-1, to=1-2]
            \arrow["\ni"{marking}, draw=none, from=1-1, to=2-1]
            \arrow[maps to, from=2-1, to=2-2]
            \arrow["{\varphi_U^{-1}}", from=3-4, to=3-5]
            \arrow["\ni"{marking}, draw=none, from=1-2, to=2-2]
            \arrow["{f_{\mbfv}}", from=3-3, to=3-4]
            \arrow[maps to, from=2-3, to=2-4]
            \arrow[from=1-3, to=1-4]
            \arrow[maps to, from=3-2, to=3-1]
            \arrow["\ni"{marking}, draw=none, from=1-3, to=2-3]
            \arrow["\ni"{marking}, draw=none, from=1-4, to=2-4]
        \end{tikzcd}\]
        If $(V,\varphi_V)$ is another chart with $x \in V$,
        then
        we have commutative diagram:
        % https://q.uiver.app/?q=WzAsMyxbMCwwLCJcXGJiUl5uIl0sWzIsMCwiXFxiYlJebiJdLFsxLDEsIlRfeChNKSJdLFswLDEsIkpfe1xcdmFycGhpX1Veey0xfSh4KX0gKFxcdmFycGhpX1ZcXGNpcmMgXFx2YXJwaGlfVV57LTF9KSJdLFswLDIsIlQgXFx2YXJwaGlfVV57LTF9IiwyXSxbMSwyLCJUIFxcdmFycGhpX1Veey0xfSJdXQ==
        \[\begin{tikzcd}[row sep = 1cm, column sep = 1cm]
            {\bbR^n} && {\bbR^n} \\
            & {T_x(M)}
            \arrow["{\d (\varphi_V\circ \varphi_U^{-1}) |_y}", from=1-1, to=1-3]
            \arrow["{T \varphi_U^{-1}}"', from=1-1, to=2-2]
            \arrow["{T \varphi_U^{-1}}", from=1-3, to=2-2]
        \end{tikzcd}\]
        where $y = \varphi_U^{-1}(x)$.
    \end{note}

    \begin{defn}
        Let $(M,A)$ be an $n$-dimensional $C^{r+1}$-manifold,
        the \tbf{space of all tangent vectors} of $M$
        is a $2n$-dimensional $C^r$-manifold defined by
        $T(M) := \bigsqcup\limits_{x \in M} T_x(M) $
        with atlas $\{(T(U),\phi_U)\}_{U \in A}$
        (called \tbf{natural atlas})
        defined by
        % https://q.uiver.app/?q=WzAsNCxbMCwwLCJcXGJpZ3NxY3VwXFxsaW1pdHNfe3ggXFxpbiBVfSBUX3goTSkiXSxbMSwwLCJcXHZhcnBoaV9VKFUpXFx0aW1lcyBcXGJiUl5uIl0sWzAsMSwiKHgsW1xcZ2FtbWFdX3tcXHNpbV9UfSkiXSxbMSwxLCIoXFx2YXJwaGlfVSh4KSxcXGQgKFxcdmFycGhpX1UgXFxjaXJjIFxcZ2FtbWEpKSJdLFswLDEsIlxccGhpX1UiXSxbMiwwLCJcXGluIiwzLHsic3R5bGUiOnsiYm9keSI6eyJuYW1lIjoibm9uZSJ9LCJoZWFkIjp7Im5hbWUiOiJub25lIn19fV0sWzIsMywiIiwzLHsic3R5bGUiOnsidGFpbCI6eyJuYW1lIjoibWFwcyB0byJ9fX1dLFszLDEsIlxcaW4iLDMseyJzdHlsZSI6eyJib2R5Ijp7Im5hbWUiOiJub25lIn0sImhlYWQiOnsibmFtZSI6Im5vbmUifX19XV0=
        \[\begin{tikzcd}[row sep = 0.6cm]
            {\bigsqcup\limits_{x \in U} T_x(M)} & {\varphi_U(U)\times \bbR^n} \\
            {(x,[\gamma]_{\sim_T})} & {(\varphi_U(x),\d (\varphi_U \circ \gamma))}
            \arrow["{\phi_U}", from=1-1, to=1-2]
            \arrow["\in"{marking}, draw=none, from=2-1, to=1-1]
            \arrow[maps to, from=2-1, to=2-2]
            \arrow["\in"{marking}, draw=none, from=2-2, to=1-2]
        \end{tikzcd}\]
        The \tbf{tangent bundle} on $M$
        is the bundle
        \begin{align*}
            p_M : T(M)  & \to M\\
            (x, [\gamma]_{\sim_T}) & \mapsto x
        \end{align*}
    \end{defn}

    \begin{defn}
        Let $(M,A), (N,B)$ be two $C^{r+1}$-manifolds,
        and $f : M \to N$
        is an $C^{r+1}$ map.
        Define an $C^r$ map between bundles $Tf : TM \to TN$
        by
        % https://q.uiver.app/?q=WzAsNixbMCwxLCIoeCxbXFxnYW1tYV1fe1xcc2ltX1R9KSJdLFsxLDEsIihmKHgpLFtmIFxcY2lyYyBcXGdhbW1hXV97XFxzaW1fVH0pIl0sWzAsMiwiKHgsIFxcZCAoXFx2YXJwaGlfVSBcXGNpcmMgXFxnYW1tYSkiXSxbMSwyLCIoZih4KSwgXFxkKFxccGhpX1YgXFxjaXJjIGYgXFxjaXJjIFxcdmFycGhpX1Veey0xfSkgXFxjaXJjIFxcZCggXFx2YXJwaGlfVSBcXGNpcmMgXFxnYW1tYSkpIl0sWzAsMCwiVGYgOiBUTSJdLFsxLDAsIlROIl0sWzAsMSwiIiwzLHsic3R5bGUiOnsidGFpbCI6eyJuYW1lIjoibWFwcyB0byJ9fX1dLFswLDIsIiIsMyx7Im9mZnNldCI6Miwic3R5bGUiOnsidGFpbCI6eyJuYW1lIjoibWFwcyB0byJ9fX1dLFsyLDAsIiIsMyx7Im9mZnNldCI6Miwic3R5bGUiOnsidGFpbCI6eyJuYW1lIjoibWFwcyB0byJ9fX1dLFsyLDMsIiIsMyx7InN0eWxlIjp7InRhaWwiOnsibmFtZSI6Im1hcHMgdG8ifX19XSxbMywxLCIiLDMseyJvZmZzZXQiOjIsInN0eWxlIjp7InRhaWwiOnsibmFtZSI6Im1hcHMgdG8ifX19XSxbMSwzLCIiLDMseyJvZmZzZXQiOjIsInN0eWxlIjp7InRhaWwiOnsibmFtZSI6Im1hcHMgdG8ifX19XSxbNCw1XSxbNCwwLCJcXG5pIiwzLHsic3R5bGUiOnsiYm9keSI6eyJuYW1lIjoibm9uZSJ9LCJoZWFkIjp7Im5hbWUiOiJub25lIn19fV0sWzUsMSwiXFxuaSIsMyx7InN0eWxlIjp7ImJvZHkiOnsibmFtZSI6Im5vbmUifSwiaGVhZCI6eyJuYW1lIjoibm9uZSJ9fX1dXQ==
        \[\begin{tikzcd}[row sep = 0.3cm]
            {Tf : TM} & TN \\
            {(x,[\gamma]_{\sim_T})} & {(f(x),[f \circ \gamma]_{\sim_T})} \\
            {(\varphi_U(x), \d (\varphi_U \circ \gamma))} & {(\phi_V(f(x)), \d(\phi_V \circ f \circ \varphi_U^{-1}) \circ \d( \varphi_U \circ \gamma))}
            \arrow[maps to, from=2-1, to=2-2]
            \arrow[shift right=2, maps to, from=2-1, to=3-1]
            \arrow[shift right=2, maps to, from=3-1, to=2-1]
            \arrow[maps to, from=3-1, to=3-2]
            \arrow[shift right=2, maps to, from=3-2, to=2-2]
            \arrow[shift right=2, maps to, from=2-2, to=3-2]
            \arrow[from=1-1, to=1-2]
            \arrow["\ni"{marking}, draw=none, from=1-1, to=2-1]
            \arrow["\ni"{marking}, draw=none, from=1-2, to=2-2]
        \end{tikzcd}\]
    \end{defn}

    \begin{note}
        In fact,
        $T$ is a functor from category of $C^{r+1}$ manifolds
        to $C^{r}$ manifolds.
        And there is natural diffeomorphism $T(M \times N) \approx_{\d} TM \times TN$.
    \end{note}

    \begin{exmp}
        Use atlas of $S^n := \{(x_0, \cdots , x_n) \in \bbR^{n+1} \mid \sqrt{\sum_i x_i^2} = 1 \}$
        given by
        stereographic projections:
        \begin{align*}
            U_0 := S^n - (1,0,\cdots , 0) & \xrightarrow{\varphi_0} \bbR^n \\
                (x_0, x_1 ,\cdots , x_n) & \longmapsto (\dfrac{x_1}{1-x_0}, \cdots , \dfrac{x_n}{1-x_0})\\
                (\dfrac{s^2 -1}{s^2+1}, \dfrac{2 x'_1}{s^2 + 1} , \cdots , \dfrac{2 x'_n}{s^2 + 1}) & \longmapsfrom (x'_1, \cdots , x'_n)\\
                \text{where } s^2 := \sum_{1 \leq i \leq n} {x'_i}^2 &  = \dfrac{1+x_0}{1-x_0}\\
            U_\infty := S^n - (-1,0,\cdots , 0) & \xrightarrow{\varphi_\infty} \bbR^n \\
                (x_0, x_1 ,\cdots , x_n) & \longmapsto (\dfrac{x_1}{1+x_0}, \cdots , \dfrac{x_n}{1+x_0})\\
                (\dfrac{1 - s'^2}{s'^2+1}, \dfrac{2 x'_1}{s'^2 + 1} , \cdots , \dfrac{2 x'_n}{s'^2 + 1}) & \longmapsfrom (x'_1, \cdots , x'_n)\\
                \text{where } s'^2 := \sum_{1 \leq i \leq n} {x'_i}^2  & = \dfrac{1-x_0}{1+x_0}
        \end{align*}
        There is a diffeomorphism $TS^{n} \times \bbR \approx_{\d} S^n \times \bbR^{n+1}$
        defined as:
        \begin{align*}
            TU_\lambda \times \bbR \approx \varphi_\lambda (U_\lambda) \times \bbR^{n} \times \bbR & \to U_\lambda \times \bbR^{n+1}\\
            (\mbfx',\mbfv,a)  & \mapsto (\varphi_\lambda^{-1}(\mbfx'), \d(\varphi_\lambda^{-1}) \mbfv + a \cdot \mbfe_{K} )\\
            (\varphi_\lambda(\mbfx), d(\varphi_\lambda) \mbfv' , p_{K} \mbfv' )& \mapsfrom (\mbfx, \mbfv')
        \end{align*}
        where $\lambda = 0 \text{ or } \infty$,
        $K = \ker \d(\varphi_0) = \ker \d(\varphi_\infty)$ is a $1$-dimensional linear subspace, $\mbfe_K$ is the unit vector in it,
        and $p_K$ is the orthogonal projection onto it.\\
        It is easy to check it is compatible with transition maps.
        
    \end{exmp}

    \begin{note}
        In fact, embedding $i : M \hookrightarrow \bbR^k$
        produces injective linear maps $T_x(M) \to T_{i(x)}\bbR^k$
    \end{note}

\end{document}