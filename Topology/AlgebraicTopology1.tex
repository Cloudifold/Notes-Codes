\documentclass[a4paper]{article}
\usepackage[dvipsnames, svgnames, x11names]{xcolor}
\usepackage{tikz-cd}
\tikzcdset{row sep/normal=1cm}
\usepackage{amsfonts}
\usepackage{hyperref}
\usepackage{amsthm}
\usepackage{enumerate}
\usepackage[left=3cm, right=3cm, top=2cm]{geometry}
\usepackage{sectsty}
\usepackage[perpage]{footmisc}
\usepackage{amsthm}
\usepackage{tikz-cd}
\usepackage{pgfplots}
\usepackage{amsmath}
\usepackage{amssymb}
\usepackage{pdfpages}
\usepackage{multirow}
\usepackage{mathtools}
\usepackage{mathrsfs}
\usepackage{ebproof}
\usepackage{graphicx}
\usepackage{epigraph}
\usepackage{enumitem}
\usepackage{indentfirst}


\theoremstyle{plain}
\newtheorem{thm}{Theorem}[section]
\newtheorem{lem}[thm]{Lemma}
\newtheorem{prop}[thm]{Proposition}
\newtheorem*{cor}{Corollary}

\theoremstyle{definition}
\newtheorem{defn}{Definition}[section]
\newtheorem{conj}{Conjecture}[section]
\newtheorem{exmp}{Example}[section]

\newtheorem*{prf}{Proof}

\theoremstyle{remark}
\newtheorem*{rem}{Remark}
\newtheorem*{note}{Note}
\newtheorem{ex}{Exercise}[section]
\newtheorem{sol}{Solution}[section]

\definecolor{background}{RGB}{255,230,230}
\pagecolor{background}

\definecolor{TextColor}{RGB}{255,122,162}

\chapterfont{\color{TextColor}}
\sectionfont{\color{TextColor}\Huge}
\setlist{nolistsep,topsep=-7pt}
\pgfplotsset{width=10cm,compat=newest}
%\usepgfplotslibrary{external}
%\tikzexternalize[prefix=tikz/]
\newcommand{\ssubparagraph}{ \parshape 1 1cm \dimexpr\linewidth-2cm\relax}
%\numberwithin{equation}{section}

% \quad : width of  1 'M'
% \qquad : 4
% \0x20 (\ ) : 1/3
% \; : 2/7
% \, : 1/6
% \! : -1/6

% \addtocounter{<envname>}{nubmer}
% Expl. \addtocounter{thm}{2}
% \setcounter{<envname>}{nubmer}
% Expl. \setcounter{ex}{3}
% -->  Exercise 1.4

\newcommand{\setexcounter}[1]{
    \setcounter{ex}{#1}
    \setcounter{sol}{#1}
}
\newcommand{\setsubsectcounter}[1]{\setcounter{subsection}{#1}}

\newcommand{\bb}[1]{\mathbb{#1}}
\newcommand{\mc}[1]{\mathcal{#1}}
\newcommand{\mf}[1]{\mathfrak{#1}}
\newcommand{\ms}[1]{\mathscr{#1}}
\newcommand{\mbf}[1]{\mathbf{#1}}
\newcommand{\bbA}{\mathbb A}
\newcommand{\bbB}{\mathbb B}
\newcommand{\bbC}{\mathbb C}
\newcommand{\bbD}{\mathbb D}
\newcommand{\bbE}{\mathbb E}
\newcommand{\bbF}{\mathbb F}
\newcommand{\bbG}{\mathbb G}
\newcommand{\bbH}{\mathbb H}
\newcommand{\bbI}{\mathbb I}
\newcommand{\bbJ}{\mathbb J}
\newcommand{\bbK}{\mathbb K}
\newcommand{\bbL}{\mathbb L}
\newcommand{\bbM}{\mathbb M}
\newcommand{\bbN}{\mathbb N}
\newcommand{\bbO}{\mathbb O}
\newcommand{\bbP}{\mathbb P}
\newcommand{\bbQ}{\mathbb Q}
\newcommand{\bbR}{\mathbb R}
\newcommand{\bbS}{\mathbb S}
\newcommand{\bbT}{\mathbb T}
\newcommand{\bbU}{\mathbb U}
\newcommand{\bbV}{\mathbb V}
\newcommand{\bbW}{\mathbb W}
\newcommand{\bbX}{\mathbb X}
\newcommand{\bbY}{\mathbb Y}
\newcommand{\bbZ}{\mathbb Z}
\newcommand{\mcA}{\mc A}
\newcommand{\mcB}{\mc B}
\newcommand{\mcC}{\mc C}
\newcommand{\mcD}{\mc D}
\newcommand{\mcE}{\mc E}
\newcommand{\mcF}{\mc F}
\newcommand{\mcG}{\mc G}
\newcommand{\mcH}{\mc H}
\newcommand{\mcI}{\mc I}
\newcommand{\mcJ}{\mc J}
\newcommand{\mcK}{\mc K}
\newcommand{\mcL}{\mc L}
\newcommand{\mcM}{\mc M}
\newcommand{\mcN}{\mc N}
\newcommand{\mcO}{\mc O}
\newcommand{\mcP}{\mc P}
\newcommand{\mcQ}{\mc Q}
\newcommand{\mcR}{\mc R}
\newcommand{\mcS}{\mc S}
\newcommand{\mcT}{\mc T}
\newcommand{\mcU}{\mc U}
\newcommand{\mcV}{\mc V}
\newcommand{\mcW}{\mc W}
\newcommand{\mcX}{\mc X}
\newcommand{\mcY}{\mc Y}
\newcommand{\mcZ}{\mc Z}
\newcommand{\msA}{\ms A}
\newcommand{\msB}{\ms B}
\newcommand{\msC}{\ms C}
\newcommand{\msD}{\ms D}
\newcommand{\msE}{\ms E}
\newcommand{\msF}{\ms F}
\newcommand{\msG}{\ms G}
\newcommand{\msH}{\ms H}
\newcommand{\msI}{\ms I}
\newcommand{\msJ}{\ms J}
\newcommand{\msK}{\ms K}
\newcommand{\msL}{\ms L}
\newcommand{\msM}{\ms M}
\newcommand{\msN}{\ms N}
\newcommand{\msO}{\ms O}
\newcommand{\msP}{\ms P}
\newcommand{\msQ}{\ms Q}
\newcommand{\msR}{\ms R}
\newcommand{\msS}{\ms S}
\newcommand{\msT}{\ms T}
\newcommand{\msU}{\ms U}
\newcommand{\msV}{\ms V}
\newcommand{\msW}{\ms W}
\newcommand{\msX}{\ms X}
\newcommand{\msY}{\ms Y}
\newcommand{\msZ}{\ms Z}
\newcommand{\mfA}{\mf A}
\newcommand{\mfB}{\mf B}
\newcommand{\mfC}{\mf C}
\newcommand{\mfD}{\mf D}
\newcommand{\mfE}{\mf E}
\newcommand{\mfF}{\mf F}
\newcommand{\mfG}{\mf G}
\newcommand{\mfH}{\mf H}
\newcommand{\mfI}{\mf I}
\newcommand{\mfJ}{\mf J}
\newcommand{\mfK}{\mf K}
\newcommand{\mfL}{\mf L}
\newcommand{\mfM}{\mf M}
\newcommand{\mfN}{\mf N}
\newcommand{\mfO}{\mf O}
\newcommand{\mfP}{\mf P}
\newcommand{\mfQ}{\mf Q}
\newcommand{\mfR}{\mf R}
\newcommand{\mfS}{\mf S}
\newcommand{\mfT}{\mf T}
\newcommand{\mfU}{\mf U}
\newcommand{\mfV}{\mf V}
\newcommand{\mfW}{\mf W}
\newcommand{\mfX}{\mf X}
\newcommand{\mfY}{\mf Y}
\newcommand{\mfZ}{\mf Z}
\newcommand{\mfa}{\mf a}
\newcommand{\mfb}{\mf b}
\newcommand{\mfc}{\mf c}
\newcommand{\mfd}{\mf d}
\newcommand{\mfe}{\mf e}
\newcommand{\mff}{\mf f}
\newcommand{\mfg}{\mf g}
\newcommand{\mfh}{\mf h}
\newcommand{\mfi}{\mf i}
\newcommand{\mfj}{\mf j}
\newcommand{\mfk}{\mf k}
\newcommand{\mfl}{\mf l}
\newcommand{\mfm}{\mf m}
\newcommand{\mfn}{\mf n}
\newcommand{\mfo}{\mf o}
\newcommand{\mfp}{\mf p}
\newcommand{\mfq}{\mf q}
\newcommand{\mfr}{\mf r}
\newcommand{\mfs}{\mf s}
\newcommand{\mft}{\mf t}
\newcommand{\mfu}{\mf u}
\newcommand{\mfv}{\mf v}
\newcommand{\mfw}{\mf w}
\newcommand{\mfx}{\mf x}
\newcommand{\mfy}{\mf y}
\newcommand{\mfz}{\mf z}
\newcommand{\mbfA}{\mathbf A}
\newcommand{\mbfB}{\mathbf B}
\newcommand{\mbfC}{\mathbf C}
\newcommand{\mbfD}{\mathbf D}
\newcommand{\mbfE}{\mathbf E}
\newcommand{\mbfF}{\mathbf F}
\newcommand{\mbfG}{\mathbf G}
\newcommand{\mbfH}{\mathbf H}
\newcommand{\mbfI}{\mathbf I}
\newcommand{\mbfJ}{\mathbf J}
\newcommand{\mbfK}{\mathbf K}
\newcommand{\mbfL}{\mathbf L}
\newcommand{\mbfM}{\mathbf M}
\newcommand{\mbfN}{\mathbf N}
\newcommand{\mbfO}{\mathbf O}
\newcommand{\mbfP}{\mathbf P}
\newcommand{\mbfQ}{\mathbf Q}
\newcommand{\mbfR}{\mathbf R}
\newcommand{\mbfS}{\mathbf S}
\newcommand{\mbfT}{\mathbf T}
\newcommand{\mbfU}{\mathbf U}
\newcommand{\mbfV}{\mathbf V}
\newcommand{\mbfW}{\mathbf W}
\newcommand{\mbfX}{\mathbf X}
\newcommand{\mbfY}{\mathbf Y}
\newcommand{\mbfZ}{\mathbf Z}
\newcommand{\bs}{\backslash}
\newcommand{\bbRn}{\mathbb R^n}
\newcommand{\ecyc}[1]{\langle #1\rangle}
\newcommand{\interior}[1]{\overset{\circ}{#1}}
\newcommand{\ol}[1]{\overline{#1}}
\newcommand{\unitgrp}[1]{#1^{\mspace{-4mu}\times}}
\newcommand{\grad}{\triangledown}
\newcommand{\normal}{\trianglelefteq}
\newcommand{\nnormal}{\ntrianglelefteq}
\newcommand{\norm}[1]{\left\lVert#1\right\rVert}
\newcommand{\gen}[1]{\langle #1 \rangle}
\newcommand{\nilradical}[1]{\sqrt{\gen{0}}_{#1}}
\newcommand{\inv}[1]{#1^{\text{-}1}}
\newcommand{\id}{\mathrm{id}}
\newcommand{\unitcell}[1]{\overset{\!\!\!\! \circ}{\bbD^{#1}}}
\newcommand{\E}{\exists}
\newcommand{\A}{\forall}

\DeclareMathOperator{\GL}{\rm{GL}}
\DeclareMathOperator{\SL}{\rm{SL}}
\DeclareMathOperator{\Mod}{\rm{Mod}}
\DeclareMathOperator{\Mor}{\rm{Mor}}
\DeclareMathOperator{\Obj}{\rm{Obj}}
\DeclareMathOperator{\Id}{\rm{Id}}
\DeclareMathOperator{\End}{\rm{End}}
\DeclareMathOperator{\Hom}{\rm{Hom}}
\DeclareMathOperator{\Res}{\rm{Res}}
\DeclareMathOperator{\Spec}{\rm{Spec}}
\DeclareMathOperator{\Proj}{\rm{Proj}}
\DeclareMathOperator{\Supp}{\rm{Supp}}
\DeclareMathOperator{\Ker}{\rm{Ker}}
\DeclareMathOperator{\Nil}{\rm{Nil}}
\DeclareMathOperator{\sh}{\rm{sh}}
\DeclareMathOperator{\Coker}{\rm{Coker}}
\DeclareMathOperator{\Rank}{\rm{Rank}}
\DeclareMathOperator{\Frac}{\rm{Frac}}
\DeclareMathOperator{\Rad}{\rm{Rad}}
\DeclareMathOperator{\Ann}{\rm{Ann}}
\DeclareMathOperator{\Disc}{\rm{Disc}}
\DeclareMathOperator{\Lcm}{\rm{lcm}}
\DeclareMathOperator{\Gcd}{\rm{gcd}}
\DeclareMathOperator{\Conv}{Conv}
\DeclareMathOperator{\Cone}{Cone}
\DeclareMathOperator{\Int}{Int}
\DeclareMathOperator{\Ord}{Ord}
\DeclareMathOperator{\Ass}{Ass}
\DeclareMathOperator{\Aut}{Aut}
\DeclareMathOperator{\Sym}{Sym}
\DeclareMathOperator{\Char}{Char}
\DeclareMathOperator{\Span}{Span}
\DeclareMathOperator{\Tor}{Tor}
\DeclareMathOperator{\Ext}{Ext}
\DeclareMathOperator{\Card}{Card}
\DeclareMathOperator{\OPT}{OPT}
\DeclareMathOperator{\Dom}{Dom}
\DeclareMathOperator{\Var}{Var}
\DeclareMathOperator{\Th}{Th}
\DeclareMathOperator{\Jac}{\rm{Jac}}
\DeclareMathOperator{\Iso}{\rm{Iso}}
\DeclareMathOperator{\Rel}{\mathrm{rel}}
\DeclareMathOperator{\Specmax}{\rm{Spec_{max}}}
\renewcommand{\Re}{\operatorname{Re}}
\renewcommand{\Im}{\operatorname{Im}}
\DeclareMathOperator{\Adj}{Adj}
\DeclarePairedDelimiter{\ceil}{\lceil}{\rceil}
\DeclarePairedDelimiter{\floor}{\lfloor}{\rfloor}

% above : global newcommands and global operators
% 
% below : local newcommands amd local operators

\newcommand{\phead}{\hspace*{0.46cm}}
\DeclareMathOperator{\Homtop}{\rm{Hom}_{\mbf{Top}}}
\DeclareMathOperator{\Maptop}{\rm{Map}_{\mbf{Top}}}


\begin{document}
    \title{Function Spaces}
    \author{{\color{pink}{Cloudi}}{\color{Aquamarine}{fold}}}
    \maketitle
    \newpage

    \setcounter{section}{-1}

    \section{Notations}

    \begin{align*}
        & \text{Category of sets } &: \ & \mbf{Set} \\
        & \text{Category of topological spaces }&: \ & \mbf{Top} \\
        & \text{Category of (one-point-)based topological spaces }&: \  & \mbf{Top}_* \\
        & \text{Category of pairs $(X,A)$ of space $X$ and subspace $A$ } &: \ & \mbf{Top}(2) \\
        & \text{Topological space $X$ with topology $\mcT$ } &: \ & X_{\mcT}\\
        & \text{Euclidean space of dimension $n$ } &: \ & \bbRn\\
        & \text{Unit cube of dimension $n$ } &: \ & I^n\\
        & \text{Boundary of $I^n$ }          &: \ & \partial I^n\\
        & \text{Unit interval $I$}           &: \ & I = I^1\\
        & \text{Unit cell of dimension $n$ } &: \ & \unitcell{n} \\
        & \text{Unit disk of dimension $n$ } &: \ & \bbD^n\\
        & \text{Unit sphere of dimension $n-1$ } &: \ & \bbS^{n-1}\\
        & \text{Inclusion or Embedding } &: \ & \hookrightarrow \\
        & \text{Monomorphsim }      &: \ & \rightarrowtail \\
        & \text{Epimorphsim }       &: \ & \twoheadrightarrow \\
        & \text{Hom functor of category $\mcC $} &: \ & \Hom_{\mcC}(-,-)\\
        & \text{Limit (inverse limit) (projective limit) } &: \ & \lim_{\leftarrow} \\
        & \text{Colimit (direct limit) (inductive limit) } &: \ & \lim_{\rightarrow} \\
    \end{align*}


    \section{Function Spaces}

    \setsubsectcounter{-1}

    \subsection{Introduction}

    Function spaces, are origins of many important constructions such as
    Loop spaces, Path spaces and so on.
    The duality between function spaces and product spaces will [ todo ]

    \subsection{Admissible Topology}

    \begin{defn}
        A topology on $\Homtop(X, Y)$ is $\mbf{admissible}$ if 
        the evaluation function $ev$ is $\mbf{continuous}$. Where $ev$ is defined by : 
        \begin{align*}
            ev  : \Homtop(X,Y) \times X & \to Y \\
            (f, x)& \mapsto f(x)
        \end{align*}
    \end{defn}

    \begin{note}
        It is possible that $\Homtop(X,Y)$ have $\mbf{no}$ admissible topology.
    \end{note}

    \subsection{Compact-Open Topology}

    \begin{defn}
        The $\mbf{compact\text{-}open}$ topology on $\Homtop(X,Y)$ is generated by 
        subbase $\{O^K\}$ where $K$ varies on all compact subsets of $X$, 
        $O$ varies on all open subsets of $Y$. 
        The definition of $O^K$ is :
        $$ O^K := \{ f \in \Homtop(X,Y) \mid f(K) \subseteq O \} $$
        We note the compact-open topology by $\mcT_{co}$
    \end{defn}

    \begin{prop}
        Property of $\mbf{compact\text{-}open}$ topology : The compact-open topology
        is coarser than any admissible topology.
        (That is, for any admissible topology $\mcT$, $\mcT_{co} \subseteq \mcT$)
    \end{prop}
    \begin{prf}
        We have to show that any open set in $\mcT_{co}$ is open in $\mcT$ if $\mcT$ is admissible.
        It suffices to show that every $O^K \in \mcT$. By definition, we have:
        $$ ev : \Homtop(X,Y)_{\mcT} \times X \to Y $$
        is continuous. Take $k \in K$ and $f \in O^K$ (That is, $f(K) \in O$).
        \par By $ev$ is continuous,
        $ ev(f, k) = f(k) \in O$ and the property of the base of finite product topology,
        we have

        $$
        \E V_{f,k}, W_k \ .\ 
        f \in V_{f,k} \in \mcT \text{ and } k \in W_k \in \mcT_Y \text{ and } ev(V_{f,k} \times W_k) \subseteq O
        $$

        \par The family $\{W_k\}_{k \in K}$ is an open cover of $K$. By compactness of $K$,
        There exists a finite subcover $\{W_{k_i}\}_{i=1,\dots ,n\ (k_i \in K)}$. Put
        $V_f := \bigcap \{ V_{f,k_i} \}_{i=1,\dots ,n} $ ( with $ev(V_{f,k_i} \times W_{k_i}) \subseteq O$ ), we have $f \in V_f$
        and $V_f$ is open in $\Homtop(X,Y)_{\mcT}$ .
        \par Then we have $V_f \subseteq O^K$, since \\
        \begin{center}
        \begin{prooftree}
            \hypo{ k \in K }
            \infer1{ \E k_i \in K \ . \ k \in W_{k_i} }
            \hypo{ g \in V_f }
            \infer2[r1]{ g(k) \in O }
        \end{prooftree}
        $\Rightarrow$
        \begin{prooftree}
            \hypo{g \in V_f }
            \infer1{ g(K) \subseteq O }
        \end{prooftree}
        \ \\
        \ \\
        r1 : $g(k)=ev(g,k) \in ev(V_f \times W_{k_i}) \subseteq ev(V_{f,k_i} \times W_{k_i}) \subseteq O $\\
        \end{center}
        
        
        So, $O^K = \bigcup \{V_f\}_{f \in O^K}$, which is a union of open sets in $\Homtop(X,Y)_{\mcT}$.
        That is, $O^K \in \mcT$. 
        \qed
    \end{prf}

    \begin{note}
        Now we denote $\Homtop(X,Y)_{\mcT_{co}}$ simply by $\Maptop(X,Y)$.
    \end{note}

    \begin{prop}
        If X is locally compact and Hausdorff, then the $\mbf{compact\text{-}open}$
        topology is admissible.
    \end{prop}
    \begin{prf}
        We have to show $ev : \Homtop(X,Y)_{\mcT_{co}} \times X \to Y$ is continuous.
        That is $V \in \mcT_Y \to \inv{ev}(V) \in \mcT_{\Homtop(X,Y) \times X }$.
        By definition, $\inv{ev}(V) = \{ (f,x) \mid f(x) \in V \}$,
        We take $(f,x) \in \inv{ev}(V)$ for the next step.
        \par  By continity of $f$, we have $\inv{f} (V)$ is open in $X$.
        By locally compactness of $X$ and $X$ is Hausdorff, there exist
        $ O_{(f,x)} \in \mcT_X$ such that $x \in O_{(f,x)} \subseteq \ol{O_{(f,x)}} \subseteq V$
        and $\ol{O_{(f,x)}}$ is compact. Put $K_{(f,x)} := \ol{O_{(f,x)}}$
        \par Now we have $(f,x) \in V^{K_{(f,x)}} \times O_{(f,x)} \subseteq \inv{ev}(V)$, that means
        $$ \inv{ev}(V) = \bigcup \left\{ V^{K_{(f,x)}} \times O_{(f,x)} \right\}_{(f,x) \in \inv{ev}(V)} $$ 
        is a union of open sets. That is, $\inv{ev}(V)$ is open.
        \qed
    \end{prf}

    
    \section{Compactly Generated Spaces}

    \setsubsectcounter{-1}

    \subsection{Introduction}

    A $\mbf{compactly\ generated}$ space (in a certain sense) is such a space that the continuous images in it of
    all $\mbf{compact\ Hausdorff}$ spaces tell you everything about its topology.
    \par Why $\mbf{compact\ Hausdorff}$? Maybe the reason is that $\mbf{compact}$ implies existance of limit(1),
    and $\mbf{Hausdorff}$ implies the uniqueness of limit(1).  

    \subsection{Related Definitios}

    \begin{defn}
        A function $f : X \to Y$ between the underlying set of topological spaces
        is $k\mbf{\text{-}continuous}$ if for all $\mbf{compact\ Hausdorff}$ spaces $C$
        and continuous functions $t : C \to X$, $f \circ t : C \to Y$ is continuous. 
    \end{defn}

    \begin{defn}
        A topological space $X$ is a $k\mbf{\text{-}space}$ if forall $f : X \to Y$ (in $\mbf{Set}$),
        $f$ is continuous $\Leftrightarrow$ $f$ is $k\mbf{\text{-}continuous}$
    \end{defn}

    \begin{note}
        Equivalent definitions of $k\mbf{\text{-}space}$ 
    \end{note}

    \subsection{Category of Compactly Generated Spaces}


    
    
\end{document}