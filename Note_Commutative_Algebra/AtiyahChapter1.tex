\documentclass[a4paper]{article}
\usepackage[dvipsnames, svgnames, x11names]{xcolor}
\usepackage{tikz-cd}
\tikzcdset{row sep/normal=1cm}
\usepackage{amsfonts}
\usepackage{hyperref}
\usepackage{amsthm}
\usepackage{enumerate}
\usepackage[left=3cm, right=3cm, top=2cm]{geometry}
\usepackage{sectsty}
\usepackage[perpage]{footmisc}
\usepackage{amsthm}
\usepackage{tikz-cd}
\usepackage{pgfplots}
\usepackage{amsmath}
\usepackage{amssymb}
\usepackage{pdfpages}
\usepackage{multirow}
\usepackage{mathtools}
\usepackage{mathrsfs}
\usepackage{ebproof}
\usepackage{graphicx}
\usepackage{epigraph}
\usepackage{enumitem}
\theoremstyle{plain}
\newtheorem{thm}{Theorem}[section]
\newtheorem{lem}[thm]{Lemma}
\newtheorem{prop}[thm]{Proposition}
\newtheorem*{cor}{Corollary}

\theoremstyle{definition}
\newtheorem{defn}{Definition}[section]
\newtheorem{conj}{Conjecture}[section]
\newtheorem{exmp}{Example}[section]

\newtheorem*{prf}{Proof}

\theoremstyle{remark}
\newtheorem*{rem}{Remark}
\newtheorem*{note}{Note}
\newtheorem{ex}{Exercise}[section]
\newtheorem{sol}{Solution}[section]

\definecolor{background}{RGB}{255,230,230}
\pagecolor{background}

\definecolor{TextColor}{RGB}{255,122,162}

\chapterfont{\color{TextColor}}
\sectionfont{\color{TextColor}\Huge}
\setlist{nolistsep,topsep=-7pt}
\pgfplotsset{width=10cm,compat=newest}
%\usepgfplotslibrary{external}
%\tikzexternalize[prefix=tikz/]
\newcommand{\ssubparagraph}{ \parshape 1 1cm \dimexpr\linewidth-2cm\relax}
%\numberwithin{equation}{section}

% \quad : width of  1 'M'
% \qquad : 4
% \0x20 (\ ) : 1/3
% \; : 2/7
% \, : 1/6
% \! : -1/6

% \addtocounter{<envname>}{nubmer}
% Expl. \addtocounter{thm}{2}
% \setcounter{<envname>}{nubmer}
% Expl. \setcounter{ex}{3}
% -->  Exercise 1.4

\newcommand{\setexcounter}[1]{
    \setcounter{ex}{#1}
    \setcounter{sol}{#1}
}
\newcommand{\setsubsectcounter}[1]{\setcounter{subsection}{#1}}

\newcommand{\bb}[1]{\mathbb{#1}}
\newcommand{\mc}[1]{\mathcal{#1}}
\newcommand{\mf}[1]{\mathfrak{#1}}
\newcommand{\ms}[1]{\mathscr{#1}}
\newcommand{\mbf}[1]{\mathbf{#1}}
\newcommand{\bbA}{\mathbb A}
\newcommand{\bbB}{\mathbb B}
\newcommand{\bbC}{\mathbb C}
\newcommand{\bbD}{\mathbb D}
\newcommand{\bbE}{\mathbb E}
\newcommand{\bbF}{\mathbb F}
\newcommand{\bbG}{\mathbb G}
\newcommand{\bbH}{\mathbb H}
\newcommand{\bbI}{\mathbb I}
\newcommand{\bbJ}{\mathbb J}
\newcommand{\bbK}{\mathbb K}
\newcommand{\bbL}{\mathbb L}
\newcommand{\bbM}{\mathbb M}
\newcommand{\bbN}{\mathbb N}
\newcommand{\bbO}{\mathbb O}
\newcommand{\bbP}{\mathbb P}
\newcommand{\bbQ}{\mathbb Q}
\newcommand{\bbR}{\mathbb R}
\newcommand{\bbS}{\mathbb S}
\newcommand{\bbT}{\mathbb T}
\newcommand{\bbU}{\mathbb U}
\newcommand{\bbV}{\mathbb V}
\newcommand{\bbW}{\mathbb W}
\newcommand{\bbX}{\mathbb X}
\newcommand{\bbY}{\mathbb Y}
\newcommand{\bbZ}{\mathbb Z}
\newcommand{\mcA}{\mc A}
\newcommand{\mcB}{\mc B}
\newcommand{\mcC}{\mc C}
\newcommand{\mcD}{\mc D}
\newcommand{\mcE}{\mc E}
\newcommand{\mcF}{\mc F}
\newcommand{\mcG}{\mc G}
\newcommand{\mcH}{\mc H}
\newcommand{\mcI}{\mc I}
\newcommand{\mcJ}{\mc J}
\newcommand{\mcK}{\mc K}
\newcommand{\mcL}{\mc L}
\newcommand{\mcM}{\mc M}
\newcommand{\mcN}{\mc N}
\newcommand{\mcO}{\mc O}
\newcommand{\mcP}{\mc P}
\newcommand{\mcQ}{\mc Q}
\newcommand{\mcR}{\mc R}
\newcommand{\mcS}{\mc S}
\newcommand{\mcT}{\mc T}
\newcommand{\mcU}{\mc U}
\newcommand{\mcV}{\mc V}
\newcommand{\mcW}{\mc W}
\newcommand{\mcX}{\mc X}
\newcommand{\mcY}{\mc Y}
\newcommand{\mcZ}{\mc Z}
\newcommand{\msA}{\ms A}
\newcommand{\msB}{\ms B}
\newcommand{\msC}{\ms C}
\newcommand{\msD}{\ms D}
\newcommand{\msE}{\ms E}
\newcommand{\msF}{\ms F}
\newcommand{\msG}{\ms G}
\newcommand{\msH}{\ms H}
\newcommand{\msI}{\ms I}
\newcommand{\msJ}{\ms J}
\newcommand{\msK}{\ms K}
\newcommand{\msL}{\ms L}
\newcommand{\msM}{\ms M}
\newcommand{\msN}{\ms N}
\newcommand{\msO}{\ms O}
\newcommand{\msP}{\ms P}
\newcommand{\msQ}{\ms Q}
\newcommand{\msR}{\ms R}
\newcommand{\msS}{\ms S}
\newcommand{\msT}{\ms T}
\newcommand{\msU}{\ms U}
\newcommand{\msV}{\ms V}
\newcommand{\msW}{\ms W}
\newcommand{\msX}{\ms X}
\newcommand{\msY}{\ms Y}
\newcommand{\msZ}{\ms Z}
\newcommand{\mfA}{\mf A}
\newcommand{\mfB}{\mf B}
\newcommand{\mfC}{\mf C}
\newcommand{\mfD}{\mf D}
\newcommand{\mfE}{\mf E}
\newcommand{\mfF}{\mf F}
\newcommand{\mfG}{\mf G}
\newcommand{\mfH}{\mf H}
\newcommand{\mfI}{\mf I}
\newcommand{\mfJ}{\mf J}
\newcommand{\mfK}{\mf K}
\newcommand{\mfL}{\mf L}
\newcommand{\mfM}{\mf M}
\newcommand{\mfN}{\mf N}
\newcommand{\mfO}{\mf O}
\newcommand{\mfP}{\mf P}
\newcommand{\mfQ}{\mf Q}
\newcommand{\mfR}{\mf R}
\newcommand{\mfS}{\mf S}
\newcommand{\mfT}{\mf T}
\newcommand{\mfU}{\mf U}
\newcommand{\mfV}{\mf V}
\newcommand{\mfW}{\mf W}
\newcommand{\mfX}{\mf X}
\newcommand{\mfY}{\mf Y}
\newcommand{\mfZ}{\mf Z}
\newcommand{\mfa}{\mf a}
\newcommand{\mfb}{\mf b}
\newcommand{\mfc}{\mf c}
\newcommand{\mfd}{\mf d}
\newcommand{\mfe}{\mf e}
\newcommand{\mff}{\mf f}
\newcommand{\mfg}{\mf g}
\newcommand{\mfh}{\mf h}
\newcommand{\mfi}{\mf i}
\newcommand{\mfj}{\mf j}
\newcommand{\mfk}{\mf k}
\newcommand{\mfl}{\mf l}
\newcommand{\mfm}{\mf m}
\newcommand{\mfn}{\mf n}
\newcommand{\mfo}{\mf o}
\newcommand{\mfp}{\mf p}
\newcommand{\mfq}{\mf q}
\newcommand{\mfr}{\mf r}
\newcommand{\mfs}{\mf s}
\newcommand{\mft}{\mf t}
\newcommand{\mfu}{\mf u}
\newcommand{\mfv}{\mf v}
\newcommand{\mfw}{\mf w}
\newcommand{\mfx}{\mf x}
\newcommand{\mfy}{\mf y}
\newcommand{\mfz}{\mf z}
\newcommand{\mbfA}{\mathbf A}
\newcommand{\mbfB}{\mathbf B}
\newcommand{\mbfC}{\mathbf C}
\newcommand{\mbfD}{\mathbf D}
\newcommand{\mbfE}{\mathbf E}
\newcommand{\mbfF}{\mathbf F}
\newcommand{\mbfG}{\mathbf G}
\newcommand{\mbfH}{\mathbf H}
\newcommand{\mbfI}{\mathbf I}
\newcommand{\mbfJ}{\mathbf J}
\newcommand{\mbfK}{\mathbf K}
\newcommand{\mbfL}{\mathbf L}
\newcommand{\mbfM}{\mathbf M}
\newcommand{\mbfN}{\mathbf N}
\newcommand{\mbfO}{\mathbf O}
\newcommand{\mbfP}{\mathbf P}
\newcommand{\mbfQ}{\mathbf Q}
\newcommand{\mbfR}{\mathbf R}
\newcommand{\mbfS}{\mathbf S}
\newcommand{\mbfT}{\mathbf T}
\newcommand{\mbfU}{\mathbf U}
\newcommand{\mbfV}{\mathbf V}
\newcommand{\mbfW}{\mathbf W}
\newcommand{\mbfX}{\mathbf X}
\newcommand{\mbfY}{\mathbf Y}
\newcommand{\mbfZ}{\mathbf Z}
\newcommand{\bs}{\backslash}
\newcommand{\bbRn}{\mathbb R^n}
\newcommand{\ecyc}[1]{\langle #1\rangle}
\newcommand{\interior}[1]{\overset{\circ}{#1}}
\newcommand{\ol}[1]{\overline{#1}}
\newcommand{\unitgrp}[1]{#1^{\mspace{-4mu}\times}}
\newcommand{\grad}{\triangledown}
\newcommand{\normal}{\trianglelefteq}
\newcommand{\nnormal}{\ntrianglelefteq}
\newcommand{\norm}[1]{\left\lVert#1\right\rVert}
\newcommand{\gen}[1]{\langle #1 \rangle}
\newcommand{\nilradical}[1]{\sqrt{\gen{0}}_{#1}}
\newcommand{\inv}[1]{#1^{\text{-}1}}
\newcommand{\id}{\mathrm{id}}
\newcommand{\unitcell}[1]{\overset{\!\!\!\! \circ}{\bbD^{#1}}}
\newcommand{\E}{\exists}
\newcommand{\A}{\forall}

\DeclareMathOperator{\GL}{\rm{GL}}
\DeclareMathOperator{\SL}{\rm{SL}}
\DeclareMathOperator{\Mod}{\rm{Mod}}
\DeclareMathOperator{\Mor}{\rm{Mor}}
\DeclareMathOperator{\Obj}{\rm{Obj}}
\DeclareMathOperator{\Id}{\rm{Id}}
\DeclareMathOperator{\End}{\rm{End}}
\DeclareMathOperator{\Hom}{\rm{Hom}}
\DeclareMathOperator{\Res}{\rm{Res}}
\DeclareMathOperator{\Spec}{\rm{Spec}}
\DeclareMathOperator{\Proj}{\rm{Proj}}
\DeclareMathOperator{\Supp}{\rm{Supp}}
\DeclareMathOperator{\Ker}{\rm{Ker}}
\DeclareMathOperator{\Nil}{\rm{Nil}}
\DeclareMathOperator{\sh}{\rm{sh}}
\DeclareMathOperator{\Coker}{\rm{Coker}}
\DeclareMathOperator{\Rank}{\rm{Rank}}
\DeclareMathOperator{\Frac}{\rm{Frac}}
\DeclareMathOperator{\Rad}{\rm{Rad}}
\DeclareMathOperator{\Ann}{\rm{Ann}}
\DeclareMathOperator{\Disc}{\rm{Disc}}
\DeclareMathOperator{\Lcm}{\rm{lcm}}
\DeclareMathOperator{\Gcd}{\rm{gcd}}
\DeclareMathOperator{\Conv}{Conv}
\DeclareMathOperator{\Cone}{Cone}
\DeclareMathOperator{\Int}{Int}
\DeclareMathOperator{\Ord}{Ord}
\DeclareMathOperator{\Ass}{Ass}
\DeclareMathOperator{\Aut}{Aut}
\DeclareMathOperator{\Sym}{Sym}
\DeclareMathOperator{\Char}{Char}
\DeclareMathOperator{\Span}{Span}
\DeclareMathOperator{\Tor}{Tor}
\DeclareMathOperator{\Ext}{Ext}
\DeclareMathOperator{\Card}{Card}
\DeclareMathOperator{\OPT}{OPT}
\DeclareMathOperator{\Dom}{Dom}
\DeclareMathOperator{\Var}{Var}
\DeclareMathOperator{\Th}{Th}
\DeclareMathOperator{\Jac}{\rm{Jac}}
\DeclareMathOperator{\Iso}{\rm{Iso}}
\DeclareMathOperator{\rel}{\mathrm{rel}}
\DeclareMathOperator{\Specmax}{\rm{Spec_{max}}}
\renewcommand{\Re}{\operatorname{Re}}
\renewcommand{\Im}{\operatorname{Im}}
\DeclareMathOperator{\Adj}{Adj}
\DeclarePairedDelimiter{\ceil}{\lceil}{\rceil}
\DeclarePairedDelimiter{\floor}{\lfloor}{\rfloor}

% above : global newcommands and global operators
% 
% below : local newcommands amd local operators

\begin{document}
    \title{Notes of Introducion to Commutative Algebra}
    \author{{\color{pink}{Cloudi}}{\color{Aquamarine}{fold}}}
    \maketitle
    \newpage

    \setcounter{section}{-1}

    \section{Notations}

    \vspace*{0.3cm}

    \begin{align*}
        & \text{Category of commutative rings }      &: \ & \mbf{CRing}\\
        & \text{Coordinate ring of $k$-variety $X$ } &: \ & k[X]\\
        & \text{Element in polynomial ring $A[x]$ }  &: \ & c \in A[x] := c=\sum_{i}c_i x^i \ (\A i \ .\ c_i \in A)\\
        & \text{Jacobson radical of ring $R$ }       &: \ & \Jac(R)\\
        & \text{Polynomial ring with $n$-variables over ring $R$ } &: \ & R[x]_n \text{ or } R[x_1, \dotsc, x_n]\\
        & \text{Nilradical of ring $A$ }             &: \ & \nilradical{A}\\
        & \text{Radical ideal of ideal $I$ }         &: \ & \sqrt{I}\\
    \end{align*}


    \section{Exercises of Chapter 1 : Rings and Ideals}

    \vspace*{0.9cm}
    
    \begin{ex}
        $A \in \mbf{CRing} $, $x \in \sqrt{\gen{0}}$ $\to$ $u \in \unitgrp{A} \to u+x \in \unitgrp{A}$
    \end{ex}

    \begin{sol}
        \ \\
        \begin{prooftree}
            \hypo{ x \in \sqrt{\gen{0}} }
            \infer1{ \E d \in \bbZ \ .\ x^{d} = 0 }
            \infer1[w1]{ (1+x)\cdot (\sum^{d-1}_{i=0}(-x)^{i})=1 }
            \infer1{ 1+x \in \unitgrp{A} }
            \hypo{ u \in \unitgrp{A} }
            \infer1{ 1+x \in \unitgrp{A} \leftrightarrow  u+ux \in \unitgrp{A} }
            \hypo{ x \in \sqrt{\gen{0}} }
            \infer1{ \A r \in A \ .\  rx \in \sqrt{\gen{0}} }
            \infer3[w2]{ u+x \in \unitgrp{A}}
        \end{prooftree}\\
        \begin{center}
        \begin{align*}
            & \text{w1} : 1-x^n = (1+x)\cdot \left( \sum^{n-1}_{i=0}x^{i} \right) \\
            & \text{w2} : u(1+\inv{u}x) = u+x,  \inv{u}x \in \unitgrp{A}
        \end{align*}
        \end{center}
        \qed
    \end{sol}

    \setexcounter{3}

    \begin{ex}
        $A \in \mbf{CRing} \to \Jac(A[x]) = \sqrt{\gen{0}}_{A[x]}$
    \end{ex}

    \begin{sol}
        \ \\
        \begin{center}
        \begin{prooftree}
            \hypo{ \text{lemma 1} }
            \hypo{ \text{lemma 2} }
            \infer2{ c \in \sqrt{\gen{0}}_{A[x]} \leftrightarrow  \A i \ .\ c_i \in \sqrt{\gen{0}}_A }
            \infer1[w1]{ q \in \Jac(A[x]) \to q \in \nilradical{A[x]} }
            \hypo{ \text{def 1} }
            \hypo{ \text{def 2} }
            \infer2[w2]{ \Jac(A[x])\supseteq  \nilradical{A[x]}}
            \infer2{ \Jac(A[x]) = \sqrt{\gen{0}}_{A[x]} }
        \end{prooftree}\\
        \begin{align*}
            & \text{def 1} : \Jac(A[x])=\bigcap_{m\in \Specmax(A[x])}m\\
            & \text{def 2} : \sqrt{\gen{0}}_{A[x]}=\bigcap_{P\in \Spec(A[x])}P\\
            & \text{lemma 1} : q \in \Jac(A[x]) \leftrightarrow \A y \in A[x]\ .\ 1 - qy \in \unitgrp{A[x]} &  \text{(Textbook Ch1 Prop 1.9)}\\
            & \text{lemma 2} : c \in \unitgrp{A[x]} \to , c_0 \in \unitgrp{A} \ \rm{and}\  \A i\neq 0 \ .\ c_i \in \sqrt{\gen{0}}_A & \text{(Textbook Ch1 Ex 2.i)}\\
            & \text{w1} : q \in \Jac(A[x]) \to 1 - q \in \unitgrp{A[x]} \to \A i, q_i \in \sqrt{\gen{0}}_{A[x]} \to q \in \nilradical{A[x]}\\
            & \text{w2} : I \in \Specmax{A} \to I \in \Spec{A}
        \end{align*}
        \end{center}
        \qed
    \end{sol}

    \setexcounter{5}

    \begin{ex}
        $A \in \mbf{CRing} \to \left( I \nsubseteq  \nilradical{A} \to \E \ e \in I, e^2 = e \neq 0 \right) \to \Jac{A} = \nilradical{A} $
    \end{ex}

    \begin{sol}
        \ \\
        \begin{center}
        \begin{prooftree}
            \hypo{I  \nsubseteq  \nilradical{A} \to \E \ e \in I, e^2 = e \neq 0 }
            \hypo{ \Jac{A} \nsubseteq \nilradical{A} }
            \infer2{ \E e \in \Jac{A} , e^2 = e \neq 0 }
            \infer1[lemma 1]{ 1 - e \in \unitgrp{A} \ \text{and} \  e(1-e)=0}
            \infer1{e = e (1 - e) \inv{(1 - e)} = 0 \inv{(1 - e)} = 0}
            \infer1{\bot}
        \end{prooftree}
        $\Rightarrow \Jac{A} \subseteq \nilradical{A}$\\
        \ \\
        \ \\
        \begin{prooftree}
            \hypo{ \Jac{A} \subseteq \nilradical{A} }
            \hypo{ \Jac(A)\supseteq  \sqrt{\gen{0}}_{A} \ \text{(lemma 2)}}
            \infer2{\Jac{A} = \nilradical{A}}
        \end{prooftree}\\
        \begin{align*}
            & \text{lemma 1} : q \in \Jac(A) \leftrightarrow \A y \in A\ .\ 1 - qy \in \unitgrp{A} &  \text{(Textbook Ch1 Prop 1.9)}\\
            & \text{lemma 2} : \text{Sol 1.4. w2}
        \end{align*}
        \end{center}
        \qed
    \end{sol}
    

    

\end{document}