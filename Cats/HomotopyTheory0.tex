\documentclass[a4paper]{article}
\usepackage[dvipsnames, svgnames, x11names]{xcolor}
\usepackage{tikz-cd}
\tikzcdset{row sep/normal=1cm}
\tikzcdset{column sep/normal=1cm}
\usepackage{amsfonts}
\usepackage{hyperref}
\usepackage{amsthm}
\usepackage{enumerate}
\usepackage[left=3cm, right=3cm, top=2cm]{geometry}
\usepackage{sectsty}
\usepackage[perpage]{footmisc}
\usepackage{amsthm}
\usepackage{tikz-cd}
\usepackage{pgfplots}
\usepackage{amsmath}
\usepackage{amssymb}
\usepackage{pdfpages}
\usepackage{multirow}
\usepackage{mathtools}
\usepackage{mathrsfs}
\usepackage{ebproof}
\usepackage{graphicx}
\usepackage{epigraph}
\usepackage{enumitem}
\usepackage{indentfirst}

\usepackage{quiver}


\theoremstyle{plain}
\newtheorem{thm}{Theorem}[section]
\newtheorem{lem}[thm]{Lemma}
\newtheorem{prop}[thm]{Proposition}
\newtheorem*{cor}{Corollary}

\theoremstyle{definition}
\newtheorem{defn}{Definition}[section]
\newtheorem{conj}{Conjecture}[section]
\newtheorem{exmp}{Example}[section]
\newtheorem{intro}[defn]{Introduction}

\newtheorem*{prf}{Proof}

\theoremstyle{remark}
\newtheorem*{rem}{Remark}
\newtheorem*{note}{Note}
\newtheorem{exercise}{Exercise}[section]
\newtheorem{solution}{Solution}[section]

\definecolor{background}{RGB}{255,230,230}
\pagecolor{background}

\tikzcdset{background color = background}

\definecolor{TextColor}{RGB}{255,122,162}

\chapterfont{\color{TextColor}}
\sectionfont{\color{TextColor}\Huge}
\setlist{nolistsep,topsep=-7pt}
\pgfplotsset{width=10cm,compat=newest}
%\usepgfplotslibrary{external}
%\tikzexternalize[prefix=tikz/]
\newcommand{\ssubparagraph}{ \parshape 1 1cm \dimexpr\linewidth-2cm\relax}
%\numberwithin{equation}{section}

% \quad : width of  1 'M'
% \qquad : 4
% \0x20 (\ ) : 1/3
% \; : 2/7
% \, : 1/6
% \! : -1/6

% \addtocounter{<envname>}{nubmer}
% Expl. \addtocounter{thm}{2}
% \setcounter{<envname>}{nubmer}
% Expl. \setcounter{ex}{3}
% -->  Exercise 1.4

\newcommand{\setexcounter}[1]{
    \setcounter{ex}{#1}
    \setcounter{sol}{#1}
}
\newcommand{\setsubsectcounter}[1]{\setcounter{subsection}{#1}}

\newcommand{\bb}[1]{\mathbb{#1}}
\newcommand{\mc}[1]{\mathcal{#1}}
\newcommand{\mf}[1]{\mathfrak{#1}}
\newcommand{\ms}[1]{\mathscr{#1}}
\newcommand{\mbf}[1]{\mathbf{#1}}
\newcommand{\bbA}{\mathbb A}
\newcommand{\bbB}{\mathbb B}
\newcommand{\bbC}{\mathbb C}
\newcommand{\bbD}{\mathbb D}
\newcommand{\bbE}{\mathbb E}
\newcommand{\bbF}{\mathbb F}
\newcommand{\bbG}{\mathbb G}
\newcommand{\bbH}{\mathbb H}
\newcommand{\bbI}{\mathbb I}
\newcommand{\bbJ}{\mathbb J}
\newcommand{\bbK}{\mathbb K}
\newcommand{\bbL}{\mathbb L}
\newcommand{\bbM}{\mathbb M}
\newcommand{\bbN}{\mathbb N}
\newcommand{\bbO}{\mathbb O}
\newcommand{\bbP}{\mathbb P}
\newcommand{\bbQ}{\mathbb Q}
\newcommand{\bbR}{\mathbb R}
\newcommand{\bbS}{\mathbb S}
\newcommand{\bbT}{\mathbb T}
\newcommand{\bbU}{\mathbb U}
\newcommand{\bbV}{\mathbb V}
\newcommand{\bbW}{\mathbb W}
\newcommand{\bbX}{\mathbb X}
\newcommand{\bbY}{\mathbb Y}
\newcommand{\bbZ}{\mathbb Z}
\newcommand{\mcA}{\mc A}
\newcommand{\mcB}{\mc B}
\newcommand{\mcC}{\mc C}
\newcommand{\mcD}{\mc D}
\newcommand{\mcE}{\mc E}
\newcommand{\mcF}{\mc F}
\newcommand{\mcG}{\mc G}
\newcommand{\mcH}{\mc H}
\newcommand{\mcI}{\mc I}
\newcommand{\mcJ}{\mc J}
\newcommand{\mcK}{\mc K}
\newcommand{\mcL}{\mc L}
\newcommand{\mcM}{\mc M}
\newcommand{\mcN}{\mc N}
\newcommand{\mcO}{\mc O}
\newcommand{\mcP}{\mc P}
\newcommand{\mcQ}{\mc Q}
\newcommand{\mcR}{\mc R}
\newcommand{\mcS}{\mc S}
\newcommand{\mcT}{\mc T}
\newcommand{\mcU}{\mc U}
\newcommand{\mcV}{\mc V}
\newcommand{\mcW}{\mc W}
\newcommand{\mcX}{\mc X}
\newcommand{\mcY}{\mc Y}
\newcommand{\mcZ}{\mc Z}
\newcommand{\msA}{\ms A}
\newcommand{\msB}{\ms B}
\newcommand{\msC}{\ms C}
\newcommand{\msD}{\ms D}
\newcommand{\msE}{\ms E}
\newcommand{\msF}{\ms F}
\newcommand{\msG}{\ms G}
\newcommand{\msH}{\ms H}
\newcommand{\msI}{\ms I}
\newcommand{\msJ}{\ms J}
\newcommand{\msK}{\ms K}
\newcommand{\msL}{\ms L}
\newcommand{\msM}{\ms M}
\newcommand{\msN}{\ms N}
\newcommand{\msO}{\ms O}
\newcommand{\msP}{\ms P}
\newcommand{\msQ}{\ms Q}
\newcommand{\msR}{\ms R}
\newcommand{\msS}{\ms S}
\newcommand{\msT}{\ms T}
\newcommand{\msU}{\ms U}
\newcommand{\msV}{\ms V}
\newcommand{\msW}{\ms W}
\newcommand{\msX}{\ms X}
\newcommand{\msY}{\ms Y}
\newcommand{\msZ}{\ms Z}
\newcommand{\mfA}{\mf A}
\newcommand{\mfB}{\mf B}
\newcommand{\mfC}{\mf C}
\newcommand{\mfD}{\mf D}
\newcommand{\mfE}{\mf E}
\newcommand{\mfF}{\mf F}
\newcommand{\mfG}{\mf G}
\newcommand{\mfH}{\mf H}
\newcommand{\mfI}{\mf I}
\newcommand{\mfJ}{\mf J}
\newcommand{\mfK}{\mf K}
\newcommand{\mfL}{\mf L}
\newcommand{\mfM}{\mf M}
\newcommand{\mfN}{\mf N}
\newcommand{\mfO}{\mf O}
\newcommand{\mfP}{\mf P}
\newcommand{\mfQ}{\mf Q}
\newcommand{\mfR}{\mf R}
\newcommand{\mfS}{\mf S}
\newcommand{\mfT}{\mf T}
\newcommand{\mfU}{\mf U}
\newcommand{\mfV}{\mf V}
\newcommand{\mfW}{\mf W}
\newcommand{\mfX}{\mf X}
\newcommand{\mfY}{\mf Y}
\newcommand{\mfZ}{\mf Z}
\newcommand{\mfa}{\mf a}
\newcommand{\mfb}{\mf b}
\newcommand{\mfc}{\mf c}
\newcommand{\mfd}{\mf d}
\newcommand{\mfe}{\mf e}
\newcommand{\mff}{\mf f}
\newcommand{\mfg}{\mf g}
\newcommand{\mfh}{\mf h}
\newcommand{\mfi}{\mf i}
\newcommand{\mfj}{\mf j}
\newcommand{\mfk}{\mf k}
\newcommand{\mfl}{\mf l}
\newcommand{\mfm}{\mf m}
\newcommand{\mfn}{\mf n}
\newcommand{\mfo}{\mf o}
\newcommand{\mfp}{\mf p}
\newcommand{\mfq}{\mf q}
\newcommand{\mfr}{\mf r}
\newcommand{\mfs}{\mf s}
\newcommand{\mft}{\mf t}
\newcommand{\mfu}{\mf u}
\newcommand{\mfv}{\mf v}
\newcommand{\mfw}{\mf w}
\newcommand{\mfx}{\mf x}
\newcommand{\mfy}{\mf y}
\newcommand{\mfz}{\mf z}
\newcommand{\mbfA}{\mathbf A}
\newcommand{\mbfB}{\mathbf B}
\newcommand{\mbfC}{\mathbf C}
\newcommand{\mbfD}{\mathbf D}
\newcommand{\mbfE}{\mathbf E}
\newcommand{\mbfF}{\mathbf F}
\newcommand{\mbfG}{\mathbf G}
\newcommand{\mbfH}{\mathbf H}
\newcommand{\mbfI}{\mathbf I}
\newcommand{\mbfJ}{\mathbf J}
\newcommand{\mbfK}{\mathbf K}
\newcommand{\mbfL}{\mathbf L}
\newcommand{\mbfM}{\mathbf M}
\newcommand{\mbfN}{\mathbf N}
\newcommand{\mbfO}{\mathbf O}
\newcommand{\mbfP}{\mathbf P}
\newcommand{\mbfQ}{\mathbf Q}
\newcommand{\mbfR}{\mathbf R}
\newcommand{\mbfS}{\mathbf S}
\newcommand{\mbfT}{\mathbf T}
\newcommand{\mbfU}{\mathbf U}
\newcommand{\mbfV}{\mathbf V}
\newcommand{\mbfW}{\mathbf W}
\newcommand{\mbfX}{\mathbf X}
\newcommand{\mbfY}{\mathbf Y}
\newcommand{\mbfZ}{\mathbf Z}

% common math symbols

\newcommand{\bs}{\backslash}
\newcommand{\bbRn}{\mathbb R^n}
\newcommand{\bbCn}{\mathbb R^n}
\newcommand{\ecyc}[1]{\langle #1\rangle}
\newcommand{\interior}[1]{\overset{\circ}{#1}}
\newcommand{\ol}[1]{\overline{#1}}
\newcommand{\unitgrp}[1]{#1^{\mspace{-4mu}\times}}
\newcommand{\grad}{\triangledown}
\newcommand{\normal}{\trianglelefteq}
\newcommand{\nnormal}{\ntrianglelefteq}
\newcommand{\norm}[1]{\left\lVert#1\right\rVert}
\newcommand{\gen}[1]{\langle #1 \rangle}
\newcommand{\nilradical}[1]{\sqrt{\gen{0}}_{#1}}
\newcommand{\inv}[1]{#1^{\text{-}1}}
\newcommand{\id}{\mathrm{id}}
\newcommand{\unitcell}[1]{\overset{\!\!\!\! \circ}{\bbD^{#1}}}
\newcommand{\E}{\exists}
\newcommand{\A}{\forall}
\newcommand{\Fct}{\mathrm{Funct}}

% Operators

\DeclareMathOperator{\GL}{\rm{GL}}
\DeclareMathOperator{\SL}{\rm{SL}}
\DeclareMathOperator{\Mod}{\rm{Mod}}
\DeclareMathOperator{\Mor}{\rm{Mor}}
\DeclareMathOperator{\Obj}{\rm{Obj}}
\DeclareMathOperator{\Id}{\rm{Id}}
\DeclareMathOperator{\End}{\rm{End}}
\DeclareMathOperator{\Hom}{\rm{Hom}}
\DeclareMathOperator{\Res}{\rm{Res}}
\DeclareMathOperator{\Spec}{\rm{Spec}}
\DeclareMathOperator{\Proj}{\rm{Proj}}
\DeclareMathOperator{\Supp}{\rm{Supp}}
\DeclareMathOperator{\Ker}{\rm{Ker}}
\DeclareMathOperator{\Nil}{\rm{Nil}}
\DeclareMathOperator{\sh}{\rm{sh}}
\DeclareMathOperator{\Coker}{\rm{Coker}}
\DeclareMathOperator{\Rank}{\rm{Rank}}
\DeclareMathOperator{\Frac}{\rm{Frac}}
\DeclareMathOperator{\Rad}{\rm{Rad}}
\DeclareMathOperator{\Ann}{\rm{Ann}}
\DeclareMathOperator{\Disc}{\rm{Disc}}
\DeclareMathOperator{\Lcm}{\rm{lcm}}
\DeclareMathOperator{\Gcd}{\rm{gcd}}
\DeclareMathOperator{\Conv}{Conv}
\DeclareMathOperator{\Cone}{Cone}
\DeclareMathOperator{\Int}{Int}
\DeclareMathOperator{\Ord}{ord}
\DeclareMathOperator{\Ass}{Ass}
\DeclareMathOperator{\Aut}{Aut}
\DeclareMathOperator{\Sym}{Sym}
\DeclareMathOperator{\Char}{Char}
\DeclareMathOperator{\Span}{Span}
\DeclareMathOperator{\Tor}{Tor}
\DeclareMathOperator{\Ext}{Ext}
\DeclareMathOperator{\Card}{Card}
\DeclareMathOperator{\OPT}{OPT}
\DeclareMathOperator{\Dom}{Dom}
\DeclareMathOperator{\Var}{Var}
\DeclareMathOperator{\Th}{Th}
\DeclareMathOperator{\Frob}{Frob}
\DeclareMathOperator{\Red}{Red}
\DeclareMathOperator{\Aff}{Aff}
\DeclareMathOperator{\Epi}{Epi}
\DeclareMathOperator{\sech}{sech}
\DeclareMathOperator{\csch}{csch}
\DeclareMathOperator*{\Argmin}{Argmin}
\DeclareMathOperator{\Zer}{Zer}
\DeclareMathOperator{\Ri}{Ri}
\DeclareMathOperator{\Prox}{Prox}
\DeclareMathOperator{\sgn}{sgn}
\DeclareMathOperator{\Fix}{Fix}
\DeclareMathOperator{\Gal}{Gal}
\renewcommand{\Re}{\operatorname{Re}}
\renewcommand{\Im}{\operatorname{Im}}
\DeclareMathOperator{\Adj}{Adj}
\renewcommand{\div}{\mathrm{div}}
\DeclareMathOperator{\Div}{Div}
\DeclareMathOperator{\Cl}{Cl}
\DeclareMathOperator{\CDiv}{CDiv}
\DeclareMathOperator{\Bl}{Bl}
\DeclareMathOperator{\codim}{codim}
\DeclareMathOperator{\Sing}{Sing}
\DeclareMathOperator{\Nef}{Nef}
\DeclareMathOperator{\NE}{NE}
\DeclareMathOperator{\Mult}{Mult}
\DeclareMathOperator{\Pic}{Pic}
\DeclareMathOperator{\Tr}{Tr}
\DeclareMathOperator{\Grass}{Grass}
\DeclareMathOperator{\LinSub}{LinSub}
\DeclareMathOperator{\Eq}{Eq}
\DeclareMathOperator{\can}{can}
\DeclareMathOperator{\Rat}{Rat}
\DeclareMathOperator{\trdeg}{trdeg}
\DeclareMathOperator{\QSym}{QSym}
\DeclareMathOperator{\ASym}{ASym}
\DeclareMathOperator{\des}{des}
\DeclareMathOperator{\exc}{exc}
\DeclareMathOperator{\maj}{maj}
\DeclareMathOperator{\wt}{wt}
\DeclareMathOperator{\SST}{SST}
\DeclareMathOperator{\ASST}{\mfA SST}
\DeclareMathOperator{\SSYT}{SSYT}
\DeclareMathOperator{\ST}{ST}
\DeclareMathOperator{\SYT}{SYT}
\DeclareMathOperator{\SV}{SV}
\DeclareMathOperator{\ex}{ex}
\DeclareMathOperator{\sort}{sort}
\DeclareMathOperator{\sq}{sq}
\DeclareMathOperator{\invcode}{invcode}
\DeclareMathOperator{\Ess}{Ess}
\DeclareMathOperator{\Flag}{Flag}
\DeclareMathOperator{\Stab}{Stab}
\DeclareMathOperator{\Orb}{Orb}
\DeclareMathOperator{\pl}{pl}
\DeclareMathOperator{\Mat}{Mat}
\DeclareMathOperator{\pt}{pt}
\DeclareMathOperator{\depth}{depth}
\DeclareMathOperator{\cyc}{cyc}
\DeclareMathOperator{\ev}{ev}
\DeclareMathOperator{\length}{lg}
\DeclareMathOperator{\Quot}{Quot}
\DeclareMathOperator{\Hilb}{Hilb}
\DeclareMathOperator{\PGL}{PGL}
\DeclareMathOperator{\PSL}{PSL}
\newcommand{\Cat}[1]{(\textrm{#1})}

% above : global newcommands and global operators
% 
% below : local newcommands amd local operators

\newcommand{\Cof}{\mathrm{Cof}}
\newcommand{\Fib}{\mathrm{Fib}}

\newcommand{\phead}{\hspace*{0.46cm}}
\DeclareMathOperator{\Homtop}{\rm{Hom}_{\mbf{Top}}}

\begin{document}
    \title{Homotopy Theory}
    \author{{\color{pink}{Cloudi}}{\color{Aquamarine}{fold}}}
    \maketitle
    \newpage

    \section{Notations}

    \begin{align*}
        & \text{Category of sets } &: \ & \mbf{Set} \\
        & \text{Category of topological spaces }&: \ & \mbf{Top} \\
        & \text{Category of (one-point-)based topological spaces }&: \  & \mbf{Top}_* \\
        & \text{Category of pairs $(X,A)$ of space $X$ and subspace $A$ } &: \ & \mbf{Top}(2) \\
        & \text{Topological space $X$ with topology $\mcT$ } &: \ & X_{\mcT}\\
        & \text{Euclidean space of dimension $n$ } &: \ & \bbRn\\
        & \text{Unit cube of dimension $n$ } &: \ & I^n\\
        & \text{Boundary of $I^n$ }          &: \ & \partial I^n\\
        & \text{Unit interval $I$}           &: \ & I = I^1\\
        & \text{Unit cell of dimension $n$ } &: \ & \unitcell{n} \\
        & \text{Unit disk of dimension $n$ } &: \ & \bbD^n\\
        & \text{Unit sphere of dimension $n-1$ } &: \ & \bbS^{n-1}\\
        & \text{Inclusion or Embedding } &: \ & \hookrightarrow \\
        & \text{Monomorphsim }      &: \ & \rightarrowtail \\
        & \text{Epimorphsim }       &: \ & \twoheadrightarrow \\
        & \text{Hom functor of category $\mcC $} &: \ & \Hom_{\mcC}(-,-)\\
        & \text{Limit (inverse limit) (projective limit) } &: \ & \lim_{\leftarrow} \\
        & \text{Colimit (direct limit) (inductive limit) } &: \ & \lim_{\rightarrow} \\
    \end{align*}

    \newpage

    \section{Preliminary Definitions}

    \begin{defn}
        A object $A$ is called a $\mbf{retract}$ of $B$ if there are
        morphisms $s : A \to B, r : B \to A$ such that $r \circ s = \id_A$.
        In this case,
        $r$ is called a $\mbf{retraction}$ of $s$ and
        $s$ is called a $section$ of $r$.
        $$ \id_A : A \xrightarrow[section]{s} B \xrightarrow[retraction]{r} A$$
    \end{defn}

    \begin{prop}
        For any category $\mcC$, $\Iso(\mcC)$ is closed under forming retracts in $\Arr(\mcC)$.
    \end{prop}
    \begin{prf}
        \par For $f \in \Iso(\mcC)$, $g$ is a retract of $f$ in $\Arr(\mcC)$, $\inv{g} = b \circ \inv{f} \circ a$
        \[\begin{tikzcd}
            \bullet & \bullet & \bullet \\
            \bullet & \bullet & \bullet
            \arrow["a"', from=2-1, to=2-2]
            \arrow[from=1-1, to=1-2]
            \arrow["f"', curve={height=6pt}, from=1-2, to=2-2]
            \arrow["g"', from=1-1, to=2-1]
            \arrow["{\inv{f}}"', curve={height=6pt}, from=2-2, to=1-2]
            \arrow["b", from=1-2, to=1-3]
            \arrow[from=2-2, to=2-3]
            \arrow["g", from=1-3, to=2-3]
        \end{tikzcd}\]
        \qed
    \end{prf}

    \newpage

    \section{Abstract Homotopy Theory}

    \subsection{Introduction}

    In generality, homotopy theory is the study of mathematical contexts in
    which morphisms are equipped with a concept of homotopy between them,
    hence with a concept of “equivalent deformations” of morphisms,
    and then iteratively with homotopies of homotopies between those, and so forth.

    \par A fundamental insight due to (Quillen 67) is that in fact
    all constructions in homotopy theory are elegantly expressible
    via just the abstract interplay of these classes of morphisms.
    This was distilled in (Quillen 67) into
    a small set of axioms called a $\mbf{model\ category}$ structure
    (because it serves to make all objects be models for homotopy types.)

    \par This abstract homotopy theory is the royal road for
    handling any flavor of homotopy theory,
    in particular the stable homotopy theory.
    Here we discuss the basic constructions and facts in abstract homotopy theory,
    then below we conclude this Introduction to Homotopy Theory by
    showing that topological spaces equipped with the above system of
    classes continuous functions is indeed an example of
    abstract homotopy theory in this sense.


    \subsection{Factorization Systems}

    \begin{defn}
        A $\mbf{category\ with\ weak\ equivalences}$ is
        \begin{enumerate}
            \item A category $\mcC$;
            \item A sub-class $W \subseteq \Mor(\mcC)$;
        \end{enumerate}
        \ \\
        such that
        \begin{enumerate}
            \item $W$ contains all isomorphisms in $\mcC$
            \item $W$ is closed under $\mbf{two\text{-}out\text{-} of\text{-} three}$ :
            \par In every commutative diagram in $\mcC$ of the form
            \begin{tikzcd}
                & Y \\
                X && Z
                \arrow[from=2-1, to=2-3]
                \arrow[from=2-1, to=1-2]
                \arrow[from=1-2, to=2-3]
            \end{tikzcd}

            if two of the three morphisms are in $W$, then so is the third.

        \end{enumerate}

    \end{defn}

    \begin{note}
        The further axioms of a $\mbf{model\ category}$ serve the sole purpose of
        making the universal homotopy theory induced by
        a category with weak equivalences be tractable.

    \end{note}

    \begin{defn}
        A $\mbf{model\ category}$ is
        \begin{enumerate}
            \item A category $\mcC$ with limits and colimits;
            \item Three sub-classes $W$, $\Fib$, $\Cof$ $ \subseteq \Mor(\mcC)$ of morphisms in $\mcC$;
        \end{enumerate}
        \ \\
        such that
        \begin{enumerate}
            \item $(\mcC, W)$ is a $\mbf{category\ with\ weak\ equivalences}$;
            \item Pairs $(W \cap \Cof, \Fib)$ and $(\Cof, W \cap \Fib)$ are both
            $\mbf{weak\ factorization\ systems}$.
        \end{enumerate}

    \end{defn}

    \begin{note}
        We say:
        \begin{itemize}
            \item elements in $W$ are $\mbf{weak\ equivalences}$;
            \item elements in $\Fib$ are $\mbf{fibrations}$;
            \item elements in $\Cof$ are $\mbf{cofibrations}$;
            \item elements in $W \cap \Fib$ are $\mbf{acyclic(trivial)\ fibrations}$;
            \item elements in $W \cap \Cof$ are $\mbf{acyclic(trivial)\ cofibrations}$;
        \end{itemize}

    \end{note}

    \begin{defn}
        $\mbf{extension, lift\ and\ lifting}$ : 
        \par We assert that diagrams, morphisms and objects below are in a category $\mcC$.
        \par Given diagram of the form
        \begin{tikzcd}
            X & Y \\
            Z
            \arrow["p"', from=1-1, to=2-1]
            \arrow["f", from=1-1, to=1-2]
        \end{tikzcd}
        , the $\mbf{extension}$ of $f$ along $p$ is a morphism $\ol{f} : Z \to Y$
        such that the diagram
        \begin{tikzcd}
            X & Y \\
            Z
            \arrow["p"', from=1-1, to=2-1]
            \arrow["f", from=1-1, to=1-2]
            \arrow["{\overline{f}}"', from=2-1, to=1-2]
        \end{tikzcd}
        commutes.

        \par Dually, given a diagram of the form
        \begin{tikzcd}
            & Z \\
            X & Y
            \arrow["f", from=2-1, to=2-2]
            \arrow["p", from=1-2, to=2-2]
        \end{tikzcd}
        , the $\mbf{lift}$ of $f$ through $p$ is a morphism $\ol{f} : X \to Z$ such that
        the diagram
        \begin{tikzcd}
            & Z \\
            X & Y
            \arrow["f", from=2-1, to=2-2]
            \arrow["p", from=1-2, to=2-2]
            \arrow["{\overline{f}}", from=2-1, to=1-2]
        \end{tikzcd}
        commutes.

        \par Combining the two cases, given a commutative square $\msC$ :
        \begin{tikzcd}
            X & Z \\
            Y & W
            \arrow["{f_b}", from=2-1, to=2-2]
            \arrow["{p_r}", from=1-2, to=2-2]
            \arrow["{p_l}"', from=1-1, to=2-1]
            \arrow["{f_t}", from=1-1, to=1-2]
        \end{tikzcd}
        the $\mbf{lifting}$ in $\msC$ is a morphism $\ol{f} : Y \to Z$
        such that the diagram
        \begin{tikzcd}
            X & Z \\
            Y & W
            \arrow["{f_b}", from=2-1, to=2-2]
            \arrow["{p_r}", from=1-2, to=2-2]
            \arrow["{\overline{f}}", from=2-1, to=1-2]
            \arrow["{p_l}"', from=1-1, to=2-1]
            \arrow["{f_t}", from=1-1, to=1-2]
        \end{tikzcd}
        commutes.

    \end{defn}

    \begin{defn}
        $\mbf{right\ and\ left\ lifting\ property}$
        \par Diagrams, morphisms and objects below are in a category $\mcC$.
        \par With a given sub-class of morphisms $K \subseteq \Mor(\mcC)$ :
        \\
        \par A morphism $p_r$ is said to have the
        $\mbf{right\ lifting\ property}$ against $K$ or to be
        a $K\mbf{\text{-}injective\ morphism}$
        if in all commutative diagrams
        \begin{tikzcd}
            X & Z \\
            Y & W
            \arrow["{f_b}", from=2-1, to=2-2]
            \arrow["{p_r}", from=1-2, to=2-2]
            \arrow["{p_l}"', from=1-1, to=2-1]
            \arrow["{f_t}", from=1-1, to=1-2]
        \end{tikzcd}
        with $p_r$ on the right and any $p_l \in K$ on the left,
        the $\mbf{lifting}$ in the diagram exists.
        \\
        \par Dually, A morphism $p_l$ is said to have the
        $\mbf{left\ lifting\ property}$ against $K$ or to be
        a $K\mbf{\text{-}projective\ morphism}$
        if in all commutative diagrams
        \begin{tikzcd}
            X & Z \\
            Y & W
            \arrow["{f_b}", from=2-1, to=2-2]
            \arrow["{p_r}", from=1-2, to=2-2]
            \arrow["{p_l}"', from=1-1, to=2-1]
            \arrow["{f_t}", from=1-1, to=1-2]
        \end{tikzcd}
        with $p_l$ on the left and any $p_r \in K$ on the right,
        the $\mbf{lifting}$ in the diagram exists.
    \end{defn}

    \begin{defn}
        A $\mbf{weak\ factorization\ system}$ (WFS) on a category $\mcC$ is
        a pair $(\Proj, \Inj)$ of two classes of morphisms of $\mcC$ such that

        \begin{enumerate}
            \item Every morphism $f : X \to Y$ in $\mcC$ can be factored as
            \begin{tikzcd}
                X & Z & Y
                \arrow["{p \in \Proj}", from=1-1, to=1-2]
                \arrow["{i \in \Inj}", from=1-2, to=1-3]
            \end{tikzcd}
            \item The classes are closed under having the $\mbf{lifting\ property}$ against each other.
            That is:
            \begin{enumerate}
                \item $\Proj$ is \emph{precisely} the class of morphisms having the \textbf{left lifting property} against $\Inj$.
                \item $\Inj$ is \emph{precisely} the class of morphisms having the $\mbf{right\ lifting\ property}$ against $\Proj$.
            \end{enumerate}
        \end{enumerate}

    \end{defn}

    \begin{rem}
        The factorization just ensured \emph{existence}.
    \end{rem}

    \begin{defn}
        For $\mcC$ a category, a $\mbf{functorial\ factorization}$ of the morphisms in $\mcC$ is a functor
        $ fact : \mcC^{\Delta[1]} \to \mcC^{\Delta[2]} $, which is the right inverse of
        the composition functor $d_1 : \mcC^{\Delta[2]} \to \mcC^{\Delta[1]} $. Equalently,
        $ d_1 \circ fact = \id_{\mcC^{\Delta[1]}} $
    \end{defn}
    
    \begin{rem}
        The notation above is defined at $\mbf{simplex\ category}$ and at $\mbf{nerve}$ of a category.
        The $\mbf{arrow\ category}$ of a category $\mcC$ is
        $\Arr(\mcC) := \mcC^{\Delta[1]} := \Fct(\Delta[1], \mcC)$ whose objects are morphisms in $\mcC$.
        $\mcC^{\Delta[2]} := \Fct(\Delta[2], \mcC)$, its objects are pairs of two composable morphisms in $\mcC$.
    \end{rem}

    \begin{defn} 
        A $weak\ factorization\ system$ is a $\mbf{functorial\ weak\ factorization\ system}$
        if the factorization of morphisms can be $chosen$ to be a $functorial\ factorization\ fact : \mcC^{\Delta[1]} \to \mcC^{\Delta[2]}$,
        such that $(d_2 \circ fact) (\Obj(\mcC^{\Delta[1]})) \subseteq \Proj$ and $(d_0 \circ fact) (\Obj(\mcC^{\Delta[1]})) \subseteq \Inj$.
        Where
        \begin{align*}
            d_2, d_0 &: &\mcC^{\Delta[2]} &\ \longrightarrow \ \mcC^{\Delta[1]}\\
            d_2 &: &\ast \overset{f}{\to} \ast \overset{s}{\to} \ast
                &\ \longmapsto \ \ast \overset{f}{\to} \ast\\
            d_0 &: &\ast \overset{f}{\to} \ast \overset{s}{\to} \ast
                &\ \longmapsto \ \ast \overset{s}{\to} \ast
        \end{align*}

    \end{defn}

    \begin{note}
        Not all $weak\ factorization\ system$ is $functorial$,
        although most (including those produced by the $\mbf{small\ object\ argument}$) are.
    \end{note}

    \begin{prop}
        Let $\mcC$ be a category, $K \subseteq \Mor(\mcC)$ a class of morphisms in $\mcC$.
        Write $K \Proj$, $K \Inj$ respectively for
        the sub-classes of $K$-projective morphisms
        and of $K$-injective morphisms.
        We have properties of $K \Proj$ and $K \Inj$ below:
        \begin{enumerate}
            \item Both contain $\Iso(\mcC)$.
            \item Both are closed under composition in $\mcC$.\\
                And $K \Proj$ is closed under $\mbf{transfinite\ composition}$.
            \item Both are closed under forming retracts in the arrow category $\Arr(\mcC)$.
            \item $K \Proj$ is closed under pushouts in $\mcC$ ("cobase change").\\
                $K \Inj$ is closed under pullbacks in $\mcC$ ("base change").
            \item $K \Proj$ is closed under coproducts in $\Arr(\mcC)$.\\
                $K \Inj$ is closed under products in $\Arr(\mcC)$.
        \end{enumerate}
    \end{prop}

    \begin{prf}
        \ 
        \par $\mbf{1.\ Containing\ isomorphisms}$\\

            Trivial for $K \Inj$:
            \begin{tikzcd}[row sep = 1.5cm, column sep = 1.5cm]
                \bullet & \bullet \\
                \bullet & \bullet
                \arrow["f", from=1-1, to=1-2]
                \arrow["{p \in K}", from=1-2, to=2-2]
                \arrow["g"', from=2-1, to=2-2]
                \arrow["{i \in \Iso}"', from=1-1, to=2-1]
                \arrow["{f \circ \inv{i}}"{marking}, dotted, from=2-1, to=1-2]
            \end{tikzcd}
            And dual is also trivial.  \\

        \par $\mbf{2.\ Clousure\ under\ composition}$\\

            \par For $K \Inj$:

            \[\begin{tikzcd}[row sep = 0.75cm, column sep = 0.95cm]
                \bullet && X \\
                && Z \\
                \bullet && Y
                \arrow["{k \in K}"', from=1-1, to=3-1]
                \arrow["g"', from=3-1, to=3-3]
                \arrow["f", from=1-1, to=1-3]
                \arrow["{p_1 \in K \Inj}", from=1-3, to=2-3]
                \arrow["{p_2\in K \Inj}", from=2-3, to=3-3]
            \end{tikzcd}\]

            \par Compose $f, p_1$:

            \[\begin{tikzcd}
                \bullet & Z \\
                \bullet & Y
                \arrow["{p_1 \circ f}", from=1-1, to=1-2]
                \arrow["{p_2 \in K \Inj}", from=1-2, to=2-2]
                \arrow["g"', from=2-1, to=2-2]
                \arrow["{k \in K}"', from=1-1, to=2-1]
                \arrow["{\E \ l_1}"{marking}, dotted, from=2-1, to=1-2]
            \end{tikzcd}\]

            \par Then we have:

            \[\begin{tikzcd}
                \bullet & X \\
                \bullet & Z
                \arrow["{ f}", from=1-1, to=1-2]
                \arrow["{p_1 \in K \Inj}", from=1-2, to=2-2]
                \arrow["{l_1}"', from=2-1, to=2-2]
                \arrow["{k \in K}"', from=1-1, to=2-1]
                \arrow["\E \ l_2"{marking}, dotted, from=2-1, to=1-2]
            \end{tikzcd}\]

            \par The $l_2$ is the expected lifting of the diagram:

            \[\begin{tikzcd}[row sep = 0.75cm, column sep = 0.95cm]
                \bullet && X \\
                && Z \\
                \bullet && Y
                \arrow["{k \in K}"', from=1-1, to=3-1]
                \arrow["g"', from=3-1, to=3-3]
                \arrow["f", from=1-1, to=1-3]
                \arrow["{p_1 \in K \Inj}", from=1-3, to=2-3]
                \arrow["{p_2\in K \Inj}", from=2-3, to=3-3]
                \arrow["{\E\ l_1}"{marking}, dotted, from=3-1, to=2-3]
                \arrow["{\E\ l_2}"{marking}, dotted, from=3-1, to=1-3]
            \end{tikzcd}\]

            \par For finite composition of morphisms in $K \Proj$, we have the dual forms of above.
            The clousure under $transfinite\ composition$ follows since it is given by
            $colimits$ of $sequential$ composition.
            And $successive$ liftings as below constitutes a cocone.
            The extension of those liftings to the colimit is just the expected lifting.

            \[\begin{tikzcd}
                \bullet && \bullet \\
                \bullet \\
                \bullet \\
                \bullet \\
                \bullet && \bullet
                \arrow[from=5-1, to=5-3]
                \arrow[from=1-3, to=5-3]
                \arrow[from=1-1, to=2-1]
                \arrow[from=2-1, to=3-1]
                \arrow["\cdots"{marking}, draw=none, from=4-1, to=5-1]
                \arrow[from=1-1, to=1-3]
                \arrow[from=3-1, to=5-3]
                \arrow[from=2-1, to=5-3]
                \arrow[dotted, from=2-1, to=1-3]
                \arrow[dotted, from=3-1, to=1-3, crossing over]
                \arrow[from=3-1, to=4-1]
                \arrow[dotted, from=4-1, to=1-3, crossing over]
                \arrow[from=4-1, to=5-3]
                \arrow["{\underset{\to}{\lim}}"{marking}, dotted, from=5-1, to=1-3, crossing over]
            \end{tikzcd}\]
        
        \par $\mbf{3.\ Closure\ under\ retracts}$\\

            \par Let $j$ be the retract in $\Arr(\mcC)$  of $i \in K \Proj$

            \[\begin{tikzcd}
                A & X & A \\
                B & Y & B
                \arrow["{i\in K \Proj}", from=1-2, to=2-2]
                \arrow[from=1-2, to=1-3]
                \arrow[from=2-2, to=2-3]
                \arrow["\id_A : \hspace*{2.77cm}"{marking}, from=1-1, to=1-2]
                \arrow["\id_B : \hspace*{2.77cm}"{marking}, from=2-1, to=2-2]
                \arrow["j", from=1-1, to=2-1]
                \arrow["j", from=1-3, to=2-3]
            \end{tikzcd}\]

            Then for a commutative square (named $\msC$):

            \[\begin{tikzcd}
                A & \bullet \\
                B & \bullet
                \arrow["{k\in K}", from=1-2, to=2-2]
                \arrow[from=1-1, to=1-2]
                \arrow[from=2-1, to=2-2]
                \arrow["j", from=1-1, to=2-1]
            \end{tikzcd}\]

            It is equalent to its $pasting\ composite$ with the retract diagram above:

            \[\begin{tikzcd}
                A & X & A & \bullet \\
                B & Y & B & \bullet
                \arrow["{k\in K}", from=1-4, to=2-4]
                \arrow[from=1-3, to=1-4]
                \arrow[from=2-3, to=2-4]
                \arrow["j", from=1-3, to=2-3]
                \arrow[from=1-2, to=1-3]
                \arrow[from=2-2, to=2-3]
                \arrow["{i \in K \Proj}", from=1-2, to=2-2]
                \arrow[from=1-1, to=1-2]
                \arrow["j", from=1-1, to=2-1]
                \arrow["s_2",from=2-1, to=2-2]
            \end{tikzcd}\]

            The expected lifting of $\msC$ is just $l_1 \circ s_2$:

            \[\begin{tikzcd}
                A & X & A & \bullet \\
                B & Y & B & \bullet
                \arrow["{k\in K}", from=1-4, to=2-4]
                \arrow[from=1-3, to=1-4]
                \arrow[from=2-3, to=2-4]
                \arrow[from=1-2, to=1-3]
                \arrow[from=2-2, to=2-3]
                \arrow["i", from=1-2, to=2-2]
                \arrow[from=1-1, to=1-2]
                \arrow["j", from=1-1, to=2-1]
                \arrow["{s_2}", from=2-1, to=2-2]
                \arrow["{\E\ l_1}"{marking}, dotted, from=2-2, to=1-4]
            \end{tikzcd}\]

        \par $\mbf{4.\ Closure\ under\ pullbacks\ and\ pushouts}$\\

            \par For all $p \in K \Inj$, and $f^* p$ the $base\ change$ of $p$ along some $f$
            we have the pullback diagram $(1)$:

            \[\begin{tikzcd}
                P & X \\
                Y & Z
                \arrow["f"', from=2-1, to=2-2]
                \arrow["p", from=1-2, to=2-2]
                \arrow["{f^*p}"', from=1-1, to=2-1]
                \arrow[from=1-1, to=1-2]
                \arrow["{\big \lrcorner}"{anchor=center, pos=0.125}, draw=none, from=1-1, to=2-2]
            \end{tikzcd}\]

            \par Our goal is to show that $f^* p$ has the $right\ lifting\ property$ against $K$.
            So, we consider $(2)$:

            \[\begin{tikzcd}
                A & P & X \\
                B & Y & Z
                \arrow["f"', from=2-2, to=2-3]
                \arrow["{p \in K \Inj}", from=1-3, to=2-3]
                \arrow["{f^*p}"', from=1-2, to=2-2]
                \arrow["p^*f", from=1-2, to=1-3]
                \arrow["{\big \lrcorner}"{anchor=center, pos=0.125}, draw=none, from=1-2, to=2-3]
                \arrow["g"',from=2-1, to=2-2]
                \arrow["w", from=1-1, to=1-2]
                \arrow["{{i \in K}}"', from=1-1, to=2-1]
            \end{tikzcd}\]

            \par Because of $p$ has the $right\ lifting\ property$ against $K$ $(3)$:

            \[\begin{tikzcd}
                A & P & X \\
                B & Y & Z
                \arrow["f"', from=2-2, to=2-3]
                \arrow["{p \in K Inj}", from=1-3, to=2-3]
                \arrow["p^*f", from=1-2, to=1-3]
                \arrow["g"', from=2-1, to=2-2]
                \arrow["w", from=1-1, to=1-2]
                \arrow["{{i \in K}}"', from=1-1, to=2-1]
                \arrow["{\E \ l_{fg}}"{marking}, dotted, from=2-1, to=1-3]
            \end{tikzcd}\]

            \par By the $universal\ property$ of pullback $(4)$:

            \[\begin{tikzcd}
                & P & X \\
                B & Y & Z
                \arrow["f"', from=2-2, to=2-3]
                \arrow["p", from=1-3, to=2-3]
                \arrow[from=1-2, to=2-2]
                \arrow["p^*f", from=1-2, to=1-3]
                \arrow["g"', from=2-1, to=2-2]
                \arrow["{\ \ \! l_{fg}}"', from=2-1, to=1-3, crossing over]
                \arrow["{\E \ l_g}"{marking}, dotted, from=2-1, to=1-2]
                \arrow["{\big \lrcorner}"{anchor=center, pos=0.125}, draw=none, from=1-2, to=2-3]
            \end{tikzcd}\]

            \par It remains to show the commutativity of that diagram:

            \[\begin{tikzcd}
                A & P \\
                B
                \arrow["{l_{g}}"', from=2-1, to=1-2]
                \arrow["{i}"', from=1-1, to=2-1]
                \arrow["w", from=1-1, to=1-2]
            \end{tikzcd}\]

            \par By the universal property of limits :
            $\Hom(-, \lim\limits_{\leftarrow} D) \overset{c}{\cong} \Cone(-, D)$.
            To prove $l_g \circ i = w$, is equalent to prove $c(l_g \circ i) = c(w)$.

            \[\begin{tikzcd}
                A & P & X \\
                B & Y & Z
                \arrow["{l_{g}}", from=2-1, to=1-2]
                \arrow["{i }"', from=1-1, to=2-1]
                \arrow["w", from=1-1, to=1-2]
                \arrow[from=1-2, to=2-2]
                \arrow["{p^*f}", from=1-2, to=1-3]
                \arrow["p", from=1-3, to=2-3]
                \arrow["f"', from=2-2, to=2-3]
                \arrow["g"', from=2-1, to=2-2]
                \arrow["{\ \ \! l_{fg}}"', from=2-1, to=1-3, crossing over]
                \arrow["{\big \lrcorner}"{anchor=center, pos=0.125}, draw=none, from=1-2, to=2-3]
            \end{tikzcd}\]

            \par Following the diagram above (we $have\ not$ said that the diagrams above is commutative before):
            \begin{align*}
                c(w) &= (p^*f \circ w, f^*p \circ w) & by\ def\\
                c(l_g \circ i) &= (p^*f \circ l_g \circ i,\ f^*p \circ l_g \circ i) & by\ def\\
                &= (l_{fg} \circ i,\ g \circ i)  & by\ diagram\ (4)\\
                &= (p^*f \circ w,\ f^*p \circ w) = c(w) & by\ diagram\ (3)\ and\ (2)\\
            \end{align*}

            \par The other case is formally dual.\\

        \par $\mbf{5.\ Closure\ under\ products\ and\ coproducts}$\\

            \par It is easy to prove that the product $\prod\limits_{s \in S} p_s$ of
            a family of arrows(objects) $\{p_s : A_s \to B_s\}_{s \in S}$ in the arrow category
            $\Arr(\mcC)$ is just the unique morphism determined by
            the morphisms $\{ p_s \circ pr_{A_s} \}_{s \in S}$ out of $\prod\limits_{s \in S} A_s$
            and the  $universal\ property$ of products $\prod\limits_{s \in S} B_s$.
            The $s$-th projection of $\prod\limits_{s \in S} p_s$ in $\Arr(\mcC)$ is just $(pr_{A_s}, pr_{B_s})$:

            \[\begin{tikzcd}
                {\prod\limits_{s \in S} A_s} & {A_s} \\
                {\prod\limits_{s \in S} B_s} & {B_s}
                \arrow["{\prod\limits_{s\in S}p_s}"', from=1-1, to=2-1]
                \arrow["{pr_{A_s}}", from=1-1, to=1-2]
                \arrow["{pr_{B_s}}"', from=2-1, to=2-2]
                \arrow["{p_s}", from=1-2, to=2-2]
            \end{tikzcd}\]

            With all $p_s \in K \Inj$, Let the diagram below commute:

            \[\begin{tikzcd}[column sep = 1.15cm]
                X & {\prod\limits_{s \in S} A_s} \\
                Y & {\prod\limits_{s \in S} B_s}
                \arrow["{\prod\limits_{s\in S}p_s}", from=1-2, to=2-2]
                \arrow["x", from=1-1, to=1-2]
                \arrow["y"', from=2-1, to=2-2]
                \arrow["{f \in K}"', from=1-1, to=2-1]
            \end{tikzcd}\]

            \par Our goal is to get a $lifting$ of the diagram above.
            By the universal property of the product. We have a family of commutative diagrams:

            \[ \left\{
            \begin{tikzcd}
                X & {A_s} \\
                Y & {B_s}
                \arrow["{f \in K}"', from=1-1, to=2-1]
                \arrow["{p_s}", from=1-2, to=2-2]
                \arrow["{pr_{B_s} \circ y}"', from=2-1, to=2-2]
                \arrow["{pr_{A_s} \circ x}", from=1-1, to=1-2]
            \end{tikzcd}
            \right\}_{s \in S} \]

            By all $p_s \in K \Inj$, we have a family of liftings:

            \[ \left\{
            \begin{tikzcd}
                X & {A_s} \\
                Y & {B_s}
                \arrow["{f \in K}"', from=1-1, to=2-1]
                \arrow["{p_s}", from=1-2, to=2-2]
                \arrow["{pr_{B_s} \circ y}"', from=2-1, to=2-2]
                \arrow["{pr_{A_s} \circ x}", from=1-1, to=1-2]
                \arrow["{\E\ l_s}"{marking}, dotted, from=2-1, to=1-2]
            \end{tikzcd}
            \right\}_{s \in S} \]

            \par Therefore, there is a unique morphism $\prod\limits_{s \in S} l_s : Y \to \prod\limits_{s \in S} A_s$
            determined by the universal property of product $\prod\limits_{s \in S} A_s$

            \[\begin{tikzcd}[row sep = 1.6cm, column sep = 1.6cm]
                X & {\prod\limits_{s \in S} A_s} & {A_s} \\
                Y & {\prod\limits_{s \in S} B_s} & {B_s}
                \arrow["{pr_{A_s}}", from=1-2, to=1-3]
                \arrow["{pr_{B_s}}"', from=2-2, to=2-3]
                \arrow["{p_s}", from=1-3, to=2-3]
                \arrow["x", from=1-1, to=1-2]
                \arrow["y"', from=2-1, to=2-2]
                \arrow["f"', from=1-1, to=2-1]
                \arrow["{l_s}"{pos = 0.4}, from=2-1, to=1-3]
                \arrow["{\prod\limits_{s\in S} p_s}", from=1-2, to=2-2, crossing over]
                \arrow["{\prod\limits_{s\in S} l_s}"{pos = 0.55}, from=2-1, to=1-2]
            \end{tikzcd}\]

            \par By the universal property of product, The diagram above (named $\msC$) commutes.
            To see this, we just need to chase two sub-diagrams:
            $\msC$ without vertice $X$ and $\msC$ without vertices $B_s$ and ${\prod\limits_{s \in S} B_s}$.
            \par Because the diagram drawed above commutes, ${\prod\limits_{s\in S} l_s}$ is the expected lifting.

        \qed
        
        \end{prf}

        \begin{prop}
            Given a model category $(\mcC, W, \Fib, \Cof)$, then its
            class of weak equivalences $W$ is closed under forming retracts in $\Arr(\mcC)$.
        \end{prop}
        \begin{prf}
            \par For every $w \in W$ and $f$ is a retract of $w$ in $\Arr(\mcC)$:
            \par First we consider the case $f \in \Fib$:

            \[\begin{tikzcd}
                A & X & A \\
                B & Y & B
                \arrow["{w}", from=1-2, to=2-2]
                \arrow["r_1", from=1-2, to=1-3]
                \arrow["r_2", from=2-2, to=2-3]
                \arrow["\id_A : \hspace*{2.77cm}"{marking}, "s_1", from=1-1, to=1-2]
                \arrow["\id_B : \hspace*{2.77cm}"{marking}, "s_2",from=2-1, to=2-2]
                \arrow["f", from=1-1, to=2-1]
                \arrow["f", from=1-3, to=2-3]
            \end{tikzcd}\]

            \par Factor $w$ through $(\Proj, \Inj) := (\Cof, W \cap \Fib)$:

            \[\begin{tikzcd}
                A & X & A \\
                A & Z & A \\
                B & Y & B
                \arrow["\id_B : \hspace*{2.77cm}"{marking}, "{s_2}", from=3-1, to=3-2]
                \arrow["{r_2}", from=3-2, to=3-3]
                \arrow["f \in \Inj"', from=2-3, to=3-3]
                \arrow["{w_2 \in \Inj}"', from=2-2, to=3-2]
                \arrow["{w_1 \in \Proj}"', from=1-2, to=2-2]
                \arrow["\id_A"', from=1-3, to=2-3]
                \arrow["\id_A"', from=1-1, to=2-1]
                \arrow["f \in \Inj"', from=2-1, to=3-1]
                \arrow["\id_A : \hspace*{2.77cm}"{marking}, "{s_1}", from=1-1, to=1-2]
                \arrow["{r_1}", from=1-2, to=1-3]
                \arrow["\id_A : \hspace*{2.77cm}"{marking}, "s", from=2-1, to=2-2]
                \arrow["r", from=2-2, to=2-3]
            \end{tikzcd}\]
            

            \par Where $s := w_1 \circ s_1$ and $r$ is a lifting of
            \begin{tikzcd}
                X & A \\
                Z & B
                \arrow["{r_2 \circ w_2}"', from=2-1, to=2-2]
                \arrow["{w_1 \in W \cap \Cof}"', from=1-1, to=2-1]
                \arrow["{\id_A \circ r_1}", from=1-1, to=1-2]
                \arrow["{f \in \Fib}", from=1-2, to=2-2]
            \end{tikzcd} (by 2-out-of-3, $w_1 \in W$).
            By $\Inj$ is closed under forming retracts in $\Arr(\mcC)$, $f \in \Inj = W \cap \Fib \subseteq W$.
            Now we consider the general case:

            \[\begin{tikzcd}
                A & X & A \\
                B & Y & B
                \arrow["{w}", from=1-2, to=2-2]
                \arrow["r_1", from=1-2, to=1-3]
                \arrow["r_2", from=2-2, to=2-3]
                \arrow["\id_A : \hspace*{2.77cm}"{marking}, "s_1", from=1-1, to=1-2]
                \arrow["\id_B : \hspace*{2.77cm}"{marking}, "s_2",from=2-1, to=2-2]
                \arrow["f", from=1-1, to=2-1]
                \arrow["f", from=1-3, to=2-3]
            \end{tikzcd}\]

            \par Factor $f$ through $(\Proj, \Inj) := (W \cap \Cof, \Fib)$:

            \[\begin{tikzcd}
                A & X & A \\
                C &   & C \\
                B & Y & B
                \arrow["\id_B : \hspace*{2.77cm}"{marking}, "{s_2}", from=3-1, to=3-2]
                \arrow["{r_2}", from=3-2, to=3-3]
                \arrow["{f_2 \in \Inj}"', from=2-3, to=3-3]
                \arrow["{f_1 \in \Proj}"', from=1-3, to=2-3]
                \arrow["{f_1 \in \Proj}"', from=1-1, to=2-1]
                \arrow["{f_2 \in \Inj}"', from=2-1, to=3-1]
                \arrow["{w}"', from=1-2, to=3-2]
                \arrow["\id_A : \hspace*{2.77cm}"{marking}, "{s_1}", from=1-1, to=1-2]
                \arrow["{r_1}", from=1-2, to=1-3]
                \arrow["f", curve={height=-12pt}, from=1-1, to=3-1]
                \arrow["f", curve={height=-12pt}, from=1-3, to=3-3]
            \end{tikzcd}\]

            \par Take the pushout $P$ of $f_1$ and $s_1$, by the universal property of pushout
            (apply it on $(s_2 \circ f_2, w)$ and $(\id_C, f_1 \circ r_1)$), we have a commutative diagram:

            \[\begin{tikzcd}
                A & X & A \\
                C & P & C \\
                B & Y & B
                \arrow["{\big \lrcorner}"{anchor=center, pos=0.125, rotate=180}, draw=none, from=2-2, to=1-1]
                \arrow["\id_B : \hspace*{2.77cm}"{marking}, "{s_2}", from=3-1, to=3-2]
                \arrow["{r_2}", from=3-2, to=3-3]
                \arrow["{f_2}"', from=2-3, to=3-3]
                \arrow["w_1"', from=1-2, to=2-2]
                \arrow["{f_1}"', from=1-3, to=2-3]
                \arrow["{f_1}"', from=1-1, to=2-1]
                \arrow["{f_2}"', from=2-1, to=3-1]
                \arrow["\id_A : \hspace*{2.77cm}"{marking}, "{s_1}", from=1-1, to=1-2]
                \arrow["{r_1}", from=1-2, to=1-3]
                \arrow["w"{pos=0.25}, curve={height=-12pt}, from=1-2, to=3-2]
                \arrow[from=2-1, to=2-2]
                \arrow["\id_C"'{pos=0.2}, curve={height=12pt}, from=2-1, to=2-3, crossing over]
                \arrow["{\E!\ \! r_3}", dotted, from=2-2, to=2-3]
                \arrow["{\E!\ \! w_2}"', dotted, from=2-2, to=3-2]
            \end{tikzcd}\]

            \par By $\Proj$ is closed under forming pushouts and $2\text{-}out\text{-}of\text{-}3$:
            
            \[\begin{tikzcd}
                A & X & A \\
                C & P & C \\
                B & Y & B
                \arrow["\id_B : \hspace*{2.77cm}"{marking}, "{s_2}", from=3-1, to=3-2]
                \arrow["{r_2}", from=3-2, to=3-3]
                \arrow["{f_2 \in \Inj}"', from=2-3, to=3-3]
                \arrow["w_2 \in W"', from=2-2, to=3-2]
                \arrow["{w_1 \in \Proj}"', from=1-2, to=2-2]
                \arrow["{f_1 \in \Proj}"', from=1-3, to=2-3]
                \arrow["{f_1 \in \Proj}"', from=1-1, to=2-1]
                \arrow["{f_2 \in \Inj}"', from=2-1, to=3-1]
                \arrow["\id_A : \hspace*{2.77cm}"{marking}, "{s_1}", from=1-1, to=1-2]
                \arrow["{r_1}", from=1-2, to=1-3]
                \arrow["\id_C : \hspace*{2.77cm}"{marking}, "s_3", from=2-1, to=2-2]
                \arrow["r_3", from=2-2, to=2-3]
                \arrow["{\big \lrcorner}"{anchor=center, pos=0.125, rotate=180}, draw=none, from=2-2, to=1-1]
            \end{tikzcd}\]

        \end{prf}

        \begin{lem}
            $\mbf{(retract\ argument)}$ For a pair of composable morphisms:
            $f : X \overset{i}{\longrightarrow} A \overset{p}{\longrightarrow}  Y$
            We have :
            \begin{enumerate}
                \item If $f$ has the $\mbf{left\ lifting\ property}$ against $p$, then $f$ is a $\mbf{retract}$ of $i$ in $\Arr(\mcC)$.
                \item If $f$ has the $\mbf{right\ lifting\ property}$ against $i$, then $f$ is a $\mbf{retract}$ of $p$ in $\Arr(\mcC)$.
            \end{enumerate}
        \end{lem}

        \begin{prf}
            Consider the first case, by $f$ has the $\mbf{left\ lifting\ property}$ against $p$, we have:
            \[\begin{tikzcd}
                X & A \\
                Y & Y
                \arrow["f"', from=1-1, to=2-1]
                \arrow["i", from=1-1, to=1-2]
                \arrow["p", from=1-2, to=2-2]
                \arrow["{\id_Y}"', from=2-1, to=2-2]
                \arrow["{\E\ g}"{marking, pos=0.45}, dotted, from=2-1, to=1-2]
            \end{tikzcd}\]
            Rearranging the diagram above, we get:
            \[\begin{tikzcd}
                X & X & X \\
                Y & A & Y
                \arrow["f", from=1-1, to=2-1]
                \arrow["\id_Y : \hspace*{2.85cm}"{marking}, "g"', from=2-1, to=2-2]
                \arrow["p"', from=2-2, to=2-3]
                \arrow["\id_X : \hspace*{2.85cm}"{marking}, "{\id_X}", from=1-1, to=1-2]
                \arrow["i", from=1-2, to=2-2]
                \arrow["{\id_X}", from=1-2, to=1-3]
                \arrow["f", from=1-3, to=2-3]
            \end{tikzcd}\]
            \par That is just what we need.
            And the second case follows by formally dual.
        \end{prf}

        \begin{intro}
            Motivation of $\mbf{small\ object\ argument}$:
            \par Given a $\mbf{Set}$ of morphisms $C \in \Mor(\mcC)$ in some category $\mcC$,
            a \emph{natual} question is how to factor a morphism $f : X \to Y$ through a relative $C$-cell complex followed by a $C$-injective morphism.
            $$ f : X \xrightarrow{\in C \text{-cell}}  \hat{X} \xrightarrow{\in C \Inj} Y $$

            \par A first approximation is attaching all \textbf{possible} $C$-cells to $X$,
            where possible $C$-cells is just an object in $C/f := (C \cap \Arr(\mcC))/f$:
            \[\begin{tikzcd}
                {\dom(c)} & X \\
                {\cod(c)} & Y
                \arrow["c"', from=1-1, to=2-1]
                \arrow[from=1-1, to=1-2]
                \arrow[from=2-1, to=2-2]
                \arrow["f", from=1-2, to=2-2]
            \end{tikzcd}\]
            
            \par The attaching pushout is:

            \[\begin{tikzcd}[row sep = 1.2cm]
                {\coprod\limits_{c \in C/f} \dom(c)} & X \\
                {\coprod\limits_{c\in C/f} \cod(c)} & X_1
                \arrow["{\coprod\limits_{c\in C/f} c}"', from=1-1, to=2-1]
                \arrow[from=1-1, to=1-2]
                \arrow["p_1"', from=2-1, to=2-2]
                \arrow["f_1", from=1-2, to=2-2]
                \arrow["{\Big \lrcorner}"{anchor=center, pos=0.125, rotate=180}, draw=none, from=2-2, to=1-1]
            \end{tikzcd}\]

            \par By the fact that the coproduct is over all commuting squres to $f$,
            we have commutative diagram:

            \[\begin{tikzcd}
                {{\coprod\limits_{c\in C/f} \dom(c)}} & X \\
                {{\coprod\limits_{c\in C/f} \cod(c)}} & Y
                \arrow[from=1-1, to=2-1]
                \arrow[from=2-1, to=2-2]
                \arrow[from=1-1, to=1-2]
                \arrow["f", from=1-2, to=2-2]
            \end{tikzcd}\]

            \par By the universal property of colimit we have that $f$ is factored through a relative $C$-cell complex $f_1$,
            the commutative diagram (1):

            \[\begin{tikzcd}
                {{\coprod\limits_{c\in C/f} \dom(c)}} & X \\
                {{\coprod\limits_{c\in C/f} \dom(c)}} & {X_1} \\
                {{\coprod\limits_{c\in C/f} \cod(c)}} & Y
                \arrow["\id"', from=1-1, to=2-1]
                \arrow[from=2-1, to=3-1]
                \arrow[from=3-1, to=3-2]
                \arrow["i_1", from=2-2, to=3-2]
                \arrow[from=1-1, to=1-2]
                \arrow["{f_1}", from=1-2, to=2-2]
                \arrow["{p_1}"', from=3-1, to=2-2]
                \arrow["{\Big \lrcorner}"{anchor=center, pos=0.125, rotate=180}, shift left=1, draw=none, from=2-2, to=1-1]
            \end{tikzcd}\]

            \par The map $p_1$ \textbf{almost} exhibits the expecting right lifting of $i_1$ against $C$,
            the failure of that to hold on is only the fact that the morphism ${{\coprod\limits_{c\in C/f} \dom(c)}} \to {X_1} $ is missing,
            which means that \textbf{not all} $c \in C/i_1$ could have a lifting.
            \par If we have $c \in C/i_1$ with $s : \cod(c) \to Y$ and $z : \dom(c) \to X_1$ be the morphism $c \to i_1$,
            we \textbf{don't always} have a $\hat{z} : \dom(c) \to X$ makes the diagram below commutes
            (equalently, $z$ factors through the $X \to X_1$).
            Only for those $c \in C/i_1$ such that $z$ factors through the $X \to X_1$,
            the lifting could be found.

            \[\begin{tikzcd}
                & X \\
                {\dom(c)} & {X_1} \\
                {\cod(c)} & Y
                \arrow["c"', from=2-1, to=3-1]
                \arrow["s"', from=3-1, to=3-2]
                \arrow["z"', from=2-1, to=2-2]
                \arrow["{i_1}", from=2-2, to=3-2]
                \arrow["{\hat{z}}", dotted, from=2-1, to=1-2]
                \arrow[from=1-2, to=2-2]
            \end{tikzcd}\]

            \par The $i_1$ is \textbf{almost} a $C$-injective morphism,
            the idea of the \textbf{small object argument} now is to fix it,
            make a \textbf{real} $C$-injective morphism. 
            The way it works is iterating the construction:
            \par Next factor $X_1 \to Y$ in the same way into $X_1 \to X_2 \to Y$.
            Since relative cell complex is closed under transfinite composition,
            for $\alpha$ an ordinal, at stage $X_{\alpha}$, the $X \to X_{\alpha}$
            is still a relative $C$-cell complex.
            \par Intuitively,
            the failure of the $X_{\alpha} \to Y$ to be a $C$-injective morphism becomes smaller and smaller,
            for the condition of the existence of the factorization
            $$\dom(c) \to X_{\alpha} \longmapsto \dom(X) \to X_{\beta} \to X_{\alpha} \hspace*{1cm} \text{(where  $\beta < \alpha$)}$$
            is less and less (intuitively) as the $X_\beta$ grow larger and larger.
            \par The concept of \textbf{small object} is just what makes this intuition precise and finishes the small object argument.
            But now we just need the following simple version:
        \end{intro}
        \begin{defn}
            For $\mcC$ a category and $C \subseteq \Mor(\mcC)$ a \textbf{subset} of morphisms in $\mcC$,
            say that $C$ have \textbf{small domains} if there exists an ordinal $\alpha$
            such that for any $c \in C$ and for any relative $C$-cell complex given by \textbf{transfinite composition of length $\alpha$}
            $$f : X \to X_1 \to X_2 \to \cdots \to X_{\beta} \to \cdots \to X_\alpha$$
            any morphism $\dom(c) \to \hat{X}$ factors through a stage $X_{\beta} \to X_{\alpha}$ of some ordinal $\beta < \alpha$:
            \[\begin{tikzcd}
                & {X_{\beta}} \\
                {\dom(c)} & {\hat{X}}
                \arrow[from=2-1, to=2-2]
                \arrow[from=1-2, to=2-2]
                \arrow[from=2-1, to=1-2]
            \end{tikzcd}\]

        \end{defn}

        \begin{prop}
            Let $\mcC$ be a locally small category with all small colimits.
            If a set $ C \subseteq  \Mor(\mcC)$ of morphisms has \textbf{small domains},
            then every morphism $f : X \rightarrow Y$ in $\mcC$ factors through a relative $C$-cell complex followed by a $C$-injective morphism:
            $$ f : X \xrightarrow{\in C \text{-cell}}  \hat{X} \xrightarrow{\in C \Inj} Y $$
        \end{prop}

        \begin{note}
            The locally smallness above is to restrict the size of $C/f := (C \cap \Arr(\mcC))/f$.
        \end{note}


    \subsection{Homotopy}

        We discuss how the concept of homotopy is abstractly realized in model categories

        \begin{defn}
            Let $\mcC$ be a model category, $X \in \Obj(C)$.\\
            \begin{enumerate}
                \item A \textbf{path space object} $\Path(X)$ for $X$ is a factorization of the diagonal $\Delta_X : X \to X \times X$ as
                    $$ \Delta_X : X \xrightarrow[\in W]{i} \Path(X) \xrightarrow[\in \Fib]{(p_0,p_1)} X \times X $$
                \item A \textbf{cylinder object} $\Cyl(X)$ for $X$ is a factorization of the codiagonal $\nabla_X : X \sqcup X \to X$ as
                    $$ \nabla_X : X \sqcup  X \xrightarrow[\in \Cof]{(i_0, i_1)} \Cyl(X) \xrightarrow[\in W]{p} X $$
            \end{enumerate}
        \end{defn}



    
    
\end{document}